\documentclass{article}

\usepackage{iclr2022_conference,times}

\usepackage{hyperref}
\usepackage{url}
\usepackage{graphicx}

\iclrfinalcopy
\usepackage{tikz}
\usepackage{comment}
\usepackage{amsmath,amssymb} 
\usepackage{color}
\usepackage{booktabs}
version https://git-lfs.github.com/spec/v1
oid sha256:90473c4d0542070db244cea73ef962d6cddc5b2a746757e6a40ddf5fdfb90ba9
size 12284

\usepackage{xcolor}
\usepackage{soul}
\usepackage{wrapfig}
\setcitestyle{numbers,square,citesep={,},aysep={,},yysep={;}}


%peilin
\newcommand{\cv}{\textit{VLM}_{\textbf{V}}\ }
\newcommand{\ct}{\textit{VLM}_{\textbf{T}}\ }
\newcommand{\peilin}[1]{\textcolor{blue}{Peilin: #1}}
\newcommand{\nihal}[1]{\textcolor{red}{Nihal: #1}}
\newcommand{\steve}[1]{\textcolor{brown}{Steve: #1}}

% credit: https://arxiv.org/pdf/2107.01294.pdf
\definecolor{prompt_color}{HTML}{EEEEEE}
\definecolor{soft_prompt}{HTML}{d9d2e9}
\definecolor{attribute_color}{HTML}{c9daf8}
\definecolor{object_color}{HTML}{ead1dc}
\newcommand{\prompt}[2][prompt_color]{{\sethlcolor{#1} \hl{\texttt{#2}}}}
\newcommand{\softprompt}[2][soft_prompt]{{\sethlcolor{#1} \hl{\texttt{#2}}}}
\newcommand{\attribute}[2][attribute_color]{{\sethlcolor{#1} \hl{\texttt{#2}}}}
\newcommand{\object}[2][object_color]{{\sethlcolor{#1} \hl{\texttt{#2}}}}

\newcommand\cspemoji{\raisebox{-1pt}{\includegraphics[width=0.8em]{figures/puzzle.png}}}
\newcommand{\cspname}{$\mathbb{CSP}$\ }
\newcommand{\cspnamenospace}{$\mathbb{CSP}$}
\usepackage{paralist}

\usepackage[frozencache=true, cachedir=minted-cache]{minted} 
\title{Learning to Compose Soft Prompts for\\ Compositional Zero-Shot Learning}


\author{Nihal V. Nayak$^{*}$,\ \ Peilin Yu\thanks{Equal contribution}\ ,\ \  Stephen H. Bach\\
Department of Computer Science\\
Brown University\\
Providence, RI 02906, USA \\
\texttt{\{nnayak2, pyu12, sbach\}@cs.brown.edu} \\
}



\newcommand{\fix}{\marginpar{FIX}}
\newcommand{\new}{\marginpar{NEW}}

\begin{document}


\maketitle
\begin{abstract}
We introduce compositional soft prompting (\cspnamenospace), a parameter-efficient learning technique to improve the zero-shot compositionality of large-scale pretrained vision-language models (VLMs) without the overhead of fine-tuning the entire model.
VLMs can represent arbitrary classes as natural language prompts in their flexible text encoders but they underperform state-of-the-art methods on compositional zero-shot benchmark tasks.
To improve VLMs, we propose a novel form of soft prompting.
We treat the attributes and objects that are composed to define classes as learnable tokens of vocabulary and tune them on multiple prompt compositions.
During inference, we recompose the learned attribute-object vocabulary in new combinations and show that \cspname outperforms the original VLM on benchmark datasets by an average of 14.7 percentage points of accuracy.
\cspname also achieves new state-of-the-art accuracies on two out of three benchmark datasets, while only fine-tuning a small number of parameters.
Further, we show that \cspname improves generalization to higher-order attribute-attribute-object compositions and combinations of pretrained attributes and fine-tuned objects.
The code is available at \href{https://github.com/BatsResearch/csp}{https://github.com/BatsResearch/csp}.
\end{abstract}
\section{Introduction}

In the most simple case, time series forecasting deals with a scalar
time-varying signal and aims to predict or forecast its values in the near future; for example, countless applications in finance, healthcare, production automatization, etc. \cite{cao2018brits,sagheer2019time,sezer2020financial} can benefit from an accurate forecasting solution.
Often not just a single scalar signal is of interest, but multiple at once,
and further time-varying signals are available and even \textsl{known for the future}.
For example, suppose one aims to forecast the energy consumption of a house, it likely depends on the social time that one seeks to forecast for (such as the next hour or day), and also on features of these time points (such as weekday, daylight, etc.), which are known already for the future. This is also the case in model predictive control \cite{camacho2013model}, where one is interested
to forecast the expected value realized by some planned action, then this action is also known at the time of forecast.
More generally,
time series forecasting, nowadays deals with quadruples $(x,y,x',y')$
of known past predictors $x$, known past targets $y$, known future predictors $x'$
and sought future targets $y'$. 

\begin{figure}[ht]
\centering

\includegraphics[width=0.4\columnwidth]{figs/ts_ps.png}
\caption{General time series setting illustrating the quadruples $(x,y,x',y')$ denoting the \textsl{past predictors}, \textsl{past targets}, \textsl{future predictors} and \textsl{future targets} respectively. Given the history information $(x, y)$ until time $t = T$ and the future predictors $(x')$ for the next $\tau$ time steps, time series forecasting predicts the target $y'$ from $t = T+1, \dots, \tau$ time steps. In the figure, $O$ and $M$ represents the respective channels of the targets and the predictors.}
\label{fig:ts_ps}
\end{figure}

Time series problems can often be addressed by methods developed initially
for images, treating them as 1-dimensional images. Especially for
time-series classification many typical time series encoder architectures
have been adapted from models for images \cite{wang2017time,ZOU201939}. 
Time series forecasting then is closely related to image outpainting \cite{wang2019srn},
the task to predict how an image likely extends to the left, right, top or bottom,
as well as to the more well-known task of image segmentation,
where for each input pixel, an output pixel has to be predicted, whose channels
encode pixel-wise classes such as vehicle, road, pedestrian say for road scenes.
Time series forecasting combines aspects from both problem settings:
information about targets from shifted positions (e.g., the past targets $y$ as
 in image outpainting) and
information about other channels from the same positions (e.g., the future predictors $x'$
 as in image segmentation).
One of the most successful, principled architectures for the image segmentation
task are U-Nets introduced in \cite{ronneberger2015u}, an architecture that successively downsamples/coarsens
its inputs and then upsamples/refines the latent representation with
deconvolutions also using the latent representations of the same detail level,
tightly coupling down- and upsampling procedures and thus yielding latent
features on the same resolution as the inputs. 


Following the great success in Natural Language Processing (NLP) applications, attention-based, esp. transformer-based
architectures \cite{vaswani2017attention} that model pairwise interactions
between sequence elements have been recently adapted for
time series forecasting. One of the significant
challenges, is that the length of the time series, are often one or two magnitudes of order larger than the (sentence-level) NLP problems. 

Plenty of approaches aim to mitigate the quadratic complexity $O(T^2)$ in
the sequence/time series length $T$ to at most $O(T\log T)$.
For example, the Informer architecture
\cite{zhou2020informer}, adapts the transformer with a sparse attention
mechanism and a successive downsampling/coarsening of the past time series. As in the original transformer, only the coarsest representation is fed
into the decoder. Possibly to remedy the loss in resolution by this procedure,
the Informer feeds its input a second time into the decoder network, this time
without any coarsening. 

While forecasting problems share many commonalities with image segmentation
problems, transformer-based architectures like the Informer do not
involve coupled down- and upscaling procedures to yield predictions
on the same resolution as the inputs. 
Thus, we propose a novel Y-shaped architecture that
\begin{enumerate}
\item Couples downscaling/upscaling to leverage both, coarse and fine-grained
       features for time series forecasting,
\item Combines the coupled scaling mechanism with sparse attention modules to capture long-range effects on all scale levels, and
\item Stabilizes encoder and decoder stacks by reconstructing the recent past.
\end{enumerate}





\section{Related Work}
\label{sec:related-works}



\paragraph{DP-ERM.}
Differentially Private Empirical Risk Minimization was first studied by
\citet{chaudhuri2011Differentially}, using output perturbation (adding noise
to the solution of the non-private ERM problem) and objective perturbation
(adding noise to the ERM objective itself).
\citet{bassily2014Private} then proposed DP-SGD and proved its
near-optimality. %
\citet{wang2017Differentially} obtained faster convergence rates using a
DP version of the SVRG algorithm
\citep{johnson2013Accelerating,xiao2014Proximal}.
DP-SGD has
become the standard approach to DP-ERM.
In our work, we show that coordinate-wise updates can have lower sensitivity
than DP-SGD updates and propose a DP-CD algorithm achieving competitive
results.
A private variant of the Frank-Wolfe algorithm (DP-FW) was also
proposed to solve \emph{constrained} DP-ERM problems
\citep{talwar2015Nearly}.  Although these algorithms achieve a good
privacy-utility trade-off in theory, we are not aware of any empirical
evaluation.
DP-FW algorithms access gradients indirectly through a linear
optimization oracle over a constrained set.
Restricting to a constrained set is not necessary in DP-CD, allowing its use for a different family of problems.

\paragraph{DP-SCO.} Recent work has also studied algorithms and
utility guarantees for
stochastic convex optimization under differential privacy constraints, a
problem very similar to DP-ERM. \citet{bassily2019Private} \citep[following work
from][]{hardt2016Train,bassily2020Stability} extended results known
for DP-ERM to this setting, showing that the population risk of DP-SCO
is asymptotically equivalent to the one of non-private SCO. Efficient
algorithms for
DP-SCO
were proposed by \citet{feldman2020Private,wang2022Differentially},
and \citet{asi2021Private,bassily2021NonEuclidean} studied stochastic
variants of DP-FW. As detailed by
\citet{dwork2015Preserving,bassily2016Algorithmic,jung2021New} results
from DP-ERM can be converted to DP-SCO.




\paragraph{Coordinate descent.}
Coordinate descent (CD) algorithms have a long history in optimization.
\citet{Luo_Tseng1992,Tseng01,Tseng_Yun09} have shown convergence results for
(block) CD algorithms for nonsmooth optimization.
\citet{Nesterov12} later proved a global non-asymptotic $1/k$ convergence
rate for CD with random choice of coordinates for a convex, smooth objective.
Parallel, proximal variants were developed by
\citet{richtarik2014Iteration,fercoq2014Accelerated}, while
\citet{hanzely2018SEGA} further considered non-separable non-smooth parts.
\citet{shalev-shwartz2013Stochastic} introduced Dual CD algorithms
for smooth ERM, showing performance similar to SVRG.
We refer to \citet{wright2015Coordinate} and \citet{shi2017Primer} for
detailed reviews on CD.
Inexact CD was studied by \citet{tappenden2016Inexact}, but their analysis
requires updates not to increase the objective, which is hardly compatible
with DP.
We obtain tighter results for inexact CD with noisy gradients
(see Remark~\ref{rmq:improvement-inexact-coordinate-descent}).



\paragraph{Private coordinate descent.}
\citet{damaskinos2021Differentially} introduced a CD method to privately solve
the dual problem associated with generalized linear models with $\ell_2$
regularization. Dual CD is tightly related to SGD, as each
coordinate in the dual is associated with one data point.
The authors briefly mention the possibility of performing primal coordinate
descent but discard it on account of the seemingly large sensitivity of its
updates.
We show that primal DP-CD is in fact quite effective, and can be used to solve more general problems than considered by \citet{damaskinos2021Differentially}.
Primal CD was successfully used by \citet{bellet2018Personalized} to privately learn personalized models from decentralized datasets.
For the smooth objective they consider,
each coordinate depends only on a subset of the full dataset, which directly
yields low coordinate-wise sensitivity updates.
In contrast, we introduce a general algorithm for composite DP-ERM, for which
a novel utility analysis was required.



\section{Preliminaries}
\label{sec:preliminaries}


In this section, we introduce important technical notions that will be used
throughout the paper.

\paragraph{Norms.}
We start by defining two conjugate norms that will be crucial in our analysis,
for they allow to keep track of coordinate-wise quantities.
Let $\scalar{u}{v} = \sum_{j=1}^p u_i v_i$ be the Euclidean dot product, let $M = \diag(M_1, \dots, M_p)$ with $M_1, \dots, M_p > 0$, and
\begin{align*}
  \norm{w}_M = \sqrt{\scalar{Mw}{w}}\enspace,\quad\quad\quad
  \norm{w}_{M^{-1}} = \sqrt{\scalar{M^{-1}w}{w}} \enspace.
\end{align*}
When $M$ is the identity matrix $I$, the $I$-norm $\norm{\cdot}_I$ is the standard $\ell_2$-norm $\norm{\cdot}_2$.

\paragraph{Regularity assumptions.}
We recall classical regularity assumptions along with ones
specific to the coordinate-wise setting.
We denote by $\nabla f$ the gradient of
a differentiable function $f$, and by $\nabla_j f$ its $j$-th coordinate.
We denote by $e_j$ the $j$-th vector of $\RR^p$'s canonical basis.

\textit{Convexity:} a differentiable function $f : \RR^p
  \rightarrow \RR$ is convex if
for all $v, w \in \RR^p$,
$f(w) \ge f(v) + \scalar{\nabla f(v)}{w - v}$.

\textit{Strong convexity:} a differentiable function $f : \RR^p \rightarrow
  \RR$ is
$\mu_M$-strongly-convex \wrt the norm $\smash{\norm{\cdot}_M}$ if
for all $v, w \in \RR^p$,
$f(w) \ge f(v) + \scalar{\nabla f(v)}{w - v} + \frac{\mu_M}{2}\norm{w - v}_M^2$.
The case $M_1=\cdots=M_p=1$ recovers standard $\mu_I$-strong convexity \wrt
the $\ell_2$-norm.

\textit{Component Lipschitzness:} a function $f : \RR^p \rightarrow \RR$
is
$L$-component-Lipschitz for $L = (L_1,\dots,L_p)$ with $L_1,\dots,L_p > 0$ if
for all $w \in \RR^p$, $t \in \RR$ and $j \in [p]$,
$\abs{f(w + t e_j) - f(w)} \le L_j \abs{t}$.
It is $\Lambda$-Lipschitz if for all $v, w \in \RR^p$,
$\abs{f(v) - f(w)} \le \Lambda \norm{v - w}_2$.


\textit{Component smoothness:} a differentiable function $f : \RR^p
  \rightarrow \RR$ is
$M$-component-smooth for $M_1,\dots,M_p > 0$ if
for all $v, w \in \RR^p$,
$f(w) \le f(v) + \scalar{\nabla f(v)}{w - v} + \frac{1}{2}\norm{w - v}_{M}^2$.
When $M_1=\dots=M_p=\beta$, $f$ is said to be $\beta$-smooth.

The above component-wise regularity hypotheses are not restrictive:
$\Lambda$-Lipschitzness
implies $(\Lambda, \dots, \Lambda)$-component-Lipschitzness and
$\beta$-smoothness implies $(\beta, \dots, \beta)$-component-smoothness.
Yet, the actual component-wise constants of a function can be much
lower than what can be deduced from their global counterparts.
This will be crucial for our analysis and in the performance of DP-CD.

\begin{remark}
  \label{rmq:constrained-regularity-assumptions}
  When $\psi$ is the characteristic function of a convex set (with separable
  components), the regularity assumptions only need to hold on this
  set. This allows considering problem~\eqref{eq:dp-erm} with a smooth
  objective under box-constraints.
\end{remark}




\paragraph{Differential privacy (DP).}

Let $\cD$ be a set of datasets and $\cF$ a set of possible outcomes.
Two datasets $D, D' \in \cD$ are said \textit{neighboring}
(denoted by $D \sim D'$) if they differ on at most one element.

\begin{definition}[Differential Privacy, \citealt{dwork2006Differential}]
  A randomized algorithm
  $\cA : \mathcal D
    \rightarrow \mathcal F$ is $(\epsilon, \delta)$-differentially private if,
  for all neighboring datasets $D, D' \in \mathcal D$ and all
  $S \subseteq \mathcal F$ in the range of $\cA$:
  \begin{align*}
    \prob{\cA(D) \in S} \le \exp(\epsilon) \prob{\cA(D') \in S} + \delta \enspace.
  \end{align*}
\end{definition}
The value of a function $h: \mathcal D \rightarrow \mathbb R^p$ can be privately
released using the Gaussian mechanism, which adds centered Gaussian noise
to $h(D)$ before releasing it \citep{dwork2013Algorithmic}.
The scale of the noise is calibrated to the sensitivity $
  \Delta(h)
  = \sup_{D \sim D'} \norm{h(D) - h(D')}_2$ of $h$.
In our setting, we will perturb coordinate-wise gradients: we denote by
$\Delta(\nabla_j \ell)$ the sensitivity of the $j$\nobreakdash-th coordinate
of gradient of the loss function $\ell$ with respect to the data.
When $\ell(\cdot;d)$ is $L$-component-Lipschitz for all $d\in\mathcal{X}$, upper
bounds on these sensitivities are readily available: we have
$\Delta(\nabla_j\ell) \le 2L_j$ for any $j\in[p]$ (see \Cref{sec:lemma-sensitivity}).
The following quantity, relating the coordinate-wise sensitivities of gradients
to coordinate-wise smoothness is central in our analysis:
\begin{align}
  \label{eq:delta-lipschitz-norm}
  \Delta_{M^{-1}}(\nabla \ell)
  = \Big(\sum_{j=1}^p \frac{1}{M_j} \Delta (\nabla_j\ell)^2\Big)^{\frac{1}{2}}
  \leq \! 2 \norm{L}_{M^{-1}}\enspace.
\end{align}
In this paper, we consider the classic central model of DP, where a trusted
curator has access to the raw dataset and releases a model trained on this
dataset\footnote{In fact, our privacy guarantees hold even if all
  intermediate iterates are released (not just the final model).}.


version https://git-lfs.github.com/spec/v1
oid sha256:773f8cd61d0a6d6593bcf6a674cb40fe521472ae453fc90bb9522717b1539912
size 18018

\input{sections/experiment}
\input{sections/attribute_attribute_object}
\input{sections/pretrained_finetune_vocab}
\section{Conclusion}
\label{conclusion}
We present an experimentally-verified simulation framework that can be used to accurately predict the deformations of a pneumatically actuated fish tail with a flexible spine.
Our pipeline can accurately learn material parameters from a quasi-static data sets without having to do expensive and time-consuming material testing. It also eliminates the need to do manual tuning of material constants to get accurate simulation results. The parameters we found are not only within typical range of measured material parameters for our materials, but can be used to successfully predict the behavior of dynamic experiments for different pressure actuation amplitudes and frequencies to within $3\%$ positional error normalized to a actuator length of \SI{10}{cm}. Although we use an isotropic corotated material, which is linear elastic, we find that this model is more sufficient to model large deformations on average giving acceptable displacement results for our engineering application. In these experiments, the damping of the material and the hydrodynamic effects are found to be negligible. This is because the actuation pressures used dominate the deformation compared to losses and hydrodynamic pressure. 

We show a data-driven approach can be used to do simple prediction on a useful performance metric such as thrust force given a suitable hardware setup. However, more work is needed to produce a more robust thrust predictor if the morphology of the actuator changes substantially. We claim that for small design changes such as the choice of silicone or the number of internal chambers this framework can be used to quickly assess the relative merits of each design with a relatively sparse data set of approximately 30 types of different actuation signals.

Our aim is to further progress towards a systematic method by which soft roboticists can simulate and optimize their designs and controllers, whether they be soft fish, manipulators, or other flavors of soft robots. A fast and physically-verified co-optimization method of design and control is the goal.


\textbf{Authors’ Note.} The first two authors contributed equally. 
Co-first authors can prioritize their names when adding this paper’s reference to their resumes.


\section*{Acknowledgments}
We thank Andrew Delworth and Elise Carman for helping us annotate the AAO-MIT-States dataset.
We appreciate the comments and advice from Cristina Menghini, Wasu Piriyakulkij and  Zheng-Xin Yong on our drafts. 
This material is based on research sponsored by Defense Advanced Research Projects Agency (DARPA)
and Air Force Research Laboratory (AFRL) under
agreement number FA8750-19-2-1006. The U.S. Government is authorized to reproduce and distribute
reprints for Governmental purposes notwithstanding
any copyright notation thereon. The views and conclusions contained herein are those of the authors and
should not be interpreted as necessarily representing
the official policies or endorsements, either expressed or implied, of Defense Advanced Research Projects Agency (DARPA) and Air Force Research Laboratory
(AFRL) or the U.S. Government. We gratefully acknowledge support from Google and Cisco. Disclosure:
Stephen Bach is an advisor to Snorkel AI, a company that provides software and services for weakly supervised machine learning.


\bibliographystyle{abbrvnat}
\bibliography{egbib}
\clearpage
\appendix
\input{appendices/appendices_list}
\end{document}