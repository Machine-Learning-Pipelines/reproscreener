\znote{May want to cite \citet{ghorbani2019investigation}}

To further understand the optimization process, we conducted a grid search of different optimization schemes by varying the step sizes and network depth, as well as comparing between traditional batch normalization and projection off E[x]. Models optimized by projecting off of E[x] were generally found to converge to full training accuracy with fewer steps than models trained with normal SGD. In fact, for most step sizes $>$ 0.01, both shallow and deep fully connected networks trained with the projection method converged in even fewer steps than when trained with batch normalization. Additionally, the optimal step size for the projection method is greater than the optimal step size for normally trained models as well as BN-trained models for shallow networks. As network depth increases, variation in convergence speed also increases so it is more difficult to make conclusive generalizations, but it can be observed that the optimal projection step size grows closer to that of the control and BN-trained models. However, when compared to batch normalization, the performance of projection-trained models is less stable against varying learning rates. Additionally, batch normalization leads to slightly improved generalization over the projection method, although both models perform better than normally trained FC6 networks on test data. On the smaller FC3 network, projection tends to result in slightly worse test performance than the control models.

\begin{figure}[th]
    \centering
    \begin{subfigure}[b]{0.49\textwidth}
        \centering
        \captionsetup{justification=centering}
        \includegraphics[width=\textwidth]{fig/proj_bn_learningCurves.png}
        \caption{TBD}
        \label{fig:overlap_approx}
    \end{subfigure}
    \hfill
    \begin{subfigure}[b]{0.49\textwidth}
        \centering
        \captionsetup{justification=centering}
        \includegraphics[width=\textwidth]{fig/proj_convergence_speeds.png}
        \caption{TBD}
        \label{fig:eigenval_approx}
    \end{subfigure}
    \caption{TBD}
    \label{fig:three graphs}
\end{figure}
