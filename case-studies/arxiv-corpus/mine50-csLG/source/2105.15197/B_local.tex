\section{Characterization of key parameters}\label{sec:localization}

\subsection{Additional example}

We present an additional example useful in commercial applications where an analyst has access to data on household spending behavior. See \cite{chernozhukov2019demand} for a motivating economic model and further interpretation.

\begin{example}[Heterogeneous average derivative estimated by lasso]\label{ex:deriv}
Let $Y$ be share of household expenditure on some good. Let $W=(D,V,X)$ concatenate log price of the good $D$, covariate of interest $V$ such as household size, and other covariates $X$ such as log prices of other goods and log total household expenditure. Let $\gamma_0(d,v,x)=E(Y \mid D=d,V=v,X=x)$ be a function estimated by lasso. The heterogeneous average derivative is
$$
\textsc{DERIV}(v)=E\left\{\frac{\partial }{\partial d} \gamma_0(D,V,X) \mid V=v\right\}=\lim_{h\rightarrow 0}E\left\{\ell_h(V)\frac{\partial }{\partial d} \gamma_0(D,V,X)\right\},$$
where
$
\ell_h(V)=(h\omega)^{-1}K\left\{(V-v)/h\right\}$, $\omega=E [h^{-1} K\left\{(V-v)/h\right\}]
$, and $K$ is a bounded and symmetric kernel that integrates to one.
\end{example}
The heterogeneous average derivative is defined with respect to some interpretable, low dimensional characteristic $V$ such as household size \cite{abrevaya2015estimating}. The same functional without the localization $\ell_h$ is the classic average derivative. 

\subsection{Riesz representers}

We begin by deriving explicit expressions for Riesz representer $\alpha_0$. Recall that the unique minimal Riesz representer $\alpha_0^{\min}$ is the projection of any valid Riesz representer $\alpha_0$ onto $\Gamma$. The explicit expressions for $\alpha_0$ help to articulate simple sufficient conditions for existence of $\alpha_0^{\min}$.

\begin{lemma}\label{lemma:RR_exists}
In Examples~\ref{ex:CATE},~\ref{ex:RDD},~\ref{ex:elasticity}, and~\ref{ex:deriv}, the minimal representer $\alpha_0^{\min}(w)$ can be obtained by projecting the following Riesz representer $\alpha_0(w)$ onto $\Gamma$.
\begin{enumerate}
    \item Example~\ref{ex:CATE}. Denote the propensity score $\pi_0(v,x)=\text{\normalfont pr}(D=1\mid V=v,X=x)$. Then
    $$
    \alpha_0(d,v,x)=\ell_h(v)\left\{\frac{d}{\pi_0(v,x)}-\frac{1-d}{1-\pi_0(v,x)}\right\}.
    $$
    Hence the minimal Riesz representer $\alpha^{\min}_0$ exists if $\pi_0(v,x)$ is bounded away from zero and one.
    \item Example~\ref{ex:RDD}. Denote the Riesz representer for the first term by $\alpha_0^{+}$ and the Riesz representer for the second term by $\alpha_0^{-}$. Then
    $$
    \alpha^+_0(d,x)=\ell_h^{+}(d),
    \quad 
     \alpha^-_0(d,x)=\ell_h^{-}(d).
    $$
    Hence the minimal Riesz representer $\alpha^{\min}_0$ exists.
    \item Example~\ref{ex:elasticity}. Denote the density $f(d,x)$. If $f(d\mid x)$ vanishes for each $d$ in the boundary of the support of $D$ given $X=x$ almost everywhere then
    $$
    \eta_0(d,x)
    =-\partial_d \log f(d \mid x).$$ Hence the minimal representer $\eta_0^{\min}$ exists if $-\partial_d \log f(d \mid x)$ is bounded above. Subsequently, $\alpha_0^{\min}$ is the solution $\alpha$ to
    $$
    \eta_0^{\min}(d,x)=E\{\alpha(X,Z) \mid D=d,X=x\}.
    $$
    \item Example~\ref{ex:deriv}. Denote the density $f(d,v,x)$. If $f(d\mid v,x)$ vanishes for each $d$ in the boundary of the support of $D$ given $(V,X)=(v,x)$ almost everywhere then
    $$
    \alpha_0(d,v,x)
    =-\ell_h(v)\partial_d \log f(d \mid v,x).$$
\end{enumerate}
\end{lemma}


Next, we characterize key quantities $(\bar{Q},\bar{\sigma},\bar{\alpha},\bar{\alpha}')$ and moments $(\sigma,\kappa,\zeta)$ that appear in Theorems~\ref{thm:dml} and~\ref{thm:var}. Recall that
$$
E[\{Y-\gamma_0(W)\}^2 \mid W]\leq \bar{\sigma}^2, \quad  \|\hat{\alpha}_{\ell}\|_{\infty}\leq\bar{\alpha}',
$$
so $\bar{\sigma}$ is a constant if noise has bounded variance and $\bar{\alpha}'$ can be imposed by trimming the algorithm $\hat{\alpha}_{\ell}$. We therefore focus on $(\bar{Q},\bar{\alpha})$ and $(\sigma,\kappa,\zeta)$.

We consider two cases: global functionals, with weighting $\ell_h$ that has a fixed bandwidth; and local functionals, with weighting $\ell_h$ that depends on some vanishing bandwidth $h\rightarrow 0$. To lighten notation, let $\|X\|_{\text{pr},q}=\{E(|X|^q)\}^{1/q}$.

\subsection{Global functionals}

\begin{lemma}[Oracle moments for global functionals]\label{lemma:global}
Suppose the weighting is bounded,  namely $\ell_h \leq C<\infty$, and $\Gamma\subset \mathbb{L}_2$. Suppose $\alpha_0^{\min}$ exists. Further suppose there exist $(c,\tilde{c},\bar c)$ bounded away from zero and above such that the following conditions hold.
\begin{enumerate}
    \item Control of representer moments. For $q=3,4$
    $$
\|\alpha_0^{\min}\|_{\text{\normalfont pr},q}\leq c \left( \|\alpha_0^{\min}\|^2_{\text{\normalfont pr},2} \vee 1\right).
$$
    \item Bounded moments. For $q=2,3,4$,
    $$
    U_1=m(W,\gamma_0)-E\{m(W,\gamma_0)\},\quad \|U_1\|_{\text{\normalfont pr},q}\leq \bar{c}.
    $$
    \item Bounded heteroscedasticity. For $q=2,3,4$,
    $$
    U_2=Y-\gamma_0(W),\quad \tilde{c}\leq \|U_2 \mid W\|_{\text{\normalfont pr},q}\leq \bar{c}.
    $$
\end{enumerate}
Then
    $$
    \tilde{c}  \bar{M}   \leq \sigma  \leq  \bar c  \sqrt{1+ \bar{M}^2}, 
 \quad \kappa,\zeta  \leq  \bar c   \{1+ c (\bar{M}^2 \vee 1) \}.
    $$
In summary,
$$
\frac{\kappa}{\sigma} \lesssim \sigma \asymp \bar{M} < \infty,\quad \kappa, \zeta\lesssim \bar{M}^2 < \infty.
$$
\end{lemma}

Clearly Assumption~\ref{assumption:cont} depends on the functional of interest. Towards a characterization of $\bar{Q}$ and $q$ in Examples~\ref{ex:elasticity} and~\ref{ex:deriv}, we prove the following technical lemma.

\begin{lemma}[A weak reverse Poincare inequality]\label{lemma:rp}
Assume that $f(d\mid x)$ vanishes for each $d$ in the boundary of the support of $D$ given $X=x$ almost everywhere. Next assume the following restrictions on $\Gamma \subset \mathbb{L}_2$.
\begin{enumerate}
    \item Each $\gamma$ in $\Gamma$ is twice continuously differentiable.
    \item For each $\gamma$ in $\Gamma$, $\|k_{\gamma}\|_{\text{ \normalfont pr},2}<\infty$ where
    $$
    k_{\gamma}(d,x)=\{-\partial_d \log f(d\mid x)\}\{\partial_d\gamma(d,x)\}-\partial_d^2 \gamma(d,x).
    $$
\end{enumerate}
Then 
$$
E[\{\partial_d\gamma(D,X)\}^2]\leq \|k_{\gamma}\|_{\text{ \normalfont pr},2} [E \{\gamma(D,X)^2\}]^{1/2}.
$$
Furthermore, $\sup_{\gamma \in \Gamma} \|k_{\gamma}\|_{\text{ \normalfont pr},2}<\infty$ if either of the following conditions hold.
\begin{enumerate}
    \item $\|\partial_d \log f(D\mid X)\|_{\text{ \normalfont pr},2}<\infty$ and for all $\gamma$ in $\Gamma$, $\|\partial_d \gamma\|_{\infty}<\infty$ and $\|\partial^2_d \gamma\|_{\text{ \normalfont pr},2}<\infty$;
    \item $-\partial_d \log f(d\mid x)$ is bounded above and for all $\gamma$ in $\Gamma$,  $\|\partial_d \gamma\|_{\text{ \normalfont pr},2}$ and $\|\partial^2_d \gamma\|_{\text{ \normalfont pr},2}<\infty$.
\end{enumerate}
\end{lemma}

With this technical lemma, we return to the characterization of $\bar{Q}$ and $q$ across examples.

\begin{lemma}[Mean square continuity for global functionals]\label{lemma:cont}
Suppose the weighting is bounded,  namely $\ell_h \leq C<\infty$, and $\Gamma\subset \mathbb{L}_2$. The following conditions are sufficient for $\bar{Q}<\infty$ with $q=1$ in Examples~\ref{ex:CATE} and~\ref{ex:RDD}, and for $\bar{Q}<\infty$ with $q=1/2$ in Examples~\ref{ex:elasticity} and~\ref{ex:deriv}.
\begin{enumerate}
    \item Example~\ref{ex:CATE}. $\pi_0(v,x)$ is bounded away from zero and one.
    \item Example~\ref{ex:RDD}. The bandwidth is fixed.
    \item Example~\ref{ex:elasticity}. $f(d\mid x)$ vanishes for each $d$ in the boundary of the support of $D$ given $X=x$ almost everywhere. $-\partial_d \log f(d \mid x)$ is a bounded above, and $\Gamma$ consists of functions
    $\gamma$ that are twice continuously differentiable in the first argument and that satisfy $E[ \{\partial_d \gamma(D,X)\}^2]<\infty$ and $E[ \{\partial^2_d \gamma(D,X)\}^2]<\infty$.
     \item Example~\ref{ex:deriv}. $f(d\mid v,x)$ vanishes for each $d$ in the boundary of the support of $D$ given $(V,X)=(v,x)$ almost everywhere. $-\partial_d \log f(d \mid v,x)$ is a bounded above, and $\Gamma$ consists of functions
    $\gamma$ that are twice continuously differentiable in the first argument and that satisfy $E[ \{\partial_d \gamma(D,V,X)\}^2]<\infty$ and $E[ \{\partial^2_d \gamma(D,V,X)\}^2]<\infty$.
\end{enumerate}
\end{lemma}
Therefore a Sobolev type property with respect to the first argument is a sufficient condition in Examples~\ref{ex:elasticity} and~\ref{ex:deriv}.

Next we examine the assumption of $\|\alpha^{\min}_0\|_{\infty} \leq \bar{\alpha}$, which depends on the functional of interest.
\begin{lemma}[Bounded Riesz representer for global functionals]\label{lemma:bounded_RR_global}
The following conditions are sufficient for $\bar{\alpha}<\infty$ in Examples~\ref{ex:CATE},~\ref{ex:RDD},~\ref{ex:elasticity}, and~\ref{ex:deriv} with $\ell_h \leq C<\infty$ and $\Gamma=\mathbb{L}_2$.
\begin{enumerate}
    \item Example~\ref{ex:CATE}. $\pi_0(v,x)$ is bounded away from zero and one.
    \item Example~\ref{ex:RDD}. The bandwidth is fixed.
    \item Example~\ref{ex:elasticity}. $\alpha_0^{\min}$ that solves
    $
    -\partial_d \log f(d \mid x)=E\{\alpha(X,Z) \mid D=d,X=x\}
    $ is bounded above.
    \item Example~\ref{ex:deriv}. $
    -\partial_d \log f(d \mid v,x)
    $ is bounded above.
\end{enumerate}
\end{lemma}

\subsection{Local functionals}

Given a local functional
$$
\theta_0^h=E\{m_h(W,\gamma_0)\}=E\{\ell_h(V) m(W,\gamma_0)\},
$$
we will now refer to the Riesz representer of $m_h$ by $\alpha^h_0$ and the Riesz representer of $m$ by $\alpha_0$ for this subsection. The latter objects correspond to the global setting where the weighting is bounded. To lighten notation, we write $\ell=\ell_h$.

\begin{lemma}[Oracle moments for local functionals]\label{lemma:local}
Suppose $\alpha_0^{\min}$ exists and $\Gamma=\mathbb{L}_2$. Further suppose there exist $(\tilde{\alpha}, \check{\alpha}, \tilde{c}, \bar c, \tilde{f}, \bar f, \bar f', h_0)$ bounded away from zero and above such that the following conditions hold.
\begin{enumerate}
    \item Control of representer absolute value:
    $$
0< \tilde{\alpha} \leq \alpha_0(w) \leq \check{\alpha}.
$$
  \item Valid neighborhood. There exists $N_{h_0}(v)=(v':|v'-v|\leq h)\subset \mathcal{V}$.
    \item Bounded moments. For all $ h\leq h_0$ and for $q=2,3,4$, 
    $$
    U_1=m_h(W,\gamma_0)-E\{m_h(W,\gamma_0)\},\quad \|U_1\|_{\text{\normalfont pr},q}\leq \bar{c}\|\ell\|_{\text{\normalfont pr},q}.
    $$
    \item Bounded heteroscedasticity. For $q=2,3,4$,
    $$
    U_2=Y-\gamma_0(W),\quad \tilde{c}\leq \|U_2 \mid W\|_{\text{\normalfont pr},q}\leq \bar{c}.
    $$
    \item Bounded density. The density $f_V$ obeys, for all $v'$ in $N_{h_0}(v)$,
    $$
  0< \tilde{f} \leq f_V(v') \leq \bar f,\quad |\partial f_V(v')| \leq \bar f'.
    $$
\end{enumerate}
Then finite sample bounds in the proof hold. In summary,
$$
\frac{\kappa_h}{\sigma_h} \lesssim h^{-1/6} \lesssim \sigma_h \asymp \bar{M}_h \asymp h^{-1/2} \rightarrow \infty,\quad \kappa_h\lesssim h^{-2/3} \rightarrow \infty,\quad \zeta_h\lesssim h^{-3/4} \rightarrow \infty.
$$

\end{lemma}

The conditions of Lemma~\ref{lemma:bounded_RR_global} suffice for $0< \tilde{\alpha} \leq \alpha_0(w) \leq \check{\alpha}$ in Examples~\ref{ex:CATE} and~\ref{ex:RDD}.

As before, Assumption~\ref{assumption:cont} depends on the functional of interest.
\begin{lemma}[Mean square continuity for local functionals]\label{lemma:cont_local}
Suppose $\alpha_0^{\min}$ exists and $\Gamma\subset \mathbb{L}_2$. Further suppose there exist $(\tilde{f}, \bar f, \bar f', h_0)$ bounded away from zero and above such that the following conditions hold.
\begin{enumerate}
  \item Valid neighborhood. There exists $N_{h_0}(v)=(v':|v'-v|\leq h)\subset \mathcal{V}$.
    \item Bounded density. The density $f_V$ obeys, for all $v'$ in $N_{h_0}(v)$,
    $$
  0< \tilde{f} \leq f_V(v') \leq \bar f,\quad |\partial f_V(v')| \leq \bar f'.
    $$
    \item The conditions of Lemma~\ref{lemma:cont} hold.
\end{enumerate}
Then the finite sample bound in the proof holds for Examples~\ref{ex:CATE} and~\ref{ex:RDD}. In summary,
$$
\bar{Q}_h\lesssim h^{-2} \rightarrow \infty.
$$
\end{lemma}

As before, the assumption of $\|\alpha^{\min,h}_0\|_{\infty} \leq \bar{\alpha}$ depends on the functional of interest.
\begin{lemma}[Bounded Riesz representer for local functionals]\label{lemma:bounded_RR_local}
Suppose $\alpha_0^{\min}$ exists and $\Gamma\subset \mathbb{L}_2$. Further suppose there exist $(\tilde{f}, \bar f, \bar f', h_0,\bar{K})$ bounded away from zero and above such that the following conditions hold.
\begin{enumerate}
  \item Valid neighborhood. There exists $N_{h_0}(v)=(v':|v'-v|\leq h)\subset \mathcal{V}$.
    \item Bounded density. The density $f_V$ obeys, for all $v'$ in $N_{h_0}(v)$,
    $$
  0< \tilde{f} \leq f_V(v') \leq \bar f,\quad |\partial f_V(v')| \leq \bar f'.
    $$
    \item Bounded kernel. $|K(u)|\leq \bar{K}$.
    \item The conditions of Lemma~\ref{lemma:bounded_RR_global} hold, allowing bandwidth to vanish.
\end{enumerate}
Then the finite sample bound in the proof holds. In summary,
$$
\bar{\alpha}_h\lesssim h^{-1}\rightarrow \infty.
$$
\end{lemma}

The main results are in terms of abstract mean square rates $\mathcal{R}(\hat{\alpha}^h_{\ell})$ and $\mathcal{P}(\hat{\alpha}^h_{\ell})$ for the local Riesz representer $\alpha_0^{\min,h}$ of the functional $m_h$. A growing literature proposes machine learning estimators $\hat{\alpha}$ with rates $\mathcal{R}(\hat{\alpha}_{\ell})$ and $\mathcal{P}(\hat{\alpha}_{\ell})$ for the global Riesz representer $\alpha^{\min}_0$ of the functional $m$.

A natural choice of estimator $\hat{\alpha}^h$ for $\alpha_0^{\min,h}$ is the localization $\ell_h$ times an estimator $\hat{\alpha}$ for $\alpha^{\min}_0$. We prove that this choice allows an analyst to translate global Riesz representer rates into local Riesz representer rates under mild regularity conditions. In Supplement~1, we confirm that this choice performs well in simulations.

\begin{lemma}[Translating global rates to local rates]\label{lemma:translate_RR}
Suppose the conditions of Lemma~\ref{lemma:bounded_RR_local} hold with $\Gamma=\mathbb{L}_2$. Then
$$
\mathcal{R}(\hat{\alpha}^h_{\ell}) \lesssim h^{-2} \mathcal{R}(\hat{\alpha}_{\ell}),\quad \mathcal{P}(\hat{\alpha}^h_{\ell}) \lesssim h^{-2} \mathcal{P}(\hat{\alpha}_{\ell}).
$$
\end{lemma}

\subsection{Approximation error}

Finally, we characterize the finite sample approximation error $\Delta_h=
n^{1/2} \sigma^{-1}|\theta_0^h-\theta_0^{\lim}|$ where 
$$
\theta^{\lim}_{0}=\lim_{h\rightarrow 0} \theta_0^h,\quad \theta_0^h=E\{m_h(W,\gamma_0)\}=E\{\ell_h(V) m(W,\gamma_0)\}.
$$
$\Delta_h$ is bias from using a semiparametric sequence to approximate a nonparametric quantity.

We define $m(v)= E [m(W, \gamma_0) \mid V=v]$ to lighten notation.

\begin{lemma}[Approximation error from localization \cite{chernozhukov2018global}]\label{lemma:approx}
Suppose there exist constants $(h_0,K, \mathsf{v}, \bar g_{\mathsf{v}}$,  $\bar f_{\mathsf{v}}$, $\tilde{f}, \bar{g})$ bounded away from zero and above such that the following conditions hold.
\begin{enumerate}
  \item Valid neighborhood. There exists $N_{h_0}(v)=(v':|v'-v|\leq h)\subset \mathcal{V}$.
    \item Differentiability. On $N_{h_0}(v)$, $m(v')$ and $f_V(v')$ are differentiable to the integer order $\mathsf{sm}$.
    \item Bounded derivatives. Let $\mathsf{v}= \mathsf{sm} \wedge \mathsf{o}$ where $\mathsf{o}$ is the order of the kernel $K$. Let $\partial^\mathsf{v}_d$ denote the $\mathsf{v}$ order derivative $\partial^{\mathsf{v}}/(\partial d)^{\mathsf{v}}$. Assume
    $$
\sup_{ v' \in N_{h_0}(v)}  \| \partial^{\mathsf{v}}_v (m(v') f_V(v')) \|_{op} \leq \bar g_{\mathsf{v}}, \quad \sup_{ v' \in N_{h_0}(v) } \|\partial^{\mathsf{v}}_v f_V(v')  \|_{op} \leq \bar f_{\mathsf{v}},
 \quad  \inf_{ v' \in N_{h_0}(v) } f_V(v') \geq \tilde{f}.
$$ 
\item Bounded conditional formula.
$
m(v)f_V(v)\leq \bar{g}.
$
\end{enumerate}
Then there exist constants $(C,h_1)$ depending only on $(h_0,K, \mathsf{v}, \bar g_{\mathsf{v}}$,  $\bar f_{\mathsf{v}}$, $\tilde{f}, \bar{g})$ such that for all $h_1$ in $(h,h_0)$, $|\theta_0^h-\theta_0^{\lim}|\leq C h^\mathsf{v}.$
In summary,
$$
\Delta_h \lesssim n^{1/2} h^{\mathsf{v}+1/2}.
$$
\end{lemma}

To summarize the characterizations in this Supplement, we provide a corollary for local functionals. Let $\mathcal{R}(\hat{\alpha}_{\ell})$ and $\mathcal{P}(\hat{\alpha}_{\ell})$ be defined as in Lemma~\ref{lemma:translate_RR}.

\begin{corollary}[Confidence interval for local functionals]\label{cor:CI_local}
Suppose the conditions of Corollary~\ref{cor:CI} and Lemmas~\ref{lemma:local},~\ref{lemma:cont_local},~\ref{lemma:bounded_RR_local},~\ref{lemma:translate_RR}, and~\ref{lemma:approx} hold. As $n\rightarrow \infty$ and $h\rightarrow 0$, suppose the regularity condition on  moments $n^{-1/2}h^{-3/2}\rightarrow 0$ as well as the following learning rate conditions:
\begin{enumerate}
    \item $\left(h^{-1}+\bar{\alpha}'\right)\{\mathcal{R}(\hat{\gamma}_{\ell})\}^{q/2}=o_p(1)$;
    \item $\bar{\sigma}h^{-1}\{\mathcal{R}(\hat{\alpha}_{\ell})\}^{1/2}=o_p(1)$;
    \item $h^{-1/2}[\{n \mathcal{R}(\hat{\gamma}_{\ell}) \mathcal{R}(\hat{\alpha}_{\ell})\}^{1/2} \wedge \{n\mathcal{P}(\hat{\gamma}_{\ell})\mathcal{R}(\hat{\alpha}_{\ell})\}^{1/2} \wedge \{n\mathcal{R}(\hat{\gamma}_{\ell})\mathcal{P}(\hat{\alpha}_{\ell})\}^{1/2}] =o_p(1)$.
\end{enumerate}
Finally suppose the approximation error condition $\Delta_h \rightarrow 0$. Then the estimator $\hat{\theta}^h$ in Algorithm~\ref{alg:target} is consistent and asymptotically Gaussian, and the confidence interval in Algorithm~\ref{alg:target} includes $\theta^{\lim}_0$ with probability approaching the nominal level. Formally,
$$
\hat{\theta}^h=\theta^{\lim}_0+o_p(1),\quad \sigma_h^{-1}n^{1/2}(\hat{\theta}^h-\theta^{\lim}_0)\leadsto \mathcal{N}(0,1),\quad  \text{\normalfont pr} \left\{\theta^{\lim}_0 \text{ in }  \left(\hat{\theta}^h\pm c_a\hat{\sigma} n^{-1/2} \right)\right\}\rightarrow 1-a.
$$
\end{corollary}