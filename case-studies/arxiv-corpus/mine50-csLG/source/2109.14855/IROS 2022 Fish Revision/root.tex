\pdfminorversion=4
\documentclass[letterpaper, 10 pt, conference]{ieeeconf}
% \documentclass[a4paper, 10 pt, conference]{ieeeconf}

\usepackage{times}
\usepackage{graphicx}
\usepackage{wrapfig}
\usepackage{xfrac}

\usepackage{multicol}
\usepackage{amsmath}
\usepackage{siunitx}
\sisetup{range-phrase=--,range-units=single}

\usepackage{circuitikz}
\usetikzlibrary{patterns,hobby,decorations.pathmorphing}

\usepackage{multirow}

\usepackage{makecell}
\renewcommand\theadfont{\bfseries}

\usepackage{booktabs}

\usepackage[hidelinks]{hyperref}
\usepackage{cleveref}
\Crefformat{figure}{#2Fig.~#1#3}
\Crefmultiformat{figure}{Figs.~#2#1#3}{ and~#2#1#3}{, #2#1#3}{ and~#2#1#3}
\crefformat{footnote}{#2\footnotemark[#1]#3}

\IEEEoverridecommandlockouts

\overrideIEEEmargins

\title{\LARGE \bf
Sim2Real for Soft Robotic Fish via Differentiable Simulation
}


\author{John Z. Zhang$^{1,2}$, Yu Zhang$^{1}$, Pingchuan Ma$^{3}$,  Elvis Nava$^{1,4}$, Tao Du$^{3}$, Philip Arm$^{1}$, \\ Wojciech Matusik$^{3}$, Robert K. Katzschmann$^{1,4}$% <-this % stops a space
\thanks{*We are grateful for funding received by the ETH AI Center, and the Defense Advanced Research Projects Agency.}%
\thanks{$^{1}$Soft Robotics Lab, ETH Zurich, Switzerland}%
\thanks{$^{2}$Mechanical Engineering, MIT, USA}%
\thanks{$^{3}$Computer Science and AI Lab, MIT, USA}%
\thanks{$^{4}$ETH AI Center, ETH Zurich, Switzerland}
\thanks{Corresponding author: Robert K. Katzschmann, {\tt\footnotesize 
% \href{mailto:johnz@mit.edu}{johnz@mit.edu}, 
% \href{mailto:zhangyu@ethz.ch}{zhangyu@ethz.ch}, 
% \href{mailto:pcma@mit.edu}{pcma@mit.edu}, 
% \href{mailto:enava@ethz.ch}{enava}@ethz.ch,
% \href{mailto:taodu@csail.mit.edu}{taodu}@csail.mit.edu,
% \href{mailto:parm@ethz.ch}{parm}@ethz.ch,
% \href{mailto:wojciech@csail.mit.edu}{wojciech}@csail.mit.edu, 
\href{mailto:rkk@ethz.ch}{rkk@ethz.ch}}}% <-this % stops a space
}


\begin{document}

\maketitle
\thispagestyle{empty}
\pagestyle{empty}

\begin{abstract}
Accurate simulation of soft mechanisms under dynamic actuation is critical for the design of soft robots.
We address this gap with our differentiable simulation tool by learning the material parameters of our soft robotic fish.
On the example of a soft robotic fish, we demonstrate an experimentally-verified, fast optimization pipeline for learning the material parameters from quasi-static data via differentiable simulation and apply it to the prediction of dynamic performance.
Our method identifies physically plausible Young’s moduli for various soft silicone elastomers and stiff acetal copolymers used in creation of our three different robotic fish tail designs. We show that our method is compatible with varying internal geometry of the actuators, such as the number of hollow cavities.
Our framework allows high fidelity prediction of dynamic behavior for composite bi-morph bending structures in real hardware to millimeter-accuracy and within $3\%$ error normalized to actuator length.
We provide a differentiable and robust estimate of the thrust force using a neural network thrust predictor; this estimate allows for accurate modeling of our experimental setup measuring bollard pull. 
This work presents a prototypical hardware and simulation problem solved using our differentiable framework; the framework can be applied to higher dimensional parameter inference, learning control policies, and computational design due to its differentiable character.
\end{abstract}



\section{Introduction}

Machine learning fundamentally relies on the availability of data, which can
be sensitive or confidential.
It is now well-known that preventing learned models from leaking information
about individual training points requires particular attention
\citep{shokri2017Membership}.
A standard approach for training models while provably controlling the amount of
leakage is to solve an empirical risk minimization (ERM) problem
under a differential privacy (DP) constraint \citep{chaudhuri2011Differentially}.
In this work, we aim to design a differentially private algorithm which
approximates the solution to a composite ERM problem of the form:
\begin{align}
  \label{eq:dp-erm}
  w^* \in
  \argmin_{w \in \mathbb{R}^p}
  \left\{
  \frac{1}{n} \sum_{i=1}^n \ell(w; d_i) + \psi(w)
  \right\}
  \enspace,
\end{align}
where $D = (d_1, \dots, d_n)$
is a dataset of $n$ samples drawn from a universe $\cX$,
$\ell: \RR^p \times \cX \rightarrow \RR$ is a loss function which is convex
and smooth in $w$, and
$\psi: \RR^p \rightarrow \RR$ is a convex regularizer which is separable (\ie
$\psi(w) = \sum_{j=1}^p \psi_j(w_j)$) and typically nonsmooth (\eg
$\ell_1$-norm).

Differential privacy constraints induce a trade-off between the privacy and
the utility (i.e., optimization error) of the solution of~\eqref{eq:dp-erm}.
This trade-off was made explicit by \citet{bassily2014Private}, who derived
lower bounds on the achievable error given a fixed privacy budget.
To solve the DP-ERM problem in practice, the most popular approaches are based
on Differentially Private variants of Stochastic Gradient Descent (DP-SGD)
\citep{bassily2014Private,abadi2016Deep,wang2017Differentially}, in which
random perturbations are added to the (stochastic) gradients.
\citet{bassily2014Private} analyzed DP-SGD in the non-smooth DP-ERM setting,
and \citet{wang2017Differentially} then proposed an efficient DP-SVRG
algorithm for composite DP-ERM.
Both algorithms match known lower bounds.
SGD-style algorithms perform well in a wide variety of settings, but
also have some flaws: they either require small (or decreasing) step
sizes or variance reduction schemes to guarantee convergence, and they
can be slow when gradients' coordinates are imbalanced.
These flaws propagate to the private counterparts of these
algorithms.
Despite a few attempts at designing other differentially private solvers for
ERM under different setups
\citep{talwar2015Nearly,damaskinos2021Differentially}, the differentially
private optimization toolbox remains limited, which undoubtedly restricts the
resolution of practical problems.



In this paper, we propose and analyze a Differentially Private proximal
Coordinate
Descent algorithm (DP-CD), which performs updates based on perturbed
coordinate-wise gradients (\ie partial derivatives).  Coordinate
Descent (CD) methods have encountered a large success in non-private
machine learning due to their simplicity and effectiveness
\citep{liu2009Blockwise,friedman2010Regularization,chang2008Coordinate,sardy2000Block},
and have seen a surge of practical and theoretical interest in the
last decade \citep{Nesterov12,wright2015Coordinate,shi2017Primer,
  richtarik2014Iteration,fercoq2014Accelerated,tappenden2016Inexact,
  hanzely2020Variance,nutini2015Coordinate,karimireddy2019Efficient}.
In contrast to SGD, they converge with constant step sizes that adapt to
the coordinate-wise smoothness of
the objective. Additionally, CD updates naturally tend to
have a lower sensitivity. Operating with partial gradients thus enables
our private algorithm to reduce the perturbation required to
guarantee privacy without resorting to
amplification by
subsampling \citep{Balle_subsampling,mironov2019Enyi}.


We propose a novel analysis of proximal CD with perturbed gradients to
derive optimal upper bounds on the privacy-utility trade-off achieved
by DP-CD.
We prove a
recursion on distances of CD iterates to an optimal point that keeps track of
coordinate-wise regularity
constants in a tight manner and allows to use
large, constant step sizes that
yield high utility. Our results highlight the fact that DP-CD
can exploit imbalanced gradient coordinates to outperform DP-SGD.
They also improve upon known convergence rates for inexact CD in the
non-private setting
\citep{tappenden2016Inexact}.
We assess the optimality of DP-CD by deriving lower bounds
that capture coordinate-wise Lipschitz regularity measures, and show that
DP-CD matches those bounds up to logarithmic factors.
Our lower bounds also suggest interesting perspectives for future work on
DP-CD algorithms.

Our theoretical results
have important consequences for practical
implementations, which heavily rely on gradient clipping to achieve good
utility.
In contrast to DP-SGD, DP-CD requires to set \emph{coordinate-wise} clipping
thresholds, which can lead to impractical coordinate-wise hyperparameter tuning.
We instead propose a simple rule for adapting these thresholds from a
single hyperparameter. We also show how the coordinate-wise smoothness
constants used by DP-CD can be
estimated privately. We validate our theory with numerical
experiments on real and synthetic datasets. These experiments further
show that even in balanced problems, DP-CD can still improve over
DP-SGD, confirming the relevance of DP-CD for DP-ERM.

Our main contributions can be summarized as follows:
\begin{enumerate}
  \item We propose the first proximal CD algorithm for composite DP-ERM,
        formally prove its utility, and highlight regimes where it outperforms DP-SGD.
  \item We show matching lower bounds under coordinate-wise regularity
        assumptions.
      \item We give practical guidelines to use DP-CD, and show its
        relevance through numerical experiments.
\end{enumerate}


The rest of this paper is organized as follows.
We first describe some mathematical background in
\Cref{sec:preliminaries}.
In \Cref{sec:diff-priv-coord}, we present our DP-CD algorithm,
show that it satisfies DP, establish utility guarantees, and
compare these guarantees with those of DP-SGD.
In \Cref{sec:utility-lower-bounds}, we derive lower bounds under
coordinate-wise regularity assumptions, and
show that DP-CD can match them. \Cref{sec:dp-cd-practice} discusses practical
questions related to gradient clipping and the private estimation of
smoothness constants.
\Cref{sec:numerical-experiments} presents our numerical experiments,
comparing DP-CD and DP-SGD on LASSO and $\ell_2$-regularized
logistic regression problems. %
Finally, we review existing work in
\Cref{sec:related-works}, and conclude with promising lines of future work in
\Cref{sec:conclusion-and-discussion}.


\section{Related Work}
\label{related_work}

\subsection{Soft underwater robots}
Soft robots are difficult to optimally design and control when compared to their rigid counterparts due to the infinite dimensionality of their compliant structures.
Due to this modeling complexity, an experienced designer must hand craft each design guided by intuition, experiments, and approximate models.
Marchese et al. offer approaches to designing and fabricating soft fluidic elastomer robots, the type of robot we are also using in this work~\cite{marchese2015recipe}.
%
Katzschmann et al. present the design, fabrication, control, and testing of a soft robotic fish with interior cavities that is hydraulically actuated. Their manually designed robot can swim at multiple depths and record aquatic life in the ocean~\cite{katzschmann2016hydraulic,katzschmann2018exploration}. 
Zhu et al. manually optimize the swimming performance of their robotic fish, Tunabot~\cite{zhu2019tuna}. The authors measured kinematics, speed, and power at increasing flapping frequencies to quantify swimming performance and find agreement in performance between real fish and their Tunabot over a wide range of frequencies.
Zheng et al. propose to design soft robots by pre-checking controllability during the numerical design phase~\cite{zheng2019controllability}. FEM is used to model the dynamics of cable-driven parallel soft robot and a differential geometric method is applied to analyze the controllability of the points of interest.
Katzschmann et al.~\cite{katzschmann2019dynamically} manually tweak the material parameters of their reduced-order FEM~\cite{thieffry2018control} with an experimental soft robotic arm to perform dynamic closed-loop control.
Van et al. present a DC motor driven soft robotic fish which is optimized for speed and efficiency based on experimental, numerical and theoretical investigation into oscillating propulsion~\cite{van2020biomimetic}.
Wolf et al. use a pneumatically-actuated fish-like stationary model to investigate how parameters like stiffness, strength, and frequency affect thrust force generation~\cite{wolf2020fish}. Wolf et al. measure thrust, side forces, and torques generated during propulsion and use a statistical linear model to examine the effects of parameter combinations on thrust generation; they show that both stiffness and frequency substantially affect swimming kinematics.
We are not aware of any work that uses a fast differentiable FEM simulation environment to learn material parameters for soft robotic fish using a bollard-pull style experimental setup.

\subsection{Differentiable soft-body simulators}
Our work is also relevant to the recent developments of robotic simulators, particularly for soft robots.
Geilinger et al. \cite{geilinger2020add} present a differentiable multi-body dynamics solver that is able to handle frictional contact for rigid and deformable objects.
Coevoet et al. \cite{coevoet2017software} notably present a non-differentiable framework for modeling, simulation, and control of soft-bodied robots using continuum mechanics for modeling the robotic components and using Lagrange multipliers for boundary conditions like actuators and contacts.
Most related to our work are the recent works on differentiable soft-body and fluid simulators~\cite{du2020stokes,du2021diffpd,hahn2019real2sim,hu2019difftaichi,hu2019chainqueen,huang2021plasticine,ma2021diffaqua}. 
These papers develop numerical methods for computing gradients in a traditional simulators. Furthermore, they demonstrate the power of gradient information in robotics applications, e.g., system identification or trajectory optimization. Most of the works present simulation results only, with ChainQueen~\cite{hu2019chainqueen} and Real2Sim~\cite{hahn2019real2sim} being two notable exceptions that discuss real-world soft-robot applications.
Notably, \cite{hahn2019real2sim} optimizes visco-elastic material parameters of a finite element simulation to approximate the dynamic deformations of real-world soft objects, such as an open-loop controlled tendon-driven crawling robot.
Bern et al. \cite{bern2020soft} have also demonstrated the use of differentiable simulation to learn from a quasi-static data set for the purpose of optimizing open-loop control inputs.
Dubied et al. \cite{dubied2022sim} is the most recent example that demonstrates sim2real matching for a soft robotic fish tail, shows system identification on a passive structure for just the Young's modulus, and investigates the mismatch in damping between reality and simulation. In this previous work, the fish tail actuation is simulated using a simplified muscle model and only one design is shown whereas in this paper, the pressure boundary condition is simulated exactly as fabricated for each pneumatic chamber geometry for three different designs. Simulating the pneumatic chambers improves accuracy and allows for physically-plausible Young's moduli and Poisson ratios to be identified. In this current work, we further demonstrate that the gradient-based optimization can be carried out to higher dimensional design spaces that include more than one material parameter.

\subsection{Hydrodynamic Surrogates}
For underwater soft robots, the challenge of simulation is exacerbated by the hydrodynamic interaction with the soft body.
Several previous works tackle the fluid-structure interaction problem through different methods, including heuristic hydrodynamics~\cite{du2021underwater,ma2021diffaqua,min2019softcon}, physically-informed neural network approaches~\cite{wandel_learning_2021}, and data-driven learning approaches~\cite{chen2018neural}.

Compared to these previous methods that simulate underwater soft robots such as~\cite{du2021underwater}, our work models pneumatic actuation using the exact chamber geometry rather than artificial muscles facilitating greater accuracy at large deformations (see \Cref{fig:pneumatic_fish_tail}), uses a neural network thrust predictor rather than approximate analytical or heuristic hydrodynamics, and presents a more sophisticated hardware pipeline that can be used to validate simulation. 
version https://git-lfs.github.com/spec/v1
oid sha256:3b88816ee467263f308e8ca2b5749af8d1e52f948eb584ace4552c0d13a221fd
size 2037

version https://git-lfs.github.com/spec/v1
oid sha256:4c9759bda7cb8993ddce6499546d5032474a6eb40b2bc2b51979f49d064d740b
size 4624

\section{Experiments}


We evaluate our algorithm on a range of continuous control tasks from OpenAI Gym \cite{gymopenai} and the meta world benchmark \cite{yu2020meta} that both use  the physics engine MuJoCo \cite{mujoco} (version 1.5). 
First, we benchmark ACC against strong methods that do not use environment specific hyerparameters.
Then we compare the performance of TQC with a fixed number of dropped targets per network with that of ACC.
Next, we evaluate the effect of more critic updates for ACC and show results in the sample efficient regime.
Further, we study the effect of ACC on the accuracy of the value estimate, and investigate the generality of ACC by applying it to TD3.







We implemented ACC on top of the PyTorch code published by the authors\footnote{\url{https://github.com/bayesgroup/tqc_pytorch}} to ensure a fair comparison.
While in general a safe strategy is to use a very high value for $d_{max}$ as it gives ACC more flexibility in choosing the right amount of bias correction we set it to $d_{max}=5$, which is the maximum value used by TQC for the number of dropped targets in the original publication.
At the beginning of the training we initialize $\beta = 2.5$ and set the step size parameter to $\alpha=0.1$.
After $T_\beta = 1000$ steps since the last update and when the next episode finishes, $\beta$ is updated with a batch that stores the most recent state-action pairs encountered in the environment and their corresponding observed discounted returns. 
After every update of $\beta$ the oldest episodes in this stored batch are removed until there are no more than $5000$ state-action pairs left.
This means that on average $\beta$ is updated with a batch whose size is a bit over $5000$. 
The updates of $\beta$ are started after $25000$ environment steps and
the moving average parameter in the normalization of the $\beta-$update is set to $0.05$. 
The  first $5000$ environment interactions are generated with a random policy after which learning starts.
We did not tune most of these additional hyperparameters and some choices are directly motivated by the environment (e.g. setting $T_\beta$ to the maximum episode length). Only for $\alpha$ we tested a few different choices but found that for reasonable values it does not have a noticeable influence on performance. 
% We spend only a very limited amount of computation time into the tuning of the previously mentioned hyperparameters.
All hyperparameters of the underlying TQC algorithm  with $N=5$ critic networks were left unchanged.




Compared to TQC the additional computational overhead caused by ACC is minimal because there is only one update to $\beta$ that is very cheap compared to one training step of the actor-critic and there are at least $T_\beta =1000$ training steps in between one update to $\beta$.





During training, the policy is evaluated every 1,000 environment steps by averaging the episode returns of $10$ rollouts with the current policy. For each task and algorithm we run 10 trials each with a different random seed.



\subsection{Comparative Evaluation}




We compare ACC to the state of the art continuous control methods SAC \cite{SAC} (with learned temperature parameter \cite{SACalgapp}) and TD3 \cite{td3} on six OpenAI Gym continuous control environments.
To make the different environments comparable we normalize the scores by dividing the achieved return by the best achieved return among all evaluations points of all algorithms for that environment.

Figure \ref{fig:comparative_aggregated_results}a)  shows the aggregated data efficiency curve over all $6$ tasks computed with the method of \cite{agarwal2021deep}, where the interquantile mean (IQM) ignores the bottom and top $25$\% of the runs across all games and computes the mean over the remaining. 
The absolute performance of ACC for each single task can be seen in Figure \ref{fig:ablation_const_number_dropped_atoms_single_curves}.
Overall, ACC reaches a much higer performance than SAC and TD3.


\subsection{Robotics Benchmark}
To investigate, if ACCs strong performance also translates into robotics environments, we evaluate ACC and SAC on $12$ of the more challenging tasks in the Meta-World benchmark \cite{yu2020meta}, which consists of several manipulation tasks with a Sawyer arm. We use version V2 and use the following $12$ tasks:
sweep, stick-pull, dial-turn, door-open, peg-insert-side, push, pick-out-of-hole, push-wall, faucet-open, hammer, stick-push, soccer.
We evaluate the single tasks in the in the MT1 version of the benchmark, where the goal and object positions change across episodes.
Different to the gym environments, $\beta$ is updated every $500$ environment steps as this is the episode length for these tasks.
Figure 
\ref{fig:comparative_aggregated_results}b)
shows the aggregated data efficiency curve in terms of success rate over all $12$ tasks computed with the method of \cite{agarwal2021deep}.


The curves demonstrate that ACC achieves drastically stronger results than SAC both in terms of data efficiency and asymptotic performance.
After $2$ million steps ACC already achieves a close to optimal task success rate which is even considerably higher than what SAC achieves at the end of the training.
This shows, that ACC is a promising approach for real world robotics applications.

\begin{figure}[t]
\footnotesize
\setlength{\tabcolsep}{1pt}
\centering 
% \hspace{0mm}
%\begin{tabular}{P{.49\linewidth}P{.49\linewidth}}
\begin{tabular}{cc}
        \includegraphics[width=.49\linewidth]{images/main_exp/sac_td3_acc_aggregated_0-eps-converted-to.pdf} &
        \includegraphics[width= .49\linewidth]{images/main_exp/meta_world_aggregated_mean_std_0-eps-converted-to.pdf} \\
        a) & b) \\
\end{tabular}
\vspace{-0.3cm}
\caption{
Sample efficiency curves aggregated from the results over several environments. The normalized IQM score and the mean of the success rate respectively is plotted against the number of environment steps. Shaded regions denote pointwise $95$\% stratified bootstrap confidence intervals according to the method of \cite{agarwal2021deep}. 
\textbf{(a)} Aggregated results over the $6$ gym continuous control tasks.
\textbf{(b)} Aggregated results over the $12$ metaworld tasks.
}
\label{fig:comparative_aggregated_results}
\vspace{-0.5cm}
\end{figure}


\subsection{Fixing the Number of Dropped Targets}



In this experiment we evaluate how well ACC performs when compared to TQC where the number of dropped targets per network $d$ is fixed to some value.
Since in the original publication for each environment the optimal value was one of the three values $0$, $2$, and $5$, we evaluated TQC with $d$ fixed to one of these values for each environment.
To ensure comparability we used the same codebase as for ACC. 
The results in Figure \ref{fig:ablation_const_number_dropped_atoms_single_curves} show that it is not possible to find one value for $d$ that performs well on all environments.
With $d=0$, TQC is substantially worse on three environments and unstable on the \textit{Ant} environment.
Setting $d=2$ is overall the best choice but still performs clearly worse for two environments and is also slightly worse for \textit{Humanoid}.
Dropping $d=5$ targets per network leads to an algorithm that can compete with ACC only on two of the six environments.
Furthermore, even if there would be one tuned parameter that performs equally well as ACC on a given set of environments we hypothesize there are likely very different environments for which the specific parameter choice will not perform well. The principled nature of ACC on the other hand provides reason to believe that it can perform robustly on a wide range of different environments. This is supported by the robust performance on all considered environments.











\begin{figure}
    \centering
    \includegraphics[width=0.93\linewidth]{images/ablation/ablation_const_drop_results_one_fig.pdf}
    \caption{Learning curves of ACC applied to TQC and TQC with different fixed choices for the number of dropped atoms $d$ on six OpenAi gym environments. We used version \textit{v3}. The shaded area represents  mean $\pm$ standard deviation over the $10$ trials. For readability the curves showing the mean are filtered  with a uniform filter of size $15$.}
    \label{fig:ablation_const_number_dropped_atoms_single_curves}
\vspace{-0.5cm}
\end{figure}



\subsection{Evaluation of Sample Efficient Variant}






\begin{figure*}[t]
\footnotesize
\centering 
%\begin{tabular}{P{.56\linewidth}P{.39\linewidth}}
\begin{tabular}{cc}
    \includegraphics[width=.56\linewidth]{images/less_steps/results_sample_efficient_all_utds_size23.pdf} &
    \includegraphics[width=.39\linewidth]{images/main_exp/acc_td3.pdf} \\
    a) & b) \\
\end{tabular}
% \hspace{0mm}
\vspace{-0.3cm}
\caption{
The mean $\pm$ standard deviation over $10$ trials. 
\textbf{(a)} Results in the sample efficient regime where tuning of hyperparameters in an inner loop is too costly with different choices for the number of value function updates per environment step.
\textbf{(b)} Results for ACC applied to TD3 compared to pure TD3.}
\label{fig:further_eval}
\vspace{-0.5cm}
\end{figure*}




In principle more critic updates per environment step should make learning faster. However, because of the bootstrapping in the target computation this can easily become unstable.
The problem is that as targets are changing faster, bias can build up easier and divergence becomes more likely.
ACC provides a way to detect upbuilding bias in the TD targets and to correct the bias accordingly.
This motivates to increase the number of gradient updates of the critic.
In TD3, SAC and TQC one critic update is performed per environment step.
We conducted an experiment to study the effect of increasing this rate up to $4$.
ACC using $4$, $2$ and $1$ updates are denoted with ACC\_4q, ACC\_2q and ACC\_1q respectively. ACC\_1q is equal to ACC from the previous experiments. We use the same notation also for TD3 and SAC.

Scaling the number of critic updates by a factor of $4$ increases the computation time by a factor of $4$. But this can be worthwhile in the sample efficient regime, where a huge number of environment interactions is not possible or the interaction cost dominate the computational costs as it is the case when training robots in the real world.
The results in Figure 
\ref{fig:further_eval}a)
show that in the sample efficient regime ACC4q further increases over plain ACC.
ACC4q reaches the final performance of TD3 and SAC in less than a third of the number of steps for five environments and for \textit{Humanoid} in roughly half the number of steps. 
Increasing the number of critic updates for TD3 and SAC shows mixed results, increasing performance for some environments while decreasing it for others. Only ACC benefits from more updates on all environments, which supports the hypothesis that ACC is successful at calibrating the critic estimate.
% such that the learning dynamics are stable also with more critic updates.

\subsection{Analysis of ACC}


\begin{figure*}[t]
\footnotesize
\centering 
%\begin{tabular}{P{.77\linewidth}P{.22\linewidth}}
\begin{tabular}{cc}
    \includegraphics[width=.77\linewidth]{images/analysis/visualize_beta_all_envs.pdf} &
    \hspace{-.4cm}\includegraphics[width=.22\linewidth]{images/analysis/value_error_aggregated_mean_std_0.pdf} \\
    a) & b) \\
\end{tabular}
% \hspace{0mm}
\vspace{-0.3cm}
\caption{
\textbf{(a)} Mean (thick line) and standard deviation (shaded area) over 10 trials of the number of dropped targets per network $d = d_{max} - \beta$ in ACC over time for different environments with a uniform filter of size 15.
\textbf{(b)} The normalized absolute error of the value estimate aggregated over the $6$ environments. Shown are the mean with stratified bootstrapped confidence intervals computed from the results of $5$ trials per environment. We used a uniform filter of size $401$ for readability.}
\label{fig:analysis}
\vspace{-0.5cm}
\end{figure*}

To evaluate the effect of ACC on the bias of the value estimate, we analyze the difference between the value estimate and the corresponding observed return when ACC is applied to TQC.
For each state-action pair encountered during exploration, we compute its value estimate at that time and at the end of the episode compare it  with the actual discounted return from that state onwards. Hence, the state-action pair was not used to update the value function at the point when the value estimate has been computed.
If an episode ends because the maximum number of episode time-steps has been reached, which is 1,000 for the considered environments, we ignore the last $100$ state-action pairs. The reason is that in TQC the value estimator is trained to ignore the episode timeout and uses a bootstrapped target also at the end of the episode. 
We normalize for different value scales by computing the absolute error between the value estimate and the observed discounted return and divide that by the absolute value of the discounted return.
Every 1,000 steps, the average over the errors of the last 1,000 state-action pairs is computed.
The aggregated results in Figure 
\ref{fig:analysis}b)
show that averaged over all environments ACC indeed achieves a lower value error than TQC with the a fixed number of dropped atoms $d$.
This supports our hypothesis that the strong performance of ACC applied to TQC indeed stems from better values estimates.



To better understand the hidden training dynamics of ACC we show in Figure
\ref{fig:analysis}a)
how the number of dropped targets per network $d = d_{max} - \beta$ evolves during training.
Interestingly, the relatively low standard deviation indicates a similar behaviour across runs for a specific environment.
However, there are large differences between the environments which indicates that it might not be possible to find a single hyperparameter that works well on a wide variety of different environments.
Further, the experiments shows that the optimal amount of overestimation correction might change over time during the training even on a single environment.

\subsection{Beyond TQC: Improving TD3 with ACC}

To demonstrate the generality of ACC, we additionally applied it to the actor-critic style TD3 algorithm \cite{td3},
which uses two critics. These are initialized differently but trained with the same target value, which is the minimum over the two targets computed from the two critics.
% This is done to prevent overestimation in the value estimates.
While this successfully prevents the overestimation bias, using the minimum of the two target estimates is very coarse and can instead lead to an underestimation bias.
We applied ACC to TD3 by defining the target for each critic network to be a convex combination between its own target and the minimum over both targets.
Let $Q_i = Q_{\bar{\theta}_i} (s_{t+1}, \pi_{\bar{\phi}} (s_{t+1}) )$, we define the $k$-th critic target
\vspace{-.1cm}
\begin{equation}
\label{eq:td3_target_acc}
    y_k = r + \gamma 
    \Big(   \beta ~ Q_k \nonumber 
     + (1-\beta) \min_{i=1,2} Q_i
    \Big),
\vspace{-.1cm}
\end{equation}
where $\beta \in [0,1]$ is the ACC parameter that is adjusted to balance between under- and overestimation.
The results are displayed in Figure 
\ref{fig:further_eval}b)
and show that ACC also improves the performance of TD3.













version https://git-lfs.github.com/spec/v1
oid sha256:b98deb8f644e4091e5bf0e5e588ab61b8cb0c82267b1e25c1ed8a442f355f40b
size 4907

\section{Results}
\label{results}

\subsection{Learned Material Parameters}

\begin{figure}[t]
    \centering
    \includegraphics[width=0.85\columnwidth]{figures/grid_search_nemo_vertical_column.png}
    \caption{\textit{Top:} Exhaustive search of the Young's moduli pair for the spine and body. The moduli pair with the lowest Euclidean loss is located at the red $\times$. The red dot indicates the start of a gradient search and the white line shows the progress towards convergence. \textit{Middle:} Convergence comparison between a gradient-based (Adam) and gradient-free search (CMA-ES). \textit{Bottom:} Comparison of static deformation of the \emph{Nemo} fish for increasing pressure in experiment and simulation. Maximum displacement error increases with pressure, but remains within the measurement error (on the order of the diameter of markers). The grid in white under the real robot has \SI{10}{mm} spacing. The reported error is normalized to the fish tail length of \SI{10}{cm}.}
    \label{fig:system_id}
\end{figure}

\begin{figure}[t]
    \centering
    \includegraphics[width=0.99\columnwidth]{figures/param_search_figure_new.png}
    \caption{Learning four material parameters from deformation data take with the \textit{Nemo} fish prototype using a gradient-based approach that is run until convergence. The final values for Young's moduli and Poisson's ratios depicted as an asterisk agree with plausible material parameter values with lower loss.}
    \label{fig:4_param_search}
\end{figure}

For the \emph{Nemo} tail actuator described in \Cref{tab:fish}, we perform a grid search of the loss landscape centered around the ground truth datasheet values for the silicone and acetal materials of the body and spine. We report that if the exact geometry of the actuator is reproduced with high fidelity in the simulation, we converge to values within the range of typical measured values for the material Young's moduli (see \Cref{fig:system_id}). Further, we see that there is a unique minimum value that is within the acceptable range of measured moduli for both parameters. Note that we assume a priori that the Poisson ratio for silicone is $\nu\approx0.5$ or \textit{nearly} incompressible, a standard assumption for silicone, and the acetal sheet Poisson ratio is $\nu=0.37$ as reported by the manufacturer. Typical values for the Young's modulus of Dragon Skin 10 range from \SI{0.1}{MPa} to \SI{0.25}{MPa} and the value for Dragon Skin 20 was measured to be in the range of \SI{1.1}{MPa}~\cite{marechal2020toward} and typical values for the Young's modulus of acetal sheets range from \SI{2.5}{GPa} to \SI{5}{GPa}.\footnote{\url{https://dielectricmfg.com/knowledge-base/acetal/}}

In \Cref{fig:4_param_search}, we demonstrate that our method can be extended to higher dimensional parameter spaces such as a search over both Young's moduli and Poisson's ratios. Note that although we allow the Poisson ratio to vary in this identification experiment, the final value to which the body Poisson's ratio converges is still \textit{nearly} incompressible as expected for silicone.


\subsection{Gradient-based and Gradient-free Solver Methods}

We compare the runtime of the gradient-free method \emph{CMA-ES} against the runtime of the gradient-based method \emph{Adam} in \Cref{tab:gradient_methods}. The comparison shows that although \emph{CMA-ES}~\cite{igel2007covariance} is slightly faster in runtime per iteration, the use of the gradient-based method Adam~\cite{kingma2014adam} is significantly more effective for convergence. These comparative experiments shown in \Cref{fig:system_id} were carried out on a computer with an Intel Core i9-9900K @ 3.60GHz with 16 cores processor and 64.0 GB of memory.

\begin{table}[htb]
    \centering
    \caption{Comparison of Adam, CMA-ES, and Grid search. The forward simulation time is 318.9 seconds equivalent to grid search.}
    % \begin{tabular}{|p{10mm}|p{12mm}|p{12mm}|p{10mm}|p{17mm}|}
    \begin{tabular}{@{}lllll@{}}
        \toprule
        \thead[l]{Method} & \thead[l]{Total\\Iterations} & \thead[l]{Time Per\\ Iteration} & \thead[l]{Total Time} & \thead[l]{Loss After \\4 Iterations} \\
        \midrule
        \textbf{Adam} & \textbf{4} & 334.4 s & \textbf{~22 m} & \textbf{0.0027} \\
        CMA-ES & 40 & 321.0 s & ~214 m & 0.021  \\
        Grid search & 25 & \textbf{318.9 s} & 132.9 m & 0.023 \\
        \bottomrule
    \end{tabular}
    \label{tab:gradient_methods}
\end{table}

\subsection{Dynamic Experiments}

For dynamic experiments, we compare the results of our simulation output with the measured data of the furthest tracked dot on the tail. In Fig.~\ref{fig:results}, we report our findings for sim2real performance in both the standard fish (\emph{Nemo}) and two other data sets for a tail with different Young's moduli for the body (\emph{Bruce}) and a tail with a greater number of air chambers (\emph{Dory}). We demonstrate that if system identification is done correctly our simulation results can predict the performance of a novel actuator design to within millimeter precision or $3\%$ max normalized error using only a quasistatic data set for training without need for material testing. The \textit{Bruce} prototype required higher pressures to get similar displacements to \textit{Nemo} since the material is stiffer. The prototype \textit{Dory} required higher pressure as well to produce similar displacements due to a greater number of actuation chambers. For videos of the dynamic experiments and simulation, we ask readers to refer to our supplemental video.

\begin{figure}[t]
    \centering
    \includegraphics[width=0.85\columnwidth]{figures/dynamic_actuation_square.png}
    \caption{Simulation and measurement data for the \emph{Nemo} fish at (a) 200 mbar and 2 Hz, (b) 200 mbar and 4 Hz, (c) the \emph{Bruce} fish at 500 mbar and 2 Hz, and (d) the \emph{Dory} fish at 350 mbar and 2 Hz. The color bands indicate the variance from $N=5$ trials. For both actuation signals, we are capable of achieving sub-millimeter accuracy between experiment and simulation. The same method is used to identify the parameters of Bruce and Dory, exhibiting accuracy still within \SI{3}{mm}. We normalize the Root Mean Square Error (RMSE) to the fish tail length of \SI{10}{cm}. As expected, higher pressures result in larger deformations with greater error. The phase lag exhibited in \emph{Bruce} and \emph{Dory} may be due to actuator dynamics present during higher actuation pressures.}
    \label{fig:results}
    \vspace{-3pt}
\end{figure}

\subsection{Learned Hydrodynamics}
\label{sec:learned_hydrodynamics}
We compare our simple predictor of thrust force with the measurement data and the theoretical thrust from EBT, a classic, non-learning-based approximate analytical model from the literature, in~\Cref{fig:thrust_prediction}. Although the model is capable of generalizing to actuation signals at frequencies previously unobserved in the training for the \emph{Nemo} prototype, for the \emph{Bruce} prototype the discrepancy is large, nearly twice the measured force, likely because of the more limited training data.
In comparing to the thrust prediction from EBT, we see that the analytical thrust is strictly positive. The negative measured force is due to recoil of the measurement system. The advantage of our approach is the differentiability and flexibility of the neural network provided enough data is collected. We note that the network is capable of learning the measurement dynamics due to the compliance of the load cell and the damping of the water though these effects can be considered separately and corrected for if desired (see \Cref{sec:load_cell_measurement}). We conjecture that the asymmetry of the measured thrust may be due to imperfections in the fabrication process of the fish tails favoring one direction more strongly.

\begin{figure}[t]
    \centering
    \includegraphics[width=1\columnwidth]{figures/hydrodynamics.png}
    \caption{Tail lateral displacement (blue) with thrust measurement and prediction. We compare the measured thrust force $\mathbf{f}_\textrm{m}$ (black), analytical thrust force from EBT $\mathbf{f}_\textrm{EBT}$ (magenta), and the neural network thrust prediction $\mathbf{f}_\textrm{thrust}$ (green). We show the thrust prediction for two training sets (a) and (c) and we report the time-averaged thrust for the two test cases (b) and (d). We note that EBT tends to over-predict the thrust measurement while our neural network thrust prediction accurately reproduces the frequency for a given actuation signal and matches amplitude more robustly than EBT.}
    \label{fig:thrust_prediction}
\end{figure}

\section{Conclusion}
\label{conclusion}
We present an experimentally-verified simulation framework that can be used to accurately predict the deformations of a pneumatically actuated fish tail with a flexible spine.
Our pipeline can accurately learn material parameters from a quasi-static data sets without having to do expensive and time-consuming material testing. It also eliminates the need to do manual tuning of material constants to get accurate simulation results. The parameters we found are not only within typical range of measured material parameters for our materials, but can be used to successfully predict the behavior of dynamic experiments for different pressure actuation amplitudes and frequencies to within $3\%$ positional error normalized to a actuator length of \SI{10}{cm}. Although we use an isotropic corotated material, which is linear elastic, we find that this model is more sufficient to model large deformations on average giving acceptable displacement results for our engineering application. In these experiments, the damping of the material and the hydrodynamic effects are found to be negligible. This is because the actuation pressures used dominate the deformation compared to losses and hydrodynamic pressure. 

We show a data-driven approach can be used to do simple prediction on a useful performance metric such as thrust force given a suitable hardware setup. However, more work is needed to produce a more robust thrust predictor if the morphology of the actuator changes substantially. We claim that for small design changes such as the choice of silicone or the number of internal chambers this framework can be used to quickly assess the relative merits of each design with a relatively sparse data set of approximately 30 types of different actuation signals.

Our aim is to further progress towards a systematic method by which soft roboticists can simulate and optimize their designs and controllers, whether they be soft fish, manipulators, or other flavors of soft robots. A fast and physically-verified co-optimization method of design and control is the goal.
\section{Appendix : Analysis}

\subsection{Additional ablation results on ETTh2 dataset}
\label{appendix:ablation_etth2}

\begin{figure}[!ht]
    \centering
    \subfloat[ETTh2 Univariate\label{fig:ablation_archi_uni_etth2}]{%
      \includegraphics[width=0.40\textwidth]{figs/archi_ablation_uni_ETTh2.png}
    }
    \subfloat[ETTh2 Multivariate\label{fig:ablation_archi_multi_etth2}]{%
      \includegraphics[width=0.40\textwidth]{figs/archi_ablation_multi_ETTh2.png}
    } 
\caption{Figures \ref{fig:ablation_archi_multi_etth2}, \ref{fig:ablation_archi_uni_etth2} illustrates the reduction in MAE loss (y-axis) by  the Yformer architecture in comparison with the Informer baseline for the ETTh2 univariate and multivariate settings respectively. The Yformer ($\alpha=0$) represent the Yformer architecture without the reconstruction loss
}
\label{fig:archi_abltation_etth2}
\end{figure}


\begin{figure}[!ht]
    \centering
    \subfloat[ETTh2 Univariate\label{fig:skipless_ablation_uni_ETTh2}]{%
      \includegraphics[width=0.40\textwidth]{figs/skipless_ablation_uni_ETTh2.png}
    }
    \subfloat[ETTh2 Multivariate\label{fig:skipless_ablation_multi_ETTh2}]{%
      \includegraphics[width=0.40\textwidth]{figs/skipless_ablation_multi_ETTh2.png}
    } 
\caption{Impact of the U-Net connection for the Yformer architecture. The Yformer$^*$ architecture represents the Yformer without the U-Net connection.}
\label{fig:skipless_ablation_2}
\end{figure}


\subsection{Performance variability analysis}

We report the standard deviation values from the multiple Yformer runs for the ETTh2 dataset and compare them with the numbers reported from the Informer baseline \cite{zhou2020informer}. The standard deviation values are quite small across the three runs of the Yformer with multiple initial seed settings illustrating the stability of Yformer across the multiple horizons.

\begin{table}[htbp!]
\caption{Comparison of Yformer model with the second best performing Informer model for performance variability analysis.}
\resizebox{1\textwidth}{!}{%
\begin{tabular}{|c|c|c|c|c|c|c|c|}
\hline
\multicolumn{1}{|c|}{Setting} & \multicolumn{1}{c|}{Model} & Metric & \multicolumn{1}{c|}{24} & \multicolumn{1}{c|}{48} & \multicolumn{1}{c|}{168} & \multicolumn{1}{c|}{336} & \multicolumn{1}{c|}{720} \\ \hline
\multirow{4}{*}{Univariate} & \multirow{2}{*}{Yformer}  & MSE    & $0.082\pm0.004$ & $0.172\pm0.016$ & $0.174\pm0.009$ & $0.224\pm0.038$ & $0.211\pm0.005$ \\ \cline{3-8} 
                                    &                           & MAE    & $0.221\pm0.006$ & $0.334\pm0.014$ & $0.337\pm0.007$ & $0.391\pm0.036$ & $0.382\pm0.005$ \\ \cline{2-8} 
                                    & \multirow{2}{*}{Informer} & MSE    & 0.093         & 0.155         & 0.232         & 0.263         & 0.277         \\ \cline{3-8} 
                                    &                           & MAE    & 0.24          & 0.314         & 0.389         & 0.417         & 0.431         \\ \hline
\multirow{4}{*}{Multivariate}        & \multirow{2}{*}{Yformer}   & MSE    & $0.412\pm0.063$             & $1.171\pm0.027$           & $2.171\pm0.105$            & $2.260\pm0.112$            & $2.595\pm0.131$              \\ \cline{3-8} 
                              &                            & MAE    & $0.498\pm0.049$             & $0.865\pm0.029$           & $1.218\pm0.047$            & $1.283\pm0.009$            & $1.337\pm0.066$              \\ \cline{2-8} 
                              & \multirow{2}{*}{Informer}  & MSE    & 0.720                    & 1.457                   & 3.489                    & 2.723                    & 3.467                    \\ \cline{3-8} 
                              &                            & MAE    & 0.665                   & 1.001                   & 1.515                    & 1.340                     & 1.473                    \\ \hline
\end{tabular}%
}
\end{table}

\newpage

\section{Appendix: Operators}

$\operatorname{\textbf{ProbSparseAttn}}$: Attention module that uses the ProbSparse method introduced in \cite{zhou2020informer}. The query matrix $\overline{\boldsymbol{Q}} \in \mathbb{R}^{L_Q \times d}$ denotes the sparse query matrix with $u$ dominant queries.

\begin{equation}
\begin{aligned}
  \mathcal{A^{\text{PropSparse}}}(\boldsymbol{\overline{Q}}, \boldsymbol{K}, \boldsymbol{V}) &= \text{Softmax}(\frac{\boldsymbol{\overline{Q}}\boldsymbol{K}^T}{\sqrt{d}})\boldsymbol{V}
    % \hat{y}_{fut} &= \hat{y}_{T:T+\tau} \\
    % \hat{y}_{re} &= \hat{y}_{T-|x_i|:T} \\
\end{aligned}
\label{eqn:probattn}
\end{equation}


$\operatorname{\textbf{MaskedAttn}}$: Canonical self-attention with masking to prevent positions from attending to subsequent positions in the future \cite{vaswani2017attention}.

$\operatorname{\textbf{Conv1d}}$: Given $N$ batches of 1D array of length $L$ and $C$ number of channels/dimensions. A convolution operation produces an output: 

\begin{equation}
\begin{aligned}
    \text{out}(N_i, C_{\text{out}_j}) = \text{bias}(C_{\text{out}_j}) +
        \sum_{k = 0}^{C_{in} - 1} \text{weight}(C_{\text{out}_j}, k)
        \star \text{input}(N_i, k)
\end{aligned}
\label{eqn:conv1d}
\end{equation}

For further reference please visit \href{https://pytorch.org/docs/stable/generated/torch.nn.Conv1d.html}{pytorch Conv1D} page 


$\operatorname{\textbf{LayerNorm}}$: Layer Normalization introduced in \cite{layernorm}, normalizes the inputs across channels/dimensions. $\operatorname{LayerNorm}$ is the default normalization in common transformer architectures \cite{vaswani2017attention}. Here, $\gamma$ and $\beta$ are learnable affine transformations.


\begin{equation}
\begin{aligned}
    \text{out}(N, *) = \frac{\text{input}(N, *) - \mathrm{E}[\text{input}(N, *)]}{ \sqrt{\mathrm{Var}[\text{input}(N, *) ] + \epsilon}} * \gamma + \beta
\end{aligned}
\label{eqn:layernorm}
\end{equation}



$\operatorname{\textbf{MaxPool}}$: Given $N$ batches of 1D array of length $L$, and $C$ number of channels/dimensions. A $\operatorname{MaxPool}$ operation produces an output. 

\begin{equation}
\begin{aligned}
    \text{out}(N_i, C_j, k) = \max_{m=0, \ldots, \text{kernel\_size} - 1}
                \text{input}(N_i, C_j, \text{stride} \times k + m)
\end{aligned}
\label{eqn:maxpool}
\end{equation}

For further reference please visit \href{https://pytorch.org/docs/stable/generated/torch.nn.MaxPool1d.html}{pytorch MaxPool1D} page 

$\operatorname{\textbf{ELU}}$: Given an input $x$, the $\operatorname{ELU}$ applies element-wise non linear activation function as shown.

\begin{equation}
\begin{aligned}
    \text{ELU}(x) = \begin{cases}
        x, & \text{ if } x > 0\\
        \alpha * (\exp(x) - 1), & \text{ if } x \leq 0
        \end{cases}
\end{aligned}
\label{eqn:elu}
\end{equation}


$\operatorname{\textbf{ConvTranspose1d}}$: Also known as deconvolution or fractionally strided convolution, uses convolution on padded input to produce upsampled outputs (see \href{https://pytorch.org/docs/stable/generated/torch.nn.ConvTranspose1d.html}{pytorch ConvTranspose1d} page).

\section{Appendix : Hyperparameters}
\label{appendix:hyperparameters}

We follow Informer \cite{zhou2020informer} baseline for all the hyperparameter setting like the convolution kernel size, stride etc. The hyperparameter tuning performed are only for the parameters mentioned below. In order to reproduce the experiments, please use the default Informer/Yformer configurations and adapt only the below mentioned parameters for each horizon.


\begin{table}[htbp!]
\caption{Optimal hyperparameters across different horizon and datasets for the univariate setting. All the remaining hyperparameters are retained from the Informer Model.}
\label{tbl:hyp_univariate}
\centering
\resizebox{1\textwidth}{!}{%
\begin{tabular}{|c|c|c|S|S|S|c|c|}
\hline
Dataset                & Horizon $\tau$& History Length & {Weight Decay} & {Learning Rate} & {Reconstruction Factor $\alpha$} & Batch Size & Encoder Blocks \\ \hline
\multirow{5}{*}{ETTh1} & 24      & 720        & 0            & 0.0001        & 0.7   & 32         & 2              \\ \cline{2-8} 
                      & 48      & 720        & 0         & 0.0001         & 0.7   & 16         & 4              \\ \cline{2-8} 
                      & 168     & 720        & 0            & 0.001         & 0.7   & 32         & 4              \\ \cline{2-8} 
                      & 336     & 720        & 0.05         & 0.0001        & 0.1   & 32         & 4              \\ \cline{2-8} 
                      & 720     & 720        & 0.05         & 0.0001        & 0.7   & 16         & 2              \\ \hline
\multirow{5}{*}{ETTh2} & 24      & 48         & 0            & 0.0001        & 0.7   & 32         & 2              \\ \cline{2-8} 
                      & 48      & 96         & 0.02         & 0.0001        & 0.3   & 32         & 4              \\ \cline{2-8} 
                      & 168     & 336        & 0.02         & 0.001         & 0.3   & 32         & 2              \\ \cline{2-8} 
                      & 336     & 336        & 0.09         & 0.0001        & 0     & 32         & 2              \\ \cline{2-8} 
                      & 720     & 336        & 0.09         & 0.0001        & 0.7   & 16         & 2              \\ \hline
\multirow{5}{*}{ETTm1} & 24      & 96         & 0.02         & 0.0001        & 0.7   & 32         & 4              \\ \cline{2-8} 
                      & 48      & 96         & 0.02         & 0.0001        & 0.7   & 32         & 4              \\ \cline{2-8} 
                      & 96      & 384        & 0.02         & 0.0001        & 0.1   & 32         & 4              \\ \cline{2-8} 
                      & 288     & 384        & 0.02         & 0.001         & 0.7   & 16         & 2              \\ \cline{2-8} 
                      & 672     & 384        & 0.07         & 0.001         & 0.3   & 16         & 2              \\ \hline
\multirow{5}{*}{ECL}   & 48      & 168        & 0            & 0.0001        & 0.7   & 16         & 2              \\ \cline{2-8} 
                      & 168     & 168        & 0.01         & 0.0001        & 0.3   & 16         & 2              \\ \cline{2-8} 
                      & 336     & 168        & 0.01         & 0.0001        & 0.7   & 16         & 2              \\ \cline{2-8} 
                      & 720     & 168        & 0            & 0.0001        & 0.1   & 16         & 2              \\ \cline{2-8} 
                      & 960     & 48         & 0            & 0.0001        & 0.5   & 16         & 4              \\ \hline
\end{tabular}%
}

\end{table}



\begin{table}[htbp!]
\caption{Optimal hyperparameters across different horizon and datasets for the multivariate setting. All the remaining hyperparameters are retained from the Informer Model.}
\label{tbl:hyp_multivariate}
\centering
\resizebox{1\textwidth}{!}{%
\begin{tabular}{|c|c|c|S|S|S|c|c|}
\hline
Dataset                & Horizon $\tau$& History Length & {Weight Decay} & {Learning Rate} & {Reconstruction Factor $\alpha$} & Batch Size & Encoder Blocks \\ \hline
\multirow{5}{*}{ETTh1} & 24      & 48         & 0            & 0.0001        & 0.7   & 32         & 3              \\ \cline{2-8} 
                      & 48      & 96         & 0.02         & 0.001         & 0.5   & 32         & 2              \\ \cline{2-8} 
                      & 168     & 168        & 0.02         & 0.001         & 0.7   & 32         & 2              \\ \cline{2-8} 
                      & 336     & 168        & 0            & 0.0001        & 0.7   & 32         & 4              \\ \cline{2-8} 
                      & 720     & 336        & 0.05         & 0.0001        & 1     & 16         & 2              \\ \hline
\multirow{5}{*}{ETTh2} & 24      & 48         & 0            & 0.0001        & 0.7   & 32         & 2              \\ \cline{2-8} 
                      & 48      & 96         & 0.02         & 0.001         & 0     & 32         & 4              \\ \cline{2-8} 
                      & 168     & 336        & 0.09         & 0.001         & 0.7   & 32         & 2              \\ \cline{2-8} 
                      & 336     & 336        & 0.07         & 0.001         & 0.3   & 32         & 2              \\ \cline{2-8} 
                      & 720     & 336        & 0            & 0.0001        & 0     & 16         & 2              \\ \hline
\multirow{5}{*}{ETTm1} & 24      & 672        & 0            & 0.0001        & 0.7   & 32         & 2              \\ \cline{2-8} 
                      & 48      & 96         & 0            & 0.0001        & 0.7   & 32         & 4              \\ \cline{2-8} 
                      & 96      & 384        & 0.05         & 0.0001        & 0.7   & 32         & 4              \\ \cline{2-8} 
                      & 288     & 672        & 0.02         & 0.001         & 0.5   & 16         & 2              \\ \cline{2-8} 
                      & 672     & 672        & 0.02         & 0.0001        & 0.3   & 16         & 2              \\ \hline
\multirow{5}{*}{ECL}   & 48      & 24         & 0            & 0.0001        & 0.7   & 16         & 3              \\ \cline{2-8} 
                      & 168     & 48         & 0            & 0.0001        & 0.7   & 16         & 3              \\ \cline{2-8} 
                      & 336     & 24         & 0            & 0.0001        & 0.5   & 16         & 2              \\ \cline{2-8} 
                      & 720     & 48         & 0            & 0.0001        & 0.7   & 16         & 2              \\ \cline{2-8} 
                      & 960     & 336        & 0            & 0.0001        & 0.7   & 16         & 2              \\ \hline
\end{tabular}%
}

\end{table}



% \addtolength{\textheight}{-9cm}

\bibliographystyle{IEEEtran}
\bibliography{references}


\end{document}
