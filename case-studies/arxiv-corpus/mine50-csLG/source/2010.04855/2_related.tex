\section{Related work}\label{sec:related}

We view nonparametric causal functions as reweightings of an underlying regression, synthesizing the $g$ formula \cite{robins1986new} and partial means \cite{newey1994kernel} frameworks. To express causal functions in this way, we build on canonical identification theorems under the assumption of selection on observables \cite{rosenbaum1983central,robins1986new,altonji2005cross}. We propose simple, global estimators that combine kernel ridge regressions. Previous works that take a global view include \cite{van2003unified,luedtke2016super,diaz2013targeted,kennedy2020optimal}, and references therein. A broad literature instead views causal functions as collections of localized treatment effects and proposes local estimators with Nadaraya--Watson smoothing, e.g. \cite{imai2004causal,rubin2005general,rubin2006extending,galvao2015uniformly,luedtke2016statistical,kennedy2017nonparametric,semenova2021debiased,kallus2018policy,chernozhukov2022debiased,fan2019estimation,zimmert2019nonparametric,colangelo2020double}, and references therein. By taking a global view rather than a local view, we propose simple estimators that can be computed once and evaluated at any value of a continuous treatment, rather than a computationally intensive procedure that must be reimplemented at any treatment value.

Our work appears to be the first to reduce estimation of dose, heterogeneous, and incremental response curves to kernel ridge regressions. Previous works incorporating the RKHS into nonparametric estimation focus on different causal functions: nonparametric instrumental variable regression \cite{carrasco2007linear,darolles2011nonparametric,singh2019kernel}, and heterogeneous treatment effect conditional on the full vector of covariates \cite{nie2021quasi}. %In particular, \cite{singh2019kernel} propose a kernel generalization of two stage least squares and prove projected mean square error rates. 
%Though \cite{singh2019kernel} focus on ill posed inverse problems and we focus on partial means, we build on their insights about the role of conditional expectation operators and conditional mean embeddings in causal inference.  
\cite{nie2021quasi} propose the R learner to estimate the heterogeneous treatment effect $\theta_0(x)=E\{Y^{(1)}-Y^{(0)} \mid X=x\}$. \cite[Section 3]{nie2021quasi} reviews the extensive literature that considers this estimand. The R learner minimizes a loss that contains inverse propensities and different regularization \cite[eq. A24]{nie2021quasi}, and it does not appear to have a closed form solution. The authors prove oracle mean square error rates. By contrast, we pursue a more general heterogeneous response curve with discrete or continuous treatment, conditional on some interpretable subvector $V$ \cite{abrevaya2015estimating}: $\theta_0(d,v)=E\{Y^{(d)} \mid V=v\}$. Unlike previous work on nonparametric causal functions in the RKHS, we (i) consider dose, heterogeneous, and incremental response curves; (ii) propose estimators with closed form solutions; and (iii) prove uniform consistency, which is an important norm for policy evaluation.

We extend the framework from causal functions to counterfactual distributions. Existing work focuses on distributional generalizations of average treatment effect (ATE) or average treatment on the treated (ATT) for binary treatment \cite{firpo2007efficient,cattaneo2010efficient,chernozhukov2013inference}, e.g. $\theta_0=\text{\normalfont pr}\{Y^{(1)}\}-\text{\normalfont pr}\{Y^{(0)}\}$. \cite{muandet2021counterfactual} propose an RKHS approach for distributional ATE and ATT with binary treatment using inverse propensity scores and an assumption on the smoothness of a ratio of densities, which differs from our approach. Unlike previous work, we (i) allow treatment to be continuous; (ii) avoid inversion of propensity scores and densities; and (iii) study a broad class of counterfactual distributions for the full population, subpopulations, and alternative populations, e.g. $\theta_0(d,v)=\text{\normalfont pr}\{Y^{(d)} \mid V=v\}$. 

%This paper subsumes our previous draft from 2020, circulated with a different title.
We provide a detailed comparison with kernel methods for binary treatment effects in Section~\ref{sec:detail}. Whereas we study causal functions, these works study causal scalars \cite{kallus2020generalized,hirshberg2019minimax,singh2021debiased}. We clarify the sense in which our causal function estimators generalize known estimators for treatment effects to new estimators for causal functions. Previous work is inherently tied to the $\mathbb{L}^2$ bounded functional perspective. However, evaluation of a causal function is not a bounded functional over all of $\mathbb{L}^2$ \cite{van1991differentiable,newey1994asymptotic}. Therefore our algorithms extend the conceptual framework of kernel methods for causal inference in a new direction. Our statistical contribution is a new, uniform analysis of response curves that goes beyond pointwise approximation of response curves by local treatment effects. 

This paper subsumes our previous draft \cite[Section 2]{singh2020kernel}.