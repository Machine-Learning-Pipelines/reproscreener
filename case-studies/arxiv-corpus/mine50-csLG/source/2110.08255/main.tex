% This is samplepaper.tex, a sample chapter demonstrating the
% LLNCS macro package for Springer Computer Science proceedings;
% Version 2.21 of 2022/01/12
%
\pdfoutput=1
\RequirePackage{amsmath}
\documentclass[runningheads]{llncs}
%
\usepackage[T1]{fontenc}
% T1 fonts will be used to generate the final print and online PDFs,
% so please use T1 fonts in your manuscript whenever possible.
% Other font encondings may result in incorrect characters.
%

% other packages
\usepackage{amssymb}
\usepackage{multirow}
\usepackage[table,xcdraw]{xcolor}
\usepackage{textcomp} % for \textquotesingle
\usepackage{subfig}
\usepackage{algorithmic}
\usepackage{algorithm}
\usepackage{graphicx}
\usepackage{hyperref}
\usepackage[normalem]{ulem}
\usepackage{multirow}
\usepackage{booktabs}
\usepackage{threeparttable}
\usepackage{siunitx}
\usepackage[misc]{ifsym}
% Used for displaying a sample figure. If possible, figure files should
% be included in EPS format.
%
% If you use the hyperref package, please uncomment the following two lines
% to display URLs in blue roman font according to Springer's eBook style:
%\usepackage{color}
%\renewcommand\UrlFont{\color{blue}\rmfamily}
%
\newcommand\doubleplus{+\kern-1.3ex+\kern0.8ex}
\begin{document}
%

\title{U-Net Inspired Transformer Architecture for \\Far Horizon Time Series Forecasting}

%
\titlerunning{Yformer for Far Horizon Time Series Forecasting}
% If the paper title is too long for the running head, you can set
% an abbreviated paper title here
%
% \orcidID{0000-0001-6356-8646}
% \orcidID{1111-2222-3333-4444}
% \orcidID{0000-0002-7402-4166}
% \orcidID{2222--3333-4444-5555} 
% \orcidID{0000-0001-5729-6023}

\author{Kiran Madhusudhanan\inst{1}\Letter \and
Johannes Burchert \inst{1} \and
Nghia Duong-Trung \inst{2}\and
Stefan Born \inst{2} \and
Lars Schmidt-Thieme \inst{1} 
}
\authorrunning{K. Madhusudhanan et al.}
% First names are abbreviated in the running head.
% If there are more than two authors, 'et al.' is used.
%
\institute{Institute for Computer Science, University of Hildesheim, Hildesheim, Germany \\
\email{\{madhusudhanan, burchert, schmidt-thieme\}@ismll.uni-hildesheim.de}
\and
Technische Universit\"at Berlin, Berlin, Germany \\
\email{nghia.duong-trung@tu-berlin.de, born@math.tu-berlin.de}}

\toctitle{U-Net Inspired Transformer Architecture for Far Horizon Time Series Forecasting}
\tocauthor{Kiran~Madhusudhanan, Johannes~Burchert, Nghia~Duong-Trung, Stefan~Born and Lars~Schmidt-Thieme}

\maketitle              % typeset the header of the contribution
%
\begin{abstract}
Time series data is ubiquitous in research as well as in a wide variety of industrial applications. Effectively analyzing the available historical data and providing insights into the far future allows us to make effective decisions. Recent research has witnessed the superior performance of transformer-based architectures, especially in the regime of far horizon time series forecasting. However, the current state of the art sparse Transformer architectures fail to couple down- and upsampling procedures to produce outputs in a similar resolution as the input. We propose a U-Net inspired Transformer architecture named Yformer, based on a novel Y-shaped encoder-decoder architecture that (1) uses direct connection from the downscaled encoder layer to the corresponding upsampled decoder layer in a U-Net inspired architecture, (2) Combines the downscaling/upsampling with sparse attention to capture long-range effects, and (3) stabilizes the encoder-decoder stacks with the addition of an auxiliary reconstruction loss. Extensive experiments have been conducted with relevant baselines on three benchmark datasets, demonstrating an average improvement of 19.82, 18.41 percentage MSE and 13.62, 11.85 percentage MAE in comparison to the baselines for the univariate and the multivariate settings respectively.

\keywords{Time series Forecasting  \and Transformer \and U-Net}
\end{abstract}

\section{Introduction}

In the most simple case, time series forecasting deals with a scalar
time-varying signal and aims to predict or forecast its values in the near future; for example, countless applications in finance, healthcare, production automatization, etc. \cite{cao2018brits,sagheer2019time,sezer2020financial} can benefit from an accurate forecasting solution.
Often not just a single scalar signal is of interest, but multiple at once,
and further time-varying signals are available and even \textsl{known for the future}.
For example, suppose one aims to forecast the energy consumption of a house, it likely depends on the social time that one seeks to forecast for (such as the next hour or day), and also on features of these time points (such as weekday, daylight, etc.), which are known already for the future. This is also the case in model predictive control \cite{camacho2013model}, where one is interested
to forecast the expected value realized by some planned action, then this action is also known at the time of forecast.
More generally,
time series forecasting, nowadays deals with quadruples $(x,y,x',y')$
of known past predictors $x$, known past targets $y$, known future predictors $x'$
and sought future targets $y'$. 

\begin{figure}[ht]
\centering

\includegraphics[width=0.4\columnwidth]{figs/ts_ps.png}
\caption{General time series setting illustrating the quadruples $(x,y,x',y')$ denoting the \textsl{past predictors}, \textsl{past targets}, \textsl{future predictors} and \textsl{future targets} respectively. Given the history information $(x, y)$ until time $t = T$ and the future predictors $(x')$ for the next $\tau$ time steps, time series forecasting predicts the target $y'$ from $t = T+1, \dots, \tau$ time steps. In the figure, $O$ and $M$ represents the respective channels of the targets and the predictors.}
\label{fig:ts_ps}
\end{figure}

Time series problems can often be addressed by methods developed initially
for images, treating them as 1-dimensional images. Especially for
time-series classification many typical time series encoder architectures
have been adapted from models for images \cite{wang2017time,ZOU201939}. 
Time series forecasting then is closely related to image outpainting \cite{wang2019srn},
the task to predict how an image likely extends to the left, right, top or bottom,
as well as to the more well-known task of image segmentation,
where for each input pixel, an output pixel has to be predicted, whose channels
encode pixel-wise classes such as vehicle, road, pedestrian say for road scenes.
Time series forecasting combines aspects from both problem settings:
information about targets from shifted positions (e.g., the past targets $y$ as
 in image outpainting) and
information about other channels from the same positions (e.g., the future predictors $x'$
 as in image segmentation).
One of the most successful, principled architectures for the image segmentation
task are U-Nets introduced in \cite{ronneberger2015u}, an architecture that successively downsamples/coarsens
its inputs and then upsamples/refines the latent representation with
deconvolutions also using the latent representations of the same detail level,
tightly coupling down- and upsampling procedures and thus yielding latent
features on the same resolution as the inputs. 


Following the great success in Natural Language Processing (NLP) applications, attention-based, esp. transformer-based
architectures \cite{vaswani2017attention} that model pairwise interactions
between sequence elements have been recently adapted for
time series forecasting. One of the significant
challenges, is that the length of the time series, are often one or two magnitudes of order larger than the (sentence-level) NLP problems. 

Plenty of approaches aim to mitigate the quadratic complexity $O(T^2)$ in
the sequence/time series length $T$ to at most $O(T\log T)$.
For example, the Informer architecture
\cite{zhou2020informer}, adapts the transformer with a sparse attention
mechanism and a successive downsampling/coarsening of the past time series. As in the original transformer, only the coarsest representation is fed
into the decoder. Possibly to remedy the loss in resolution by this procedure,
the Informer feeds its input a second time into the decoder network, this time
without any coarsening. 

While forecasting problems share many commonalities with image segmentation
problems, transformer-based architectures like the Informer do not
involve coupled down- and upscaling procedures to yield predictions
on the same resolution as the inputs. 
Thus, we propose a novel Y-shaped architecture that
\begin{enumerate}
\item Couples downscaling/upscaling to leverage both, coarse and fine-grained
       features for time series forecasting,
\item Combines the coupled scaling mechanism with sparse attention modules to capture long-range effects on all scale levels, and
\item Stabilizes encoder and decoder stacks by reconstructing the recent past.
\end{enumerate}





version https://git-lfs.github.com/spec/v1
oid sha256:5638d77e93a44dc2859bfbace73d275915f2e5b8ad919b25198e4b3c3b4f853b
size 3735


version https://git-lfs.github.com/spec/v1
oid sha256:d9c64af75406281f324b5118bf0fe8f9691f9ca3f415f72e8345bc5bc50ba1a9
size 5195


version https://git-lfs.github.com/spec/v1
oid sha256:c65db184e19d9295968256baac82954360646a290daa1d6f159a97905967c0b6
size 10075


\section{Experiments}


We evaluate our algorithm on a range of continuous control tasks from OpenAI Gym \cite{gymopenai} and the meta world benchmark \cite{yu2020meta} that both use  the physics engine MuJoCo \cite{mujoco} (version 1.5). 
First, we benchmark ACC against strong methods that do not use environment specific hyerparameters.
Then we compare the performance of TQC with a fixed number of dropped targets per network with that of ACC.
Next, we evaluate the effect of more critic updates for ACC and show results in the sample efficient regime.
Further, we study the effect of ACC on the accuracy of the value estimate, and investigate the generality of ACC by applying it to TD3.







We implemented ACC on top of the PyTorch code published by the authors\footnote{\url{https://github.com/bayesgroup/tqc_pytorch}} to ensure a fair comparison.
While in general a safe strategy is to use a very high value for $d_{max}$ as it gives ACC more flexibility in choosing the right amount of bias correction we set it to $d_{max}=5$, which is the maximum value used by TQC for the number of dropped targets in the original publication.
At the beginning of the training we initialize $\beta = 2.5$ and set the step size parameter to $\alpha=0.1$.
After $T_\beta = 1000$ steps since the last update and when the next episode finishes, $\beta$ is updated with a batch that stores the most recent state-action pairs encountered in the environment and their corresponding observed discounted returns. 
After every update of $\beta$ the oldest episodes in this stored batch are removed until there are no more than $5000$ state-action pairs left.
This means that on average $\beta$ is updated with a batch whose size is a bit over $5000$. 
The updates of $\beta$ are started after $25000$ environment steps and
the moving average parameter in the normalization of the $\beta-$update is set to $0.05$. 
The  first $5000$ environment interactions are generated with a random policy after which learning starts.
We did not tune most of these additional hyperparameters and some choices are directly motivated by the environment (e.g. setting $T_\beta$ to the maximum episode length). Only for $\alpha$ we tested a few different choices but found that for reasonable values it does not have a noticeable influence on performance. 
% We spend only a very limited amount of computation time into the tuning of the previously mentioned hyperparameters.
All hyperparameters of the underlying TQC algorithm  with $N=5$ critic networks were left unchanged.




Compared to TQC the additional computational overhead caused by ACC is minimal because there is only one update to $\beta$ that is very cheap compared to one training step of the actor-critic and there are at least $T_\beta =1000$ training steps in between one update to $\beta$.





During training, the policy is evaluated every 1,000 environment steps by averaging the episode returns of $10$ rollouts with the current policy. For each task and algorithm we run 10 trials each with a different random seed.



\subsection{Comparative Evaluation}




We compare ACC to the state of the art continuous control methods SAC \cite{SAC} (with learned temperature parameter \cite{SACalgapp}) and TD3 \cite{td3} on six OpenAI Gym continuous control environments.
To make the different environments comparable we normalize the scores by dividing the achieved return by the best achieved return among all evaluations points of all algorithms for that environment.

Figure \ref{fig:comparative_aggregated_results}a)  shows the aggregated data efficiency curve over all $6$ tasks computed with the method of \cite{agarwal2021deep}, where the interquantile mean (IQM) ignores the bottom and top $25$\% of the runs across all games and computes the mean over the remaining. 
The absolute performance of ACC for each single task can be seen in Figure \ref{fig:ablation_const_number_dropped_atoms_single_curves}.
Overall, ACC reaches a much higer performance than SAC and TD3.


\subsection{Robotics Benchmark}
To investigate, if ACCs strong performance also translates into robotics environments, we evaluate ACC and SAC on $12$ of the more challenging tasks in the Meta-World benchmark \cite{yu2020meta}, which consists of several manipulation tasks with a Sawyer arm. We use version V2 and use the following $12$ tasks:
sweep, stick-pull, dial-turn, door-open, peg-insert-side, push, pick-out-of-hole, push-wall, faucet-open, hammer, stick-push, soccer.
We evaluate the single tasks in the in the MT1 version of the benchmark, where the goal and object positions change across episodes.
Different to the gym environments, $\beta$ is updated every $500$ environment steps as this is the episode length for these tasks.
Figure 
\ref{fig:comparative_aggregated_results}b)
shows the aggregated data efficiency curve in terms of success rate over all $12$ tasks computed with the method of \cite{agarwal2021deep}.


The curves demonstrate that ACC achieves drastically stronger results than SAC both in terms of data efficiency and asymptotic performance.
After $2$ million steps ACC already achieves a close to optimal task success rate which is even considerably higher than what SAC achieves at the end of the training.
This shows, that ACC is a promising approach for real world robotics applications.

\begin{figure}[t]
\footnotesize
\setlength{\tabcolsep}{1pt}
\centering 
% \hspace{0mm}
%\begin{tabular}{P{.49\linewidth}P{.49\linewidth}}
\begin{tabular}{cc}
        \includegraphics[width=.49\linewidth]{images/main_exp/sac_td3_acc_aggregated_0-eps-converted-to.pdf} &
        \includegraphics[width= .49\linewidth]{images/main_exp/meta_world_aggregated_mean_std_0-eps-converted-to.pdf} \\
        a) & b) \\
\end{tabular}
\vspace{-0.3cm}
\caption{
Sample efficiency curves aggregated from the results over several environments. The normalized IQM score and the mean of the success rate respectively is plotted against the number of environment steps. Shaded regions denote pointwise $95$\% stratified bootstrap confidence intervals according to the method of \cite{agarwal2021deep}. 
\textbf{(a)} Aggregated results over the $6$ gym continuous control tasks.
\textbf{(b)} Aggregated results over the $12$ metaworld tasks.
}
\label{fig:comparative_aggregated_results}
\vspace{-0.5cm}
\end{figure}


\subsection{Fixing the Number of Dropped Targets}



In this experiment we evaluate how well ACC performs when compared to TQC where the number of dropped targets per network $d$ is fixed to some value.
Since in the original publication for each environment the optimal value was one of the three values $0$, $2$, and $5$, we evaluated TQC with $d$ fixed to one of these values for each environment.
To ensure comparability we used the same codebase as for ACC. 
The results in Figure \ref{fig:ablation_const_number_dropped_atoms_single_curves} show that it is not possible to find one value for $d$ that performs well on all environments.
With $d=0$, TQC is substantially worse on three environments and unstable on the \textit{Ant} environment.
Setting $d=2$ is overall the best choice but still performs clearly worse for two environments and is also slightly worse for \textit{Humanoid}.
Dropping $d=5$ targets per network leads to an algorithm that can compete with ACC only on two of the six environments.
Furthermore, even if there would be one tuned parameter that performs equally well as ACC on a given set of environments we hypothesize there are likely very different environments for which the specific parameter choice will not perform well. The principled nature of ACC on the other hand provides reason to believe that it can perform robustly on a wide range of different environments. This is supported by the robust performance on all considered environments.











\begin{figure}
    \centering
    \includegraphics[width=0.93\linewidth]{images/ablation/ablation_const_drop_results_one_fig.pdf}
    \caption{Learning curves of ACC applied to TQC and TQC with different fixed choices for the number of dropped atoms $d$ on six OpenAi gym environments. We used version \textit{v3}. The shaded area represents  mean $\pm$ standard deviation over the $10$ trials. For readability the curves showing the mean are filtered  with a uniform filter of size $15$.}
    \label{fig:ablation_const_number_dropped_atoms_single_curves}
\vspace{-0.5cm}
\end{figure}



\subsection{Evaluation of Sample Efficient Variant}






\begin{figure*}[t]
\footnotesize
\centering 
%\begin{tabular}{P{.56\linewidth}P{.39\linewidth}}
\begin{tabular}{cc}
    \includegraphics[width=.56\linewidth]{images/less_steps/results_sample_efficient_all_utds_size23.pdf} &
    \includegraphics[width=.39\linewidth]{images/main_exp/acc_td3.pdf} \\
    a) & b) \\
\end{tabular}
% \hspace{0mm}
\vspace{-0.3cm}
\caption{
The mean $\pm$ standard deviation over $10$ trials. 
\textbf{(a)} Results in the sample efficient regime where tuning of hyperparameters in an inner loop is too costly with different choices for the number of value function updates per environment step.
\textbf{(b)} Results for ACC applied to TD3 compared to pure TD3.}
\label{fig:further_eval}
\vspace{-0.5cm}
\end{figure*}




In principle more critic updates per environment step should make learning faster. However, because of the bootstrapping in the target computation this can easily become unstable.
The problem is that as targets are changing faster, bias can build up easier and divergence becomes more likely.
ACC provides a way to detect upbuilding bias in the TD targets and to correct the bias accordingly.
This motivates to increase the number of gradient updates of the critic.
In TD3, SAC and TQC one critic update is performed per environment step.
We conducted an experiment to study the effect of increasing this rate up to $4$.
ACC using $4$, $2$ and $1$ updates are denoted with ACC\_4q, ACC\_2q and ACC\_1q respectively. ACC\_1q is equal to ACC from the previous experiments. We use the same notation also for TD3 and SAC.

Scaling the number of critic updates by a factor of $4$ increases the computation time by a factor of $4$. But this can be worthwhile in the sample efficient regime, where a huge number of environment interactions is not possible or the interaction cost dominate the computational costs as it is the case when training robots in the real world.
The results in Figure 
\ref{fig:further_eval}a)
show that in the sample efficient regime ACC4q further increases over plain ACC.
ACC4q reaches the final performance of TD3 and SAC in less than a third of the number of steps for five environments and for \textit{Humanoid} in roughly half the number of steps. 
Increasing the number of critic updates for TD3 and SAC shows mixed results, increasing performance for some environments while decreasing it for others. Only ACC benefits from more updates on all environments, which supports the hypothesis that ACC is successful at calibrating the critic estimate.
% such that the learning dynamics are stable also with more critic updates.

\subsection{Analysis of ACC}


\begin{figure*}[t]
\footnotesize
\centering 
%\begin{tabular}{P{.77\linewidth}P{.22\linewidth}}
\begin{tabular}{cc}
    \includegraphics[width=.77\linewidth]{images/analysis/visualize_beta_all_envs.pdf} &
    \hspace{-.4cm}\includegraphics[width=.22\linewidth]{images/analysis/value_error_aggregated_mean_std_0.pdf} \\
    a) & b) \\
\end{tabular}
% \hspace{0mm}
\vspace{-0.3cm}
\caption{
\textbf{(a)} Mean (thick line) and standard deviation (shaded area) over 10 trials of the number of dropped targets per network $d = d_{max} - \beta$ in ACC over time for different environments with a uniform filter of size 15.
\textbf{(b)} The normalized absolute error of the value estimate aggregated over the $6$ environments. Shown are the mean with stratified bootstrapped confidence intervals computed from the results of $5$ trials per environment. We used a uniform filter of size $401$ for readability.}
\label{fig:analysis}
\vspace{-0.5cm}
\end{figure*}

To evaluate the effect of ACC on the bias of the value estimate, we analyze the difference between the value estimate and the corresponding observed return when ACC is applied to TQC.
For each state-action pair encountered during exploration, we compute its value estimate at that time and at the end of the episode compare it  with the actual discounted return from that state onwards. Hence, the state-action pair was not used to update the value function at the point when the value estimate has been computed.
If an episode ends because the maximum number of episode time-steps has been reached, which is 1,000 for the considered environments, we ignore the last $100$ state-action pairs. The reason is that in TQC the value estimator is trained to ignore the episode timeout and uses a bootstrapped target also at the end of the episode. 
We normalize for different value scales by computing the absolute error between the value estimate and the observed discounted return and divide that by the absolute value of the discounted return.
Every 1,000 steps, the average over the errors of the last 1,000 state-action pairs is computed.
The aggregated results in Figure 
\ref{fig:analysis}b)
show that averaged over all environments ACC indeed achieves a lower value error than TQC with the a fixed number of dropped atoms $d$.
This supports our hypothesis that the strong performance of ACC applied to TQC indeed stems from better values estimates.



To better understand the hidden training dynamics of ACC we show in Figure
\ref{fig:analysis}a)
how the number of dropped targets per network $d = d_{max} - \beta$ evolves during training.
Interestingly, the relatively low standard deviation indicates a similar behaviour across runs for a specific environment.
However, there are large differences between the environments which indicates that it might not be possible to find a single hyperparameter that works well on a wide variety of different environments.
Further, the experiments shows that the optimal amount of overestimation correction might change over time during the training even on a single environment.

\subsection{Beyond TQC: Improving TD3 with ACC}

To demonstrate the generality of ACC, we additionally applied it to the actor-critic style TD3 algorithm \cite{td3},
which uses two critics. These are initialized differently but trained with the same target value, which is the minimum over the two targets computed from the two critics.
% This is done to prevent overestimation in the value estimates.
While this successfully prevents the overestimation bias, using the minimum of the two target estimates is very coarse and can instead lead to an underestimation bias.
We applied ACC to TD3 by defining the target for each critic network to be a convex combination between its own target and the minimum over both targets.
Let $Q_i = Q_{\bar{\theta}_i} (s_{t+1}, \pi_{\bar{\phi}} (s_{t+1}) )$, we define the $k$-th critic target
\vspace{-.1cm}
\begin{equation}
\label{eq:td3_target_acc}
    y_k = r + \gamma 
    \Big(   \beta ~ Q_k \nonumber 
     + (1-\beta) \min_{i=1,2} Q_i
    \Big),
\vspace{-.1cm}
\end{equation}
where $\beta \in [0,1]$ is the ACC parameter that is adjusted to balance between under- and overestimation.
The results are displayed in Figure 
\ref{fig:further_eval}b)
and show that ACC also improves the performance of TD3.














\begin{table}[tb]
% \begin{wraptable}{r}{4.5cm}
% \vspace{-25pt}
\ifeccv
\caption{The impact of various fusion methods with 2D GT input. Note that processing all views together with an early fusion as done in FLEX outperforms the other variations by a large margin.}
\fi
\begin{center}
\begin{tabular}{|c|c|}
\hline
%\makecell{\textbf{2D from} $\rightarrow$ \\ \textbf{\#Views}\downarrow$}& \textbf{GT } & \textbf{\cite{iskakov2019learnable}}\\
\textbf{Method} & \textbf{MPJPE } \\
\hline
Averaged $K$ views & 36.4 \\
\hline
Late fusion & 31.0 \\
\hline
FLEX & \textbf{22.9} \\
\hline
\end{tabular}
\end{center}
\ifeccv
\else
\vspace{-15pt}
\caption{The impact of various fusion methods with 2D GT input. Note that processing all views together with an early fusion as done in FLEX outperforms the other variations by a large margin.}
\fi
\label{tab:ablation}
% \end{wraptable}
\vspace{-10pt}
\end{table}

%%% FLIPPED VERSION - WIDER VERSION
% \begin{table}[tb]
% % \begin{wraptable}{r}{4.5cm}
% % \vspace{-25pt}
% \begin{center}
% \begin{tabular}{|c|c|c|c|}
% \hline
% %\makecell{\textbf{2D from} $\rightarrow$ \\ \textbf{\#Views}\downarrow$}& \textbf{GT } & \textbf{\cite{iskakov2019learnable}}\\
% \textbf{Method} & Avg. $K$ views & Late fusion & FLEX\\ \hline
% \textbf{MPJPE } & 36.4 & 31.0 & \textbf{22.9}\\ \hline
% \end{tabular}
% \end{center}
% \vspace{-14pt}
% \caption{The impact of various fusion methods with 2D GT input. Note that processing all views together with an early fusion as done in FLEX outperforms the other variations by a large margin.}
% \label{tab:ablation}
% % \end{wraptable}
% \vspace{-10pt}
% \end{table}

\section{Conclusion}
\label{conclusion}
We present an experimentally-verified simulation framework that can be used to accurately predict the deformations of a pneumatically actuated fish tail with a flexible spine.
Our pipeline can accurately learn material parameters from a quasi-static data sets without having to do expensive and time-consuming material testing. It also eliminates the need to do manual tuning of material constants to get accurate simulation results. The parameters we found are not only within typical range of measured material parameters for our materials, but can be used to successfully predict the behavior of dynamic experiments for different pressure actuation amplitudes and frequencies to within $3\%$ positional error normalized to a actuator length of \SI{10}{cm}. Although we use an isotropic corotated material, which is linear elastic, we find that this model is more sufficient to model large deformations on average giving acceptable displacement results for our engineering application. In these experiments, the damping of the material and the hydrodynamic effects are found to be negligible. This is because the actuation pressures used dominate the deformation compared to losses and hydrodynamic pressure. 

We show a data-driven approach can be used to do simple prediction on a useful performance metric such as thrust force given a suitable hardware setup. However, more work is needed to produce a more robust thrust predictor if the morphology of the actuator changes substantially. We claim that for small design changes such as the choice of silicone or the number of internal chambers this framework can be used to quickly assess the relative merits of each design with a relatively sparse data set of approximately 30 types of different actuation signals.

Our aim is to further progress towards a systematic method by which soft roboticists can simulate and optimize their designs and controllers, whether they be soft fish, manipulators, or other flavors of soft robots. A fast and physically-verified co-optimization method of design and control is the goal.


% % ---- Bibliography ----

% % BibTeX users should specify bibliography style 'splncs04'.
% % References will then be sorted and formatted in the correct style.

\bibliographystyle{splncs04}
\begin{thebibliography}{10}
    \providecommand{\url}[1]{\texttt{#1}}
    \providecommand{\urlprefix}{URL }
    \providecommand{\doi}[1]{https://doi.org/#1}
    
    \bibitem{layernorm}
    Ba, L.J., Kiros, J.R., Hinton, G.E.: Layer normalization. CoRR  (2016)
    
    \bibitem{BoxArima}
    Box, G.E.P., Jenkins, G.M.: Some recent advances in forecasting and control.
      Journal of the Royal Statistical Society. Series C (Applied Statistics)
      (1968)
    
    \bibitem{camacho2013model}
    Camacho, E.F., Alba, C.B.: Model predictive control. Springer science \&
      business media (2013)
    
    \bibitem{cao2018brits}
    Cao, W., Wang, D., Li, J., Zhou, H., Li, L., Li, Y.: Brits: Bidirectional
      recurrent imputation for time series. In: NeurIPS (2018)
    
    \bibitem{cirstea2022triformer}
    Cirstea, R.G., Guo, C., Yang, B., Kieu, T., Dong, X., Pan, S.: Triformer:
      Triangular, variable-specific attentions for long sequence multivariate time
      series forecasting. IJCAI  (2022)
    
    \bibitem{elu}
    Clevert, D., Unterthiner, T., Hochreiter, S.: Fast and accurate deep network
      learning by exponential linear units (elus). In: ICLR (2016)
    
    \bibitem{10.1145/3292500.3330662}
    Fan, C., Zhang, Y., Pan, Y., Li, X., Zhang, C., Yuan, R., Wu, D., Wang, W.,
      Pei, J., Huang, H.: Multi-horizon time series forecasting with temporal
      attention learning. In: SIGKDD (2019)
    
    \bibitem{he2016deep}
    He, K., Zhang, X., Ren, S., Sun, J.: Deep residual learning for image
      recognition. In: CVPR (2016)
    
    \bibitem{huang2017densely}
    Huang, G., Liu, Z., Van Der~Maaten, L., Weinberger, K.Q.: Densely connected
      convolutional networks. In: CVPR (2017)
    
    \bibitem{hyndman2018forecasting}
    Hyndman, R.J., Athanasopoulos, G.: Forecasting: principles and practice. OTexts
      (2018)
    
    \bibitem{Jarrett2020Target-Embedding}
    Jarrett, D., van~der Schaar, M.: Target-embedding autoencoders for supervised
      representation learning. In: ICLR (2020)
    
    \bibitem{jawed2019multi}
    Jawed, S., Rashed, A., Schmidt-Thieme, L.: Multi-step forecasting via
      multi-task learning. In: IEEE Big Data (2019)
    
    \bibitem{tungbcc19}
    Kieu, T., Yang, B., Guo, C., S.~Jensen, C.: Outlier detection for time series
      with recurrent autoencoder ensembles. In: IJCAI (2019)
    
    \bibitem{kitaev2020reformer}
    Kitaev, N., Kaiser, L., Levskaya, A.: Reformer: The efficient transformer. In:
      ICLR (2020)
    
    \bibitem{alexnet}
    Krizhevsky, A., Sutskever, I., Hinton, G.E.: Imagenet classification with deep
      convolutional neural networks. NeurIPS  (2012)
    
    \bibitem{lai2018modeling}
    Lai, G., Chang, W.C., Yang, Y., Liu, H.: Modeling long-and short-term temporal
      patterns with deep neural networks. In: SIGIR (2018)
    
    \bibitem{le2018supervised}
    Le, L., Patterson, A., White, M.: Supervised autoencoders: Improving
      generalization performance with unsupervised regularizers. NeurIPS  (2018)
    
    \bibitem{li2019enhancing}
    Li, S., Jin, X., Xuan, Y., Zhou, X., Chen, W., Wang, Y.X., Yan, X.: Enhancing
      the locality and breaking the memory bottleneck of transformer on time series
      forecasting. NeurIPS  (2019)
    
    \bibitem{tft}
    Lim, B., Arık, S.{\"O}., Loeff, N., Pfister, T.: Temporal fusion transformers
      for interpretable multi-horizon time series forecasting. Int. J. Forecast.
      (2021)
    
    \bibitem{lin2017feature}
    Lin, T.Y., Doll{\'a}r, P., Girshick, R., He, K., Hariharan, B., Belongie, S.:
      Feature pyramid networks for object detection. In: CVPR (2017)
    
    \bibitem{nbeats}
    Oreshkin, B.N., Carpov, D., Chapados, N., Bengio, Y.: N-beats: Neural basis
      expansion analysis for interpretable time series forecasting. In: ICLR (2020)
    
    \bibitem{perslev2019u}
    Perslev, M., Jensen, M., Darkner, S., Jennum, P.J., Igel, C.: U-time: A fully
      convolutional network for time series segmentation applied to sleep staging.
      NeurIPS  (2019)
    
    \bibitem{petit2021u}
    Petit, O., Thome, N., Rambour, C., Themyr, L., Collins, T., Soler, L.: U-net
      transformer: Self and cross attention for medical image segmentation. In:
      International Workshop on MLMI (2021)
    
    \bibitem{qin2017dual}
    Qin, Y., Song, D., Chen, H., Cheng, W., Jiang, G., Cottrell, G.W.: A dual-stage
      attention-based recurrent neural network for time series prediction. In:
      IJCAI (2017)
    
    \bibitem{RangapuramDeepState}
    Rangapuram, S.S., Seeger, M.W., Gasthaus, J., Stella, L., Wang, Y.,
      Januschowski, T.: Deep state space models for time series forecasting. In:
      NeurIPS (2018)
    
    \bibitem{ronneberger2015u}
    Ronneberger, O., Fischer, P., Brox, T.: U-net: Convolutional networks for
      biomedical image segmentation. In: MICCAI (2015)
    
    \bibitem{sagheer2019time}
    Sagheer, A., Kotb, M.: Time series forecasting of petroleum production using
      deep lstm recurrent networks. Neurocomputing  (2019)
    
    \bibitem{salinas2020deepar}
    Salinas, D., Flunkert, V., Gasthaus, J., Januschowski, T.: Deepar:
      Probabilistic forecasting with autoregressive recurrent networks. Int. J.
      Forecast.  (2020)
    
    \bibitem{sezer2020financial}
    Sezer, O.B., Gudelek, M.U., Ozbayoglu, A.M.: Financial time series forecasting
      with deep learning: A systematic literature review: 2005--2019. Applied soft
      computing  (2020)
    
    \bibitem{vaswani2017attention}
    Vaswani, A., Shazeer, N., Parmar, N., Uszkoreit, J., Jones, L., Gomez, A.N.,
      Kaiser, {\L}., Polosukhin, I.: Attention is all you need. In: NeurIPS (2017)
    
    \bibitem{wang2020linformer}
    Wang, S., Li, B.Z., Khabsa, M., Fang, H., Ma, H.: Linformer: Self-attention
      with linear complexity. ArXiv  (2020)
    
    \bibitem{wang2019srn}
    Wang, Y., Tao, X., Shen, X., Jia, J.: Wide-context semantic image
      extrapolation. In: CVPR (2019)
    
    \bibitem{wang2017time}
    Wang, Z., Yan, W., Oates, T.: Time series classification from scratch with deep
      neural networks: A strong baseline. In: IJCNN (2017)
    
    \bibitem{wu2021autoformer}
    Wu, H., Xu, J., Wang, J., Long, M.: Autoformer: Decomposition transformers with
      auto-correlation for long-term series forecasting. NeurIPS  (2021)
    
    \bibitem{zhou2020informer}
    Zhou, H., Zhang, S., Peng, J., Zhang, S., Li, J., Xiong, H., Zhang, W.:
      Informer: Beyond efficient transformer for long sequence time-series
      forecasting. In: AAAI (2021)
    
    \bibitem{zhou2021nnformer}
    Zhou, H.Y., Guo, J., Zhang, Y., Yu, L., Wang, L., Yu, Y.: nnformer: Interleaved
      transformer for volumetric segmentation. ArXiv  (2021)
    
    \bibitem{ZOU201939}
    Zou, X., Wang, Z., Li, Q., Sheng, W.: Integration of residual network and
      convolutional neural network along with various activation functions and
      global pooling for time series classification. Neurocomputing  (2019)
    
    \end{thebibliography}
    


\clearpage
\section{Appendix : Analysis}

\subsection{Additional ablation results on ETTh2 dataset}
\label{appendix:ablation_etth2}

\begin{figure}[!ht]
    \centering
    \subfloat[ETTh2 Univariate\label{fig:ablation_archi_uni_etth2}]{%
      \includegraphics[width=0.40\textwidth]{figs/archi_ablation_uni_ETTh2.png}
    }
    \subfloat[ETTh2 Multivariate\label{fig:ablation_archi_multi_etth2}]{%
      \includegraphics[width=0.40\textwidth]{figs/archi_ablation_multi_ETTh2.png}
    } 
\caption{Figures \ref{fig:ablation_archi_multi_etth2}, \ref{fig:ablation_archi_uni_etth2} illustrates the reduction in MAE loss (y-axis) by  the Yformer architecture in comparison with the Informer baseline for the ETTh2 univariate and multivariate settings respectively. The Yformer ($\alpha=0$) represent the Yformer architecture without the reconstruction loss
}
\label{fig:archi_abltation_etth2}
\end{figure}


\begin{figure}[!ht]
    \centering
    \subfloat[ETTh2 Univariate\label{fig:skipless_ablation_uni_ETTh2}]{%
      \includegraphics[width=0.40\textwidth]{figs/skipless_ablation_uni_ETTh2.png}
    }
    \subfloat[ETTh2 Multivariate\label{fig:skipless_ablation_multi_ETTh2}]{%
      \includegraphics[width=0.40\textwidth]{figs/skipless_ablation_multi_ETTh2.png}
    } 
\caption{Impact of the U-Net connection for the Yformer architecture. The Yformer$^*$ architecture represents the Yformer without the U-Net connection.}
\label{fig:skipless_ablation_2}
\end{figure}


\subsection{Performance variability analysis}

We report the standard deviation values from the multiple Yformer runs for the ETTh2 dataset and compare them with the numbers reported from the Informer baseline \cite{zhou2020informer}. The standard deviation values are quite small across the three runs of the Yformer with multiple initial seed settings illustrating the stability of Yformer across the multiple horizons.

\begin{table}[htbp!]
\caption{Comparison of Yformer model with the second best performing Informer model for performance variability analysis.}
\resizebox{1\textwidth}{!}{%
\begin{tabular}{|c|c|c|c|c|c|c|c|}
\hline
\multicolumn{1}{|c|}{Setting} & \multicolumn{1}{c|}{Model} & Metric & \multicolumn{1}{c|}{24} & \multicolumn{1}{c|}{48} & \multicolumn{1}{c|}{168} & \multicolumn{1}{c|}{336} & \multicolumn{1}{c|}{720} \\ \hline
\multirow{4}{*}{Univariate} & \multirow{2}{*}{Yformer}  & MSE    & $0.082\pm0.004$ & $0.172\pm0.016$ & $0.174\pm0.009$ & $0.224\pm0.038$ & $0.211\pm0.005$ \\ \cline{3-8} 
                                    &                           & MAE    & $0.221\pm0.006$ & $0.334\pm0.014$ & $0.337\pm0.007$ & $0.391\pm0.036$ & $0.382\pm0.005$ \\ \cline{2-8} 
                                    & \multirow{2}{*}{Informer} & MSE    & 0.093         & 0.155         & 0.232         & 0.263         & 0.277         \\ \cline{3-8} 
                                    &                           & MAE    & 0.24          & 0.314         & 0.389         & 0.417         & 0.431         \\ \hline
\multirow{4}{*}{Multivariate}        & \multirow{2}{*}{Yformer}   & MSE    & $0.412\pm0.063$             & $1.171\pm0.027$           & $2.171\pm0.105$            & $2.260\pm0.112$            & $2.595\pm0.131$              \\ \cline{3-8} 
                              &                            & MAE    & $0.498\pm0.049$             & $0.865\pm0.029$           & $1.218\pm0.047$            & $1.283\pm0.009$            & $1.337\pm0.066$              \\ \cline{2-8} 
                              & \multirow{2}{*}{Informer}  & MSE    & 0.720                    & 1.457                   & 3.489                    & 2.723                    & 3.467                    \\ \cline{3-8} 
                              &                            & MAE    & 0.665                   & 1.001                   & 1.515                    & 1.340                     & 1.473                    \\ \hline
\end{tabular}%
}
\end{table}

\newpage

\section{Appendix: Operators}

$\operatorname{\textbf{ProbSparseAttn}}$: Attention module that uses the ProbSparse method introduced in \cite{zhou2020informer}. The query matrix $\overline{\boldsymbol{Q}} \in \mathbb{R}^{L_Q \times d}$ denotes the sparse query matrix with $u$ dominant queries.

\begin{equation}
\begin{aligned}
  \mathcal{A^{\text{PropSparse}}}(\boldsymbol{\overline{Q}}, \boldsymbol{K}, \boldsymbol{V}) &= \text{Softmax}(\frac{\boldsymbol{\overline{Q}}\boldsymbol{K}^T}{\sqrt{d}})\boldsymbol{V}
    % \hat{y}_{fut} &= \hat{y}_{T:T+\tau} \\
    % \hat{y}_{re} &= \hat{y}_{T-|x_i|:T} \\
\end{aligned}
\label{eqn:probattn}
\end{equation}


$\operatorname{\textbf{MaskedAttn}}$: Canonical self-attention with masking to prevent positions from attending to subsequent positions in the future \cite{vaswani2017attention}.

$\operatorname{\textbf{Conv1d}}$: Given $N$ batches of 1D array of length $L$ and $C$ number of channels/dimensions. A convolution operation produces an output: 

\begin{equation}
\begin{aligned}
    \text{out}(N_i, C_{\text{out}_j}) = \text{bias}(C_{\text{out}_j}) +
        \sum_{k = 0}^{C_{in} - 1} \text{weight}(C_{\text{out}_j}, k)
        \star \text{input}(N_i, k)
\end{aligned}
\label{eqn:conv1d}
\end{equation}

For further reference please visit \href{https://pytorch.org/docs/stable/generated/torch.nn.Conv1d.html}{pytorch Conv1D} page 


$\operatorname{\textbf{LayerNorm}}$: Layer Normalization introduced in \cite{layernorm}, normalizes the inputs across channels/dimensions. $\operatorname{LayerNorm}$ is the default normalization in common transformer architectures \cite{vaswani2017attention}. Here, $\gamma$ and $\beta$ are learnable affine transformations.


\begin{equation}
\begin{aligned}
    \text{out}(N, *) = \frac{\text{input}(N, *) - \mathrm{E}[\text{input}(N, *)]}{ \sqrt{\mathrm{Var}[\text{input}(N, *) ] + \epsilon}} * \gamma + \beta
\end{aligned}
\label{eqn:layernorm}
\end{equation}



$\operatorname{\textbf{MaxPool}}$: Given $N$ batches of 1D array of length $L$, and $C$ number of channels/dimensions. A $\operatorname{MaxPool}$ operation produces an output. 

\begin{equation}
\begin{aligned}
    \text{out}(N_i, C_j, k) = \max_{m=0, \ldots, \text{kernel\_size} - 1}
                \text{input}(N_i, C_j, \text{stride} \times k + m)
\end{aligned}
\label{eqn:maxpool}
\end{equation}

For further reference please visit \href{https://pytorch.org/docs/stable/generated/torch.nn.MaxPool1d.html}{pytorch MaxPool1D} page 

$\operatorname{\textbf{ELU}}$: Given an input $x$, the $\operatorname{ELU}$ applies element-wise non linear activation function as shown.

\begin{equation}
\begin{aligned}
    \text{ELU}(x) = \begin{cases}
        x, & \text{ if } x > 0\\
        \alpha * (\exp(x) - 1), & \text{ if } x \leq 0
        \end{cases}
\end{aligned}
\label{eqn:elu}
\end{equation}


$\operatorname{\textbf{ConvTranspose1d}}$: Also known as deconvolution or fractionally strided convolution, uses convolution on padded input to produce upsampled outputs (see \href{https://pytorch.org/docs/stable/generated/torch.nn.ConvTranspose1d.html}{pytorch ConvTranspose1d} page).

\section{Appendix : Hyperparameters}
\label{appendix:hyperparameters}

We follow Informer \cite{zhou2020informer} baseline for all the hyperparameter setting like the convolution kernel size, stride etc. The hyperparameter tuning performed are only for the parameters mentioned below. In order to reproduce the experiments, please use the default Informer/Yformer configurations and adapt only the below mentioned parameters for each horizon.


\begin{table}[htbp!]
\caption{Optimal hyperparameters across different horizon and datasets for the univariate setting. All the remaining hyperparameters are retained from the Informer Model.}
\label{tbl:hyp_univariate}
\centering
\resizebox{1\textwidth}{!}{%
\begin{tabular}{|c|c|c|S|S|S|c|c|}
\hline
Dataset                & Horizon $\tau$& History Length & {Weight Decay} & {Learning Rate} & {Reconstruction Factor $\alpha$} & Batch Size & Encoder Blocks \\ \hline
\multirow{5}{*}{ETTh1} & 24      & 720        & 0            & 0.0001        & 0.7   & 32         & 2              \\ \cline{2-8} 
                      & 48      & 720        & 0         & 0.0001         & 0.7   & 16         & 4              \\ \cline{2-8} 
                      & 168     & 720        & 0            & 0.001         & 0.7   & 32         & 4              \\ \cline{2-8} 
                      & 336     & 720        & 0.05         & 0.0001        & 0.1   & 32         & 4              \\ \cline{2-8} 
                      & 720     & 720        & 0.05         & 0.0001        & 0.7   & 16         & 2              \\ \hline
\multirow{5}{*}{ETTh2} & 24      & 48         & 0            & 0.0001        & 0.7   & 32         & 2              \\ \cline{2-8} 
                      & 48      & 96         & 0.02         & 0.0001        & 0.3   & 32         & 4              \\ \cline{2-8} 
                      & 168     & 336        & 0.02         & 0.001         & 0.3   & 32         & 2              \\ \cline{2-8} 
                      & 336     & 336        & 0.09         & 0.0001        & 0     & 32         & 2              \\ \cline{2-8} 
                      & 720     & 336        & 0.09         & 0.0001        & 0.7   & 16         & 2              \\ \hline
\multirow{5}{*}{ETTm1} & 24      & 96         & 0.02         & 0.0001        & 0.7   & 32         & 4              \\ \cline{2-8} 
                      & 48      & 96         & 0.02         & 0.0001        & 0.7   & 32         & 4              \\ \cline{2-8} 
                      & 96      & 384        & 0.02         & 0.0001        & 0.1   & 32         & 4              \\ \cline{2-8} 
                      & 288     & 384        & 0.02         & 0.001         & 0.7   & 16         & 2              \\ \cline{2-8} 
                      & 672     & 384        & 0.07         & 0.001         & 0.3   & 16         & 2              \\ \hline
\multirow{5}{*}{ECL}   & 48      & 168        & 0            & 0.0001        & 0.7   & 16         & 2              \\ \cline{2-8} 
                      & 168     & 168        & 0.01         & 0.0001        & 0.3   & 16         & 2              \\ \cline{2-8} 
                      & 336     & 168        & 0.01         & 0.0001        & 0.7   & 16         & 2              \\ \cline{2-8} 
                      & 720     & 168        & 0            & 0.0001        & 0.1   & 16         & 2              \\ \cline{2-8} 
                      & 960     & 48         & 0            & 0.0001        & 0.5   & 16         & 4              \\ \hline
\end{tabular}%
}

\end{table}



\begin{table}[htbp!]
\caption{Optimal hyperparameters across different horizon and datasets for the multivariate setting. All the remaining hyperparameters are retained from the Informer Model.}
\label{tbl:hyp_multivariate}
\centering
\resizebox{1\textwidth}{!}{%
\begin{tabular}{|c|c|c|S|S|S|c|c|}
\hline
Dataset                & Horizon $\tau$& History Length & {Weight Decay} & {Learning Rate} & {Reconstruction Factor $\alpha$} & Batch Size & Encoder Blocks \\ \hline
\multirow{5}{*}{ETTh1} & 24      & 48         & 0            & 0.0001        & 0.7   & 32         & 3              \\ \cline{2-8} 
                      & 48      & 96         & 0.02         & 0.001         & 0.5   & 32         & 2              \\ \cline{2-8} 
                      & 168     & 168        & 0.02         & 0.001         & 0.7   & 32         & 2              \\ \cline{2-8} 
                      & 336     & 168        & 0            & 0.0001        & 0.7   & 32         & 4              \\ \cline{2-8} 
                      & 720     & 336        & 0.05         & 0.0001        & 1     & 16         & 2              \\ \hline
\multirow{5}{*}{ETTh2} & 24      & 48         & 0            & 0.0001        & 0.7   & 32         & 2              \\ \cline{2-8} 
                      & 48      & 96         & 0.02         & 0.001         & 0     & 32         & 4              \\ \cline{2-8} 
                      & 168     & 336        & 0.09         & 0.001         & 0.7   & 32         & 2              \\ \cline{2-8} 
                      & 336     & 336        & 0.07         & 0.001         & 0.3   & 32         & 2              \\ \cline{2-8} 
                      & 720     & 336        & 0            & 0.0001        & 0     & 16         & 2              \\ \hline
\multirow{5}{*}{ETTm1} & 24      & 672        & 0            & 0.0001        & 0.7   & 32         & 2              \\ \cline{2-8} 
                      & 48      & 96         & 0            & 0.0001        & 0.7   & 32         & 4              \\ \cline{2-8} 
                      & 96      & 384        & 0.05         & 0.0001        & 0.7   & 32         & 4              \\ \cline{2-8} 
                      & 288     & 672        & 0.02         & 0.001         & 0.5   & 16         & 2              \\ \cline{2-8} 
                      & 672     & 672        & 0.02         & 0.0001        & 0.3   & 16         & 2              \\ \hline
\multirow{5}{*}{ECL}   & 48      & 24         & 0            & 0.0001        & 0.7   & 16         & 3              \\ \cline{2-8} 
                      & 168     & 48         & 0            & 0.0001        & 0.7   & 16         & 3              \\ \cline{2-8} 
                      & 336     & 24         & 0            & 0.0001        & 0.5   & 16         & 2              \\ \cline{2-8} 
                      & 720     & 48         & 0            & 0.0001        & 0.7   & 16         & 2              \\ \cline{2-8} 
                      & 960     & 336        & 0            & 0.0001        & 0.7   & 16         & 2              \\ \hline
\end{tabular}%
}

\end{table}



    

\end{document}
