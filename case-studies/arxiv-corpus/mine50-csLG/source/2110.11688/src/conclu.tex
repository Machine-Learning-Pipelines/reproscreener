

\section{Conclusion and Discussion}
\label{sec:conclusion-and-discussion}

We presented the first differentially private proximal coordinate descent
algorithm for composite DP-ERM.
Using an original approach to analyze proximal CD with perturbed
gradients, we derived optimal upper bounds on the privacy-utility trade-off
achieved by DP-CD. We also prove new lower bounds under a
component-Lipschitzness
assumption,
and showed that DP-CD matches these bounds.
Our results demonstrate that DP-CD strongly outperforms DP-SGD when
gradients' coordinates are imbalanced. Numerical experiments show that DP-CD
also performs very well in balanced regimes.
The choice of coordinate-wise clipping
thresholds is crucial for
DP-CD to achieve good utility in practice, and we provided a simple rule to
set them.



Although DP-CD already achieves good utility when most coordinates have small
sensitivity, our lower bounds suggest that even better utility could be
achieved by dynamically allocating more privacy budget to
coordinates with largest sensitivities.
A promising direction is to design DP-CD algorithms that leverage active set
methods
\citep{yuan2010Comparison,lewis2016Proximal,nutini2017Let,desantis2016Fast,Massias_Gramfort_Salmon18}, which could provide practical alternatives to recent DP-SGD approaches that use a
subspace assumption \citep{zhou2021Bypassing,kairouz2021Nearly}.
Finally, we believe that adaptive clipping techniques
\citep{pichapati2019AdaCliP,thakkar2019Differentially} may help to further
improve the practical performance of DP-CD when coordinate-wise
smoothness constants are more balanced.




