% \section{Double Descent from an Implicit Label Noise Perspective}
% \section{Adversarial perturbation can cause label noise implicitly}
% \section{Explore Label Noise in Adversarial Training}
% \label{sect:reason}

% \chengyu{Make sure we show the point that adversarial training only \emph{magnifies} the label noise and thus makes the robust overfitting more evident.}

% \chengyu{Avoid mention double descent to the best}

% \chengyu{maybe also no need to define ``implicit label noise''. Then people will not require us to demonstrate such a new type of label noise in the reality.}

% In this section, we present a novel perspective to understand the double descent in adversarial training. The implicit label noise is originated from the improper labeling of the adversarial examples and can induce double descent in adversarial training.

% In this section, we first show that traditional adversarial perturbation does not cause label noise directly. We then argue that adversarial perturbation cause label noise implicitly, but significantly.

% To understand intriguing behaviors of adversarial training such as robust overfitting we focus on its training set. We first show that label noise does not explicitly exist in the adversarially augmented training set. We then argue that label noise will implicitly exist in the adversarially augmented training set due to the distribution mismatch and improper label construction in the common practice.


 




% version https://git-lfs.github.com/spec/v1
oid sha256:5abff5344bbf79a2c8780edff8201c0be72fea4edcd76fdc2c4fc9345f72d151
size 5002


version https://git-lfs.github.com/spec/v1
oid sha256:9956ecf8f72f8cc864de3724da585c24c2cb6e97975537af55723da00e24101f
size 19514


% \subsection{Equivalence between implicit label noise and label flipping noise}
% \subsection{Influence of implicit label noise in adversarial training}
% \subsection{Influence of implicit label noise}
% \subsection{Implicit label noise is a specific type of label noise}
% \smallsection{Connection and difference between implicit label noise and conventional label noise}
\smallsection{Intuitive interpretation of label noise in adversarial training}
% \chengyu{If no section 4.1, can also remove this section}
% In this section, we connect the implicit label noise to a more familiar definition of label noise and show it can have a significant impact.
%     \begin{proposition}[Implicit label noise is equivalent to instance-dependent and class-dependent label noise]
%     \label{proposition:label-noise}
%     Let $p_e(j, x) = P(\tilde{Y}\ne j | Y = j, x)$ be a typical definition of label noise which depends on both the class $j$ and input $x$. Then implicit label noise is equivalent to
%     \begin{equation}
%       % p(Y\ne Y^* | x) = 1 - \sum_j p(Y= j | Y^* = j, x) p(Y^* = j|x),
%       P(\tilde{Y}\ne Y | x) = 1 - \sum_j (1 - p_e(j, x)) P(Y = j|x),
%     \end{equation}
%   % It can easily seen that if $p(Y\ne Y^* | x) > 0$, $p(Y\ne j | Y^* = j, x) > 0$ for some $j$.
%   \end{proposition}
%   \begin{proof}
%   See Appendix~\ref{sect:label-noise-more-proof}.
%   \end{proof}
We introduce a simple example to help understand the emergence of label noise in adversarial training.
% Towards an intuitive understanding of implicit label noise, we introduce a simple example to discuss the differences and connections between implicit label noise and convention label noise.
% We now try to quantify the implicit label noise given its connection with typical label noise.
% \chengyu{We now discuss a simplified example of implicit label noise, and discuss its connection and difference with conventional label noise.}
\begin{example}
% [Quantify the implicit label noise]
[Label noise due to a symmetric distribution shift]
\label{example:label-noise-influence}
% We would like to note that the adversarial training setting would amplify the impact of the implicit label noise, since it adds perturbations to every training sample. 
% In fact, Theorem 3.1 in our main paper, which lower-bounds the implicit label noise by the distance between the assigned label distribution and the true label distribution, provides a way to intuitively quantify the implicit label noise. 
Let $\mathcal{D}=\{(x_i, y_i)\}_{i\in[N]}$ be a clean labeled training subset where all inputs $x_i=x$ are identical and have a one-hot true label distribution, i.e., $P(Y|x) = \mathbf{1}_y$. %, and there is no label noise in $\mathcal{D}$, i.e. $y = \tilde{y}$. 

We now construct an adversarially augmented training subset $\mathcal{D'} = \{(x'_i, \tilde{y}'_i)\}_{i\in[N]}$, where $\tilde{y}' = y$ and $x'$ is generated based on adversarial perturbation that distorts the true label distribution symmetrically. Specifically,
$$
P(Y'= j' | x') =
\begin{cases} 
1 - \eta, & \text{if}~~j = y, \\
\eta/(K-1), & \text{otherwise}. \\
\end{cases}
$$
Then by Lemma~\ref{theorem:implicit-label-noise} 
% we have $P(\tilde{Y}' \ne Y' | x') \ge \eta$.
% we have $E_{j'} P_e(j', x') \ge \eta$.
we have $p_e (\mathcal{D}')\gtrsim \eta$.
% \ge % \left\| P(Y^*|x) -  P(Y^*_\delta|x_\delta).\right\|_{\text{TV}}
% which is equivalent to $p_e(j,x) = \sigma$ by Proposition~\ref{proposition:label-noise}. % , meaning $10\%$ label noise is injected in $\mathcal{S}$.
% which means the label noise injected in $\mathcal{S}_\delta$ is at least $10\%$  by Proposition~\ref{proposition:label-noise}.
% the total variation distance between these distributions is 0.1, which means the implicit label noise is at least 0.1. This is already equivalent to 10\% label noise based on the connection between implicit label noise and the (instance-wise) probabilistic definition of label flipping noise (see Remark 3.2 in our paper).
% \jingbo{what is this 10? I didn't quite get this.}
\end{example}
   
% \chengyu{Talk about observation distribution, noisy process and true distribution?}

One can find that there is indeed $\eta$ faction of noisy labels in $D'$. This is because if we sample the labels of $x'$ based on its true label distribution, we expect $1 - \eta$ faction of $x'$ are labeled as $y$, while $\eta$ fraction of $x'$ are labeled to be other classes. However, in $D'$, all $x'$ are assigned with label $y$ , which means $\eta$ fraction of $x'$ are labeled incorrectly. In realistic datasets we can consider inputs with similar features for such reasoning.

% \chengyu{Add a interpretation from population view?? I remember there is a case about dogs and cats in previous revision.}
% Conventionally speaking, label noise is perceived as the fraction of the noisy labels in the training set, i.e. the assigned labels that are different from their corresponding true labels. However, in the adversarially augmented training set no assigned label is noisy since $\tilde{y}' = y'$ 
% ~\footnote{Recall $\tilde{y}' = \argmax_j P(\tilde{Y'}=j|x')$ and $y' = \argmax_j P(Y'=j|x')$}
% for every augmented input $x'$. Yet, rather counter-intuitively, at least $\eta$ label noise exists in $\mathcal{D}'$, which is due to the fact that every input is now more likely to be mislabeled after adversarial perturbation.


% \chengyu{Difference from a process view. There is no noisy process defined, only the final assigned distribution after noisy process is known. But as long as the final distribution is different, suggests the annnotation must go through some unknown noisy process.}
% \chengyu{Why same argmax doesn't mean there is no label noise? Because label noise is always associated with a noisy random process. Because the change of the underlying true distribution should be reflected in the sampled labels, otherwise there must be some noisy process an annotator goes through.}

The above example also shows that label noise in adversarial training may be stronger than one's impression. Even a slight distortion of the true label distribution, e.g. $\eta=0.1$, will be equivalent to at least $10\%$ noisy label in the training set. This is because the true label distribution of every training input is distorted, resulting in significant noise in the population. 
% \sout{Such example also implies that even static adversarial perturbation~\footnote{namely the adversarial perturbation is added to the training set only once and the standard training is performed subsequently} can produce clear double descent as shown in Appendix~\ref{sect:exp-static}.} Therefore we believe implicit label noise can be an important source of label noise that makes double descent more evident in adversarial training.



    % An informal proof can be sketched from a frequentist's view and help the understanding of implicit label noise.
    % % One can interpret the implicit label noise in a frequentist's view.
    % Say there are $M$ identical copies of $x_\delta$ in the training set $\mathcal{D}_\delta$, with their true labels and traditional adversarial labels distributing according to $p(y^*_\delta | x_\delta)$ and $p(\tilde{y}_\delta| x_\delta)$, respectively.
    % % by Remark~\ref{remark:common-practice} and Assumption~\ref{assumption:clean-dataset}, respectively.
    % % The true label of $x_\delta$ is sampled based on $p(y_\delta |x + \delta)$, the assigned label is sampled based on $p(y|x)$. 
    % The number of copies that have the same true label and assigned label is $ M \sum_j \min \{p(\tilde{Y}_\delta=j|x_\delta), p(Y^*_\delta=j |x_\delta)\}$.
    % The fraction of label noise exists in $\mathcal{D}_\delta$ is thus $1 - \sum_j \min \{p(\tilde{Y}_\delta=j|x_\delta), p(Y^*_\delta=j|x_\delta)\} = \|p(\tilde{y}_\delta|x_\delta) - p(y^*_\delta|x_\delta) \|_{\text{TV}}$  by the definition of the total variation distance.
    


% ----------------------------------------------    
% ----------------------------------------------
\smallsection{Dependence of label noise in adversarial training}
\label{sect:dependence-label-noise}
    % We now show the implicit label noise in adversarial training depends on the perturbation radius and the data quality given mild assumptions on the probabilistic classifier.
    % and show how these factors affect double descent in adversarial training.
    % \begin{corollary}[Dependence of implicit label noise]
    % % Assume the true label distribution $P(Y^*|x)$ is locally convex around $x$ and can be asymptotically described as
    % % \chengyu{Twice differentiable is also enough. Use Taylor approximate in first order to express gradient with perturbed distance. But here need to show it is negative correlation, meaning norm after hessian multiplication is negative correlated}
    % Assume $f(x)_y$ is $L$-locally Lipschitz around $x$ with Hessian bounded below. Let $m = \inf_{z \in \mathcal{B}_\varepsilon(x)} \sigma_{\min} (\nabla^2 f(z)_y) > 0$, we have
    % \begin{equation}
    % \label{theo:label-noise-dependence}
    %     % \|p_Y - p_{Y'}\|_{\text{TV}} 
    %     P(\tilde{Y’}\ne Y' | x')
    %     \ge
    %     \frac{\varepsilon}{2} (1 - q(x)) \frac{m}{L}  - \frac{\varepsilon^2}{4} M,
    % \end{equation}
    % where $q(x)$ is the data quality.
    % % \begin{equation}
    % % \label{eq:label-noise-assumption}
    % %     %  \|\nabla_x~ p(y^*=j|x)\| \propto 
    % %     % \begin{cases}
    % %     % 1 -  p(y^*=j|x),& \quad  p(y^*=j|x) \to 1\\
    % %     %      p(y^*=j|x),& \quad  p(y^*=j|x) \to 0,
    % %     % \end{cases}   
    % %      \|\nabla_x~ P(Y^*=j|x)\| \propto 1 -  P(Y^*=j|x),
    % % \end{equation}
    % % as $P(Y^*=j|x) \to 1$.
    % % % $\max_j p(y^*=j|x) \approx 1$, % Given in assumption
    % % % where $j^* = \argmax~p(y=j|x)$, given by above lemma 2.2
    % % We have
    % % $$
    % % \underline{\min}~P(\tilde{Y}_\delta \ne Y^*_\delta | x_\delta) \propto \varepsilon (1 - q(x)),
    % % $$
    % % where $\underline{\min}$ means the lower bound of the minimum label noise, and $q(x)$ is the data quality (\ref{definition:data-quality}).
    % \end{corollary}
    % \begin{proof}
    % % See Appendix~\ref{sect:label-noise-more-proof}, where we also show that Assumption (\ref{eq:label-noise-assumption}) holds true for a Gaussian mixture model.
    % Let $f(x) = f(x)_y$. Assume $f$ is locally Lipschitz around $x$. Let $x^* = \argmin_{z \in X, f(z) = 1} \|x - z\|$ (local maximum closest to $x$). Because $x^*$ is the local maximum and $f$ is continuously differentiable, $\nabla f(x^*) = 0$, thus
    % $$
    % \nabla f(x) 
    % & = \nabla f(x^*) + \nabla^2 f(z) (x - x^*) = \nabla^2 f(z) (x - x^*).
    % $$
    % Therefore we have
    % $$
    % \begin{aligned}
    % \|\nabla f(x)\| 
    % & =  \|\nabla^2 f(z) (x - x^*) \| \\
    % & \ge \sigma_{min} (\nabla^2 f(z)) \|x - x^*\| \\
    % & \ge \sigma_{min} (\nabla^2 f(z)) \frac{|f(x^*) - f(x)|}{L(f)} \\
    % & = \frac{\sigma_{min} (\nabla^2 f(z))}{L(f)} |1 - f(x)| \\
    % & = \frac{\sigma_{min} (\nabla^2 f(z))}{L(f)} |1 - q(x)| \\
    % \end{aligned}
    % $$
    % \end{proof}
    % The above theorem shows that, % when the data quality of the clean example is relatively high,
    % the probability of the true label of the clean example is relatively high, 
    Theorem~\ref{theo:main} shows that
    the label noise in adversarial training is proportional to (1) the perturbation radius (2) the data quality. 
    % Larger perturbation radius and low data quality induces higher implicit label noise, which echos the empirical observations made in Appendix~\ref{sect:double-descent-adversarial}.
    % Since implicit label noise modulates the double descent, and by Theorem~\ref{theo:label-noise-perturbation} it depends on the perturbation radius and data quality, the double descent in adversarial training should strongly correlate with the perturbation radius and data quality. 
    % Indeed, it has been observed respectively that small perturbation radius will not induce robust overfitting~\citep{Dong2021ExploringMI}, and high-quality data will not induce robust overfitting~\citep{Dong2021DataPF}.
    % , which will subsequently affect the double descent curves.
    Considering label noise can be an important source of variance in the generalization of deep neural networks~\citep{Nakkiran2020DeepDD, Yang2020RethinkingBT}, such dependence of label noise explains the intriguing observations in the literature that robust overfitting (or epoch-wise double descent) in adversarial training will vanish with small perturbation radii~\citep{Dong2021ExploringMI} or high-quality data~\citep{Dong2021DataPF}. 
    We conduct more controlled experiments to verify this correlation empirically, as shown in Figure~\ref{fig:dependence-perturbation-quality}, 
    % More controlled experiments are conduct in Appendix~\ref{sect:double-descent-adversarial} to verify this correlation empirically.
    
    \begin{figure*}[!ht]
      \centering
      \includegraphics[width=0.8\textwidth]{figures/dependence-perturbation-quality.pdf}
      \caption{(Left) Dependence of robust overfitting on the perturbation radius. A training subset of size 5k is randomly sampled to speed up the training.
      % As the perturbation radius (used for both training and testing) employed in adversarial training increases, both the epoch-wise and model-wise double descent become more prominent.
      $\varepsilon = 0/255$ indicates the standard training where no double descent occurs. 
      (Right) Dependence of robust overfitting on the data quality with a fixed perturbation radius ($\varepsilon = 8/255$). To construct a training subset with high data quality, we first calculate the predictive probability based on an ensemble of multiple models. We then rank all training examples based on the predictive probability and select the top-k ones.
      The curves are smoothed by a window of $5$ epochs to reduce overlapping.
      Here we conduct PGD training on CIFAR-10 with WRN-28-5. 
      More experiment details can be found in the Appendix. % ~\ref{sect:double-descent-adversarial}.
      % As the quality of training data in adversarial training degrades, both the epoch-wise and model-wise double descent become more prominent. In the epoch-wise double descent figure, We smooth each curve by a window of 5 epochs to reduce the overlapping area. For the model-wise double descent the test error at the last checkpoint (solid line) and the test error at the best checkpoint (dashed line) are both shown. 
    %   \chengyu{change data quality to $1 - f_\theta(y|x)$}
      }
      \vspace{-1em}
    \label{fig:dependence-perturbation-quality}
    \end{figure*}

version https://git-lfs.github.com/spec/v1
oid sha256:fb0eb5cb2492c8e56a702e69f7a49008cf0d73cbdb20a740d3ec707281443a22
size 9155





    



% ----------------------------------------------
% ----------------------------------------------
% \subsection{Implicit label noise increases variance in adversarial training}
% \subsection{Implicit label noise induces double descent}
% \label{sect: noise-variance}

% \note{replace this part with results on cifar-10h}
% \chengyu{This should be okay. Augmentation is a good way to show implicit label noise. We just probably need to show before that implicit label noise is nothing but label noise if in a large dataset.}



% % We now show the implicit label noise induces double descent in adversarial training. 
% \chengyu{Move to related work}
% In standard training, the effect of label noise on double descent has been rigorously studied based upon both analytical settings~\citep{Mei2019TheGE, Hastie2019SurprisesIH, Deng2019AMO, Belkin2020TwoMO} and bias-variance analyses~\citep{Jacot2020ImplicitRO, Yang2020RethinkingBT, dAscoli2020DoubleTI}.
% % , with an emphasis on model-wise double descent. 
% % Here, we follow a bias-variance understanding and empirically show that the implicit label noise can promote variance during training and thus produces the epoch-wise double descent.
% Since implicit label noise is just a special case of label noise (Remark~\ref{remark:label-noise}), 
% % \jingbo{do you mean that label flipping is a special case of implicit label noise?} \chengyu{See Remark 2.2}
% and adversarial training can be viewed as standard training on an augmented dataset (Equation~(\ref{eq:outer-minimization})), % it can be inferred that implicit label noise will cause double descent in adversarial training. 
% it can be inferred that implicit label noise will increase the variance and make an evident double descent in adversarial training.
% %  we will not repeat the analyses here, but instead demonstrate in a scenario other than adversarial training where implicit label noise causes double descent.
% To demonstrate this in a straightforward way, in Figure~\ref{fig:dependence-variance} \jingbo{we have a undefined reference here} we employ standard training on a dataset augmented by fixed adversarial perturbation and show it can indeed produce double descent.



% To clearly show the implicit label noise promotes the variance and induces double descent, we employ \emph{adversarial augmentation}, namely the adversarial perturbation is generated by a surrogate model and applied to the training set only once. 
% Standard training is then conducted on the augmented training set to simulate the effect of adversarial training. 
% Such experiment excludes the possibility that the double descent (or robust overfitting) results from the variation of the adversary strength during training.

% Figure~\ref{fig:dependence-variance} shows the adversarial augmentation induces the epoch-wise double descent similar to adversarial training. 
% We further conduct the training on multiple independent training subsets and perform a bias-variance decomposition of the $0$-$1$ loss (see Appendix~\ref{sect:bias-variance-0-1loss}). 
% One can find that the bias almost monotonically decreases throughout the training while the variance increase significantly when the overfitting happens and larger perturbation radius will induces higher variance. 

% \chengyu{Maybe just show figure (a), remove bias-variance analyses and move to appendix.}
% \begin{figure*}[htbp]
%   \centering
%   \includegraphics[width=0.95\textwidth]{figures/reason-variance.pdf}
%   \caption{``Risk'' (Average test error over independent training subsets) obtained when training on an adversarially augmented dataset, as well as the ``Bias'' and ``Variance'' following a bias-variance decomposition of the 0-1 loss. Detailed experiment settings can be found in Appendix~\ref{sect: exp-ad-augment}.
%   }
% \label{fig:dependence-variance}
% \end{figure*}




% Finally, we note that our above analyses echo the existing works. 
% Since implicit label noise modulates the double descent, and by Theorem~\ref{theo:label-noise-perturbation} it depends on the perturbation radius and data quality, the double descent in adversarial training should strongly correlate with the perturbation radius and data quality. Indeed, it has been observed respectively that small perturbation radius will not induce robust overfitting~\citep{Dong2021ExploringMI}, and high-quality data will not induce robust overfitting~\citep{Dong2021DataPF}.
% for small perturbation radius and high-quality dataset, the double descent may not be observed in adversarial training, which echos the recent empirical observation made in \citet{Dong2021ExploringMI} and \citet{Dong2021DataPF} respectively.