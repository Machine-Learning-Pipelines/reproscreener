\section{Comparison to kernel methods for causal scalars}\label{sec:detail}

%\subsection{Overview}

We now connect our kernel methods for causal functions with related kernel methods for treatment effects. Recall the definition $\theta_0^{ATE}(d)=E\{Y^{(d)}\}$. We allow treatment to be continuous, so $\theta_0^{ATE}$ is a causal function called the dose response. In related work, treatment is binary, so $\theta_0^{ATE}$ is a vector of two causal scalars $\theta^{ATE}_0(1),\theta^{ATE}_0(0)$ whose difference is the treatment effect. 

We clarify three points. (i) There is a sense in which our algorithms generalize known estimators for treatment effects to new estimators for causal functions. (ii) A treatment effect is a bounded functional over $\mathbb{L}^2$ with a balancing weight representation, while a response curve is not. Our key insight is that a response curve is a bounded functional over the RKHS $\mathcal{H}$, which is a subset of $\mathbb{L}^2$. (iii) Our theoretical contribution is a new, uniform analysis of response curves. The analysis goes beyond pointwise approximation of response curves by local treatment effects.

%\subsection{Balancing weight representation}

We begin by reviewing the theory of balancing weights, which are popular in causal inference with binary treatments. For clarity, in this section we emphasize a fixed treatment value by writing $d^*\in \mathcal{D}$.  The following representation is well known.

\begin{proposition}[Existence for treatment effects; Point 3.1 of \cite{hernan2010causal}]\label{prop:balance_exists}
Suppose selection on observables (stated in Supplement~\ref{sec:id}) holds and treatment is binary. Fix $d^*\in\mathcal{D}$. If $\text{\normalfont pr}(D=d^* \mid X)$ is bounded away from zero almost surely, then there exists balancing weight $\alpha_0\in \mathbb{L}^2$ such that for all $\gamma\in \mathbb{L}^2$, $\int \gamma(d^*,x)\mathrm{d}\text{\normalfont pr}(x)=\langle \gamma,\alpha_0 \rangle_{\mathbb{L}^2}$. In particular, $\theta_0^{ATE}(d^*)=\int y\alpha_0(d,x)\mathrm{d}\text{\normalfont pr}(d,x,y)=\langle \gamma_0,\alpha_0 \rangle_{\mathbb{L}^2}$ and the balancing weight is $\alpha_0(d,x)=1(d=d^*)/\text{\normalfont pr}(D=d^* \mid x)$.
\end{proposition}

In summary, a treatment effect has two representations: the primal representation of Lemma~\ref{theorem:id_treatment} as a partial mean of the regression $\gamma_0(d,x)=E(Y \mid D=d,X=x)$, and the dual representation of Proposition~\ref{prop:balance_exists} as a reweighting of the outcome $Y$ using the balancing weight $\alpha_0(d,x)=1(d=d^*)/\text{\normalfont pr}(D=d^* \mid x)$. Clearly, the two representations are related by the law of iterated expectations. Moreover, from the closed form of $\alpha_0$, we require $\text{\normalfont pr}(D=d^* \mid X)>0$ for $\alpha_0$ to exist. This property keenly relies on the treatment being discrete. Indeed, it is well known that a balancing weight representation does not exist for response curves.

\begin{proposition}[Non-existence for response curves \cite{van1991differentiable,newey1994asymptotic}]\label{prop:balance_dne}
Suppose selection on observables (stated in Supplement~\ref{sec:id}) holds and treatment is continuous. Fix $d^*\in\mathcal{D}$. Even if the density $f(d^* \mid X)$ is bounded away from zero almost surely, there does not exist a balancing weight $\alpha_0\in \mathbb{L}^2$ such that for all $\gamma\in \mathbb{L}^2$, $\int \gamma(d^*,x)\mathrm{d}\text{\normalfont pr}(x)=\langle \gamma,\alpha_0 \rangle_{\mathbb{L}^2}$. In particular, without further restrictions, there does not exist $\alpha_0\in \mathbb{L}^2$ such that  $\theta_0^{ATE}(d^*)=\int y\alpha_0(d,x)\mathrm{d}\text{\normalfont pr}(d,x,y)=\langle \gamma_0,\alpha_0 \rangle_{\mathbb{L}^2}$.
\end{proposition}

 Whereas a binary treatment effect is a bounded functional over $\mathbb{L}^2$ with a balancing weight representation, a dose response is not a bounded functional over $\mathbb{L}^2$ and does not have a balancing weight representation in the classic sense. From a functional analytic perspective, this discrepancy is the reason why the problems we study are nonparametric whereas previous work on kernel methods for treatment effects are semiparametric. See Supplement~\ref{sec:balancing} for discussion. 



Our key insight is that the dose response is a bounded functional over the RKHS $\mathcal{H}$, which is a subset of $\mathbb{L}^2$. This fact follows from three simple observations: (i) the dose response is a partial mean; (ii) in the RKHS, a partial mean can be reformulated as a kind of evaluation; and (iii) the RKHS $\mathcal{H}$ is the subset of $\mathbb{L}^2$ for which evaluation is a bounded functional. Through this lens, Theorem~\ref{theorem:representation_treatment} shows that there can exist a function $\tilde{\alpha}_0\in\mathcal{H}$ such that $\theta_0^{ATE}(d)=\langle \gamma_0,\tilde{\alpha}_0 \rangle_{\mathcal{H}}$ even when there does not exist a function $\alpha_0\in \mathbb{L}^2$ such that $\theta_0^{ATE}(d)=\langle \gamma_0,\alpha_0 \rangle_{\mathbb{L}^2}$.

%\subsection{Closed form solution}

What is the relationship between between our kernel methods for causal functions and existing kernel methods for treatment effects? There is a sense in which our dose response estimator, which is the simplest case of our framework, is a relaxation of kernel balancing weight estimators from binary treatment to continuous treatment. We formalize this connection as follows.

\begin{corollary}[Relaxation of balancing weight estimators]\label{cor:connect}
Suppose treatment is binary, and take $k_{\mathcal{D}}(d,d')=1(d=d')$ to be the treatment kernel. Then $\hat{\theta}^{ATE}(d)=n^{-1}\sum_{i=1}^n Y_i \hat{\alpha}_i$, where $\hat{\alpha}_i=\hat{\alpha}(D_i,X_i)$ and $\hat{\alpha}$ is a ridge regularized estimator of $\alpha_0 \in \mathbb{L}^2$.
\end{corollary}

See Supplement~\ref{sec:balancing} for the proof. The balancing weight estimator $\hat{\alpha}$ minimizes a generalized balancing weight loss with ridge regularization; see \cite[eq. 8]{kallus2020generalized}, \cite[eq. 1]{hirshberg2019minimax}, and \cite[Definition 3.2]{singh2021debiased} for various formulations. Corollary~\ref{cor:connect} provides intuition for our tensor product RKHS construction. Our product kernel construction ensures that using the binary treatment kernel amounts to subsetting, which recovers previous algorithms. The tensor product RKHS provides a natural way to relax binary treatment to continuous treatment while retaining computational tractability.

As argued in Proposition~\ref{prop:balance_dne}, the balancing weight $\alpha_0\in \mathbb{L}^2$ does not exist for the dose response. Nonetheless, our key insight in Theorem~\ref{theorem:representation_treatment} is that a function $\tilde{\alpha}_0\in\mathcal{H}$ does exist to serve a similar purpose. By combining the partial mean perspective with the technique of kernel mean embedding, we demonstrate that our framework easily extends to conditional nonparametric causal functions, e.g. the heterogeneous response curve $\theta_0^{CATE}(d,v)$, which are substantially more challenging than unconditional nonparametric causal functions, e.g. the dose response $\theta^{ATE}_0(d)$.

Perhaps the most surprising consequence of our construction is the closed form solution for causal functions. In particular, each closed form solution is a reweighting of the observed outcomes with empirical weights that we characterize even though a population balancing weight in $\mathbb{L}^2$ does not exist. 
In sum, previous work \cite{kallus2020generalized,hirshberg2019minimax,singh2021debiased} on kernel methods for treatment effects is inherently tied to the $\mathbb{L}^2$ population balancing weight perspective; our algorithms apply the conceptual framework of kernel methods to new classes of causal functions. The following corollary reinterprets Algorithm~\ref{algorithm:treatment} through this lens. 

\begin{corollary}[Closed form reweighting even when balancing weight does not exist]\label{cor:extend}
Suppose treatment is continuous, with $k_{D}$ that is continuous and bounded. Then 
$\hat{\theta}^{ATE}(d)=n^{-1}\sum_{i=1}^n Y_i\hat{\alpha}_i^{ATE}$,
$\hat{\theta}^{DS}(d,\tilde{\text{\normalfont pr}})=n^{-1}\sum_{i=1}^{n}Y_i\hat{\alpha}_i^{DS}$,  $\hat{\theta}^{ATT}(d,d')=n^{-1}\sum_{i=1}^n Y_i\hat{\alpha}_i^{ATT}$, and  
$\hat{\theta}^{CATE}(d,v)=n^{-1}\sum_{i=1}^n Y_i\hat{\alpha}_i^{CATE} $, where the weights have closed form solutions given in Supplement~\ref{sec:balancing}. Likewise for incremental functions, e.g. $\hat{\theta}^{\nabla:ATE}(d)=n^{-1}\sum_{i=1}^n Y_i\hat{\alpha}_i^{\nabla:ATE}$.
\end{corollary}

%\subsection{Global versus local estimation}

Each of our proposed causal function estimators is global. In particular, within Corollary~\ref{cor:extend}, the weights $(\hat{\alpha}_j^{ATE},\hat{\alpha}_j^{DS},\hat{\alpha}_j^{ATT},\hat{\alpha}_j^{CATE})$ $(j=1,...,n)$ depend on all of the observations as refracted through the ridge regularized empirical covariance and the kernel evaluations $k(D_i,d)$. This approach departs from a localization approach to causal functions whereby the weight assigned to each observation is determined by Nadaraya--Watson smoothing \cite{kennedy2017nonparametric,kallus2018policy,colangelo2020double,chernozhukov2022debiased}. In the localization approach, the weight is $k^{NW}\{(D_i-d)/h\}$ where $k^{NW}$ is a Nadaraya--Watson kernel and $h$ is a vanishing bandwidth. By contrast, we consider a fixed kernel and vanishing ridge regularization.

The global perspective has three main advantages. First, our estimators can be computed once and evaluated at any value of a continuous treatment.  By contrast, a localized estimator is a computationally intensive procedure that must be reimplemented at any treatment value. Second, our estimators are constructed from function classes with designed-in smoothness properties, which leads to smoother and therefore more plausible response curves. We compare our smooth estimate with a jagged localizing estimate in the program evaluation of Supplement~\ref{sec:experiments}. Third, we prove uniform consistency of response curves, whereas localizations of previous results would only lead to pointwise consistency. These uniform guarantees are the focus of the next section.
