\label{sec:app_codes}
%We include 3 packages in the source codes: \emph{closeformhessian}, \emph{hessian\_eigenthings}, and \emph{eigen\_pac\_bayes}.

%\emph{closeformhessian} gives the methods to calculate matrices we used in our experiments, like $\E[\mM]$ and $\E[\vx\vx^T]$, and others mentioned in \cref{sec:appendix_derivation_main}. Most of the computational functions are in decomp.py. The names should be self-explanatory, except that we use some old namings (U for $\mG$, UTAU for $\mM$).

%\emph{hessian\_eigenthings} is used to calculate eigenvalues and eigenvectors as described in \cref{sec:appendix_eigencomp}, modified from \citet{hessian-eigenthings}. 

The complete set of experiment code for this paper are included in the folder \textbf{Dissecting\_Hessian\_Codebase} of the supplemental material.

All of our experiment code are are based on Python and PyTorch \citep{NEURIPS2019_9015}. The README file in the code base folder provides a brief introduction to the structure of the code base along with required environments.
% We include 3 packages in the source codes: \emph{eigen\_pac\_bayes}, \emph{hessian\_eigenspace\_overlap}, and \emph{LowRankOutHessian}

% \emph{eigen\_pac\_bayes} is the package to compute the PAC-Bayes bounds described in \cref{sec:pac} and \cref{sec:appendix_pac}. The \emph{hpb} folder contains the core files for the package and the \emph{algos} folder contains packages described above. Files in the main folder are example codes to run this package. Files ended with run.py and run\_2.py are used for datasets with $\pm 1$ labels like MNIST-2. In most cases, files ended with run.py are for networks with 1 hidden layer and files ended with run\_2.py are for networks with 2 hidden layers. Files ended with run\_10class.py are used for datasets with multiple classes like MNIST. They do not actually require the number of classes to be 10.

% \emph{LowRankOutHessian} examines the structure of the output Hessian $\E[\mM]$. The results of these experiments are in \cref{sec:emp_outlier} and \cref{sec:appendix_low_rank}.

% \emph{hessian\_eigenspace\_overlap} contains our main experiment codes, basically everything except those in the other 2 packages. The codes are mainly used to investigate the structure of eigenspace and eigenvectors of the layer-wise Hessian.

% In the \emph{algos} folder in \emph{hessian\_eigenspace\_overlap}, there are several packages contain the core algorithms used in our experiments. \emph{closeformhessian} gives the methods to calculate matrices we used in our experiments, like $\E[\mM]$ and $\E[\vx\vx^T]$, and others mentioned in \cref{sec:appendix_derivation_main}. Most of the computational functions are in decomp.py. The names should be self-explanatory, except that we use some old namings (U for $\mG$, UTAU for $\mM$). \emph{hessian\_eigenthings} is used to calculate eigenvalues and eigenvectors as described in \cref{sec:appendix_eigencomp}, modified from \citet{hessian-eigenthings}. \emph{pyhessian} is the package in \citet{yao2019pyhessian}. \emph{cfhag} gives the similar methods as in the \emph{closeformhessian} but for convolutional layers.

% The base codes used for training and testing neural networks are in the \emph{CODEBASE} folde in \emph{hessian\_eigenspace\_overlap}.