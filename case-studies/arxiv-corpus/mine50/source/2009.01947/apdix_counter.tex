\section{Counterexample for \thresam with Non-monotone Submodular Functions}
\label{sec:counterexample}
\shortciteS{Fahrbach2018} proposed a subroutine, 
\thresam, which returns a solution
$S \subseteq \mathcal{N}$ that $\ex{f(S)/S} \ge (1-\epsi)\tau$
within logarithmic rounds and linear time. 
The full pseudocode for \thresam is given in Alg.~\ref{alg:thresh}.
The notation $\mathcal U ( S, t )$ represents the uniform distribution 
over subsets of $S$ of size $t$. 
\thresam relies upon the procedure \reducedmean, given in Alg.~\ref{alg:rm}.
The Bernoulli distribution input to \reducedmean is the distribution 
$\mathcal D_t$, which is defined as follows. 
\begin{definition} \label{def:indicator}
  Conditioned on the current state of the algorithm,
  consider the process where the set $T \sim \mathcal U (A, t - 1)$ and then
  the element $x \sim A \setminus T$ are drawn uniformly at random.
  Let $\mathcal D_t$ denote the probability distribution over the indicator
  random variable 
  $$I_t = \mathbb I [f( S \cup T  + x) - f( S \cup T ) \ge \tau ]. $$
\end{definition}

Below, we state the lemma of \thresam in \shortciteS{Fahrbach2018}.
\begin{lemma}[\citeS{Fahrbach2018}] \label{lemm:thresh} 
  The algorithm \thresam outputs $S \subseteq \mathcal N$ with
  $|S| \le k$ in $O( \log (n / \delta ) / \epsi )$ adaptive
  rounds such that the following properties hold with 
  probability at least $1 - \delta$: 
  \begin{itemize}
    \item[1.] There are $O( n / \epsi )$ oracle queries in expectation.
    \item[2.] The expected average marginal $\ex{f(S)/|S|} \ge (1-\epsi)\tau$.
    \item[3.]  If $|S| < k$, then $f_x(S) < \tau$ for all 
    $x \in \mathcal N$.
  \end{itemize}
\end{lemma}

In \shortciteS{Fahrbach2018a} and \shortciteS{kuhnle2021nearly},
the above Lemma is used with non-monotone submodular functions;
however, in the case that $f$ is non-monotone, the lemma does not hold.
Alg.~\ref{alg:rm} only checks (on Line \ref{line:Check}) if there is 
more than a constant fraction of 
elements whose marginal gains are larger than the threshold $\tau$.
If there exist elements with
large magnitude, \textit{negative} marginal gains, then
the average marginal gain may fail to satisfy
the lower bound in Lemma~\ref{lemm:thresh}.
As for the proof in \shortciteS{Fahrbach2018a},
the following inequality does not hold (needed for the proof of Lemma
3.3 of \shortciteS{Fahrbach2018a}):
\[\ex{\marge{T}{S}}\ge (\ex{I_1}+\ex{I_2} +\ldots +\ex{I_t})\tau,\]
where $|T| = t^*$ and $t\ge t^*/(1+\hat{\epsi})$.
Next, we give a counterexample for the two versions of \thresam used in 
\shortciteS{Fahrbach2018} and \shortciteS{Fahrbach2018a} 
where the only difference is that
the if condition in Alg.~\ref{alg:thresh}
on Line~\ref{line:stopWithA} changes to $|A|<3k$
in \shortciteS{Fahrbach2018a}.
\begin{algorithm}[t] 
  \caption{The \reducedmean algorithm of \shortciteS{Fahrbach2018}} \label{alg:rm}
  \begin{algorithmic}[1]
  \State \textbf{Input:} access to a Bernoulli distribution $\mathcal D$,
 error $\epsi$, failure probability $\delta$
 \State Set number of samples $m \gets 16 \lceil \log (2 / \delta ) / \epsi^2 \rceil$
 \State Sample $X_1, X_2, \ldots, X_m \sim \mathcal D$
 \State Set $\bar \mu \gets \frac{1}{m} \sum_{i = 1}^m X_i$
 \If{ $\bar \mu \le 1 - 1.5 \epsi$ }
 \State \textbf{return} \texttt{true}
 \EndIf
 \State \textbf{return} \texttt{false}
 \end{algorithmic}
 \end{algorithm}
 \begin{algorithm}[t]
  \caption{The threshold sampling algorithm of \shortciteS{Fahrbach2018}}
  \label{alg:thresh}
  \begin{algorithmic}[1]
    \Procedure{\thresam}{$f, k, \tau, \epsi,\delta$}
    \State \textbf{Input:} evaluation oracle $f:2^{\mathcal N} \to \reals$, constraint $k$,
    threshold $\tau$, error $\epsi$, failure probability $\delta$
    \State Set smaller error $\hat{\epsi} \gets \epsi / 3$
    \State Set iteration bounds $r \gets \lceil \log_{(1 - \hat \epsi)^{-1}}(2n / \delta) \rceil, m \gets \lceil \log(k) / \hat \epsi \rceil$
    \State Set smaller failure probability $\hat \delta \gets \delta / (2r(m + 1))$
    \State Initialize $S \gets \emptyset, A \gets N$
    \For{ $r$ sequential rounds }
    \State Filter $A \gets \{ x \in A : \Delta (x, S) \ge \tau \}$\label{line:filter}
    \If{$|A| = 0$}\label{line:stopWithA}
    \State \textbf{break}
    \EndIf
    \For{$i = 0$ to $m$ in parallel}\label{thresh:for}
    \State Set $t \gets \min \{ \lfloor (1 + \hat \epsi)^i \rfloor, |A|\}$
    \State $rm[t] \gets $\reducedmean $( \mathcal D_t, \hat \epsi, \hat \delta )$ \label{line:Check}
    \EndFor
    \State $t' \gets \min t$ such that $rm[t]$ is \texttt{true}
    \State Sample $T \sim \mathcal U \left( A, \min \{t', k - |S| \} \right)$
    \State Update $S \gets S \cup T$
    \If{ $|S| = k$ }
    \State \textbf{break}
    \EndIf
    \EndFor
    \State \textbf{return} $S$
    \EndProcedure
\end{algorithmic}
\end{algorithm}

% \textbf{Counterexample 1.}
% Let $|\mathcal{N}|=15$, and for any $B \subseteq \mathcal{N}$,
% % $f(B) = 5-\left||B|-5 \right|$.
% \begin{equation*}
%   f(B)= \left\{
%     \begin{aligned}
%       &|B|,&|B| \le 12\\
%       &60-4|B|, &|B| \ge 13
%     \end{aligned}
%   \right. .
% \end{equation*}
% Run $\thresam(f, k=15, \tau=1, \epsi=0.3, \delta=0.1)$.
% \begin{proof}
%   For any $B \subseteq \mathcal{N}$ and $x\in \mathcal{N} \backslash B$, 
%   the above set function follows that
%   \begin{equation*}
%     \marge{x}{B}= \left\{
%       \begin{aligned}
%         &1&, |B| \le 11\\
%         &-4 &, |B| \ge 12
%       \end{aligned}
%     \right. ,
%   \end{equation*}
%   which means that $f$ is a non-monotone submodular function.

%   With the input value of $k$, $\tau$, $\epsi$ and $\delta$, 
%   we can get that $r=55,m=28$.
%   There are 55 rounds of the outer \textbf{for} loop,
%   and $t$ is selected from $\{1,2,\ldots, 11,13,14\}$ in the inner for loop.
%   At the first round, after filtering, it holds that $A=\mathcal{N}$.
%   When $t\le 11$, since $\marge{x}{B}=1$ with $|B|\le 11$,
%   random sampling in \textsc{Reduced-Mean} always samples 1,
%   which makes \textsc{Reduced-Mean} returns a \texttt{false}.
%   When $t=13$, since $\marge{x}{B}=-4$ with $|B| =12$,
%   random sampling in \textsc{Reduced-Mean} always samples 0.
%   In this case, it will always return a \texttt{true}.
%   Therefore, $t'=13$ in the first iteration of the outer for loop,
%   and we sample 13 elements from $A$ to $S$.
%   At the second round, all the elements in $A$ will be filtered out,
%   since, for any $x \in \mathcal{N}$, it holds that $\marge{x}{S}=-4 < 0$.
%   \textsc{Threshold-Sampling} algorithm terminates here
%   and returns a set $S$ with 13 elements.

%   In this example, we will always return a set $S$ that $|S|=13$ and $f(S)=8$.
%   But $\ex{f(S)/|S|}=8/13 < (1-\epsi)\tau=0.7$.
% \end{proof}

% \textbf{Counterexample 2.}
% For $m$ submodular functions $\{g_i: 2^{\mathcal{N}_i} \to \reals\} _{i=1}^m$,
% where $|\mathcal{N}_i| = 2$ for any $i$,
% suppose that for any $i$ and $B \subseteq \mathcal{N}_i$,
% $g_i(B) = 1-||B|-1|$.
% Define a submodular function $f: 2^{\mathcal{N}} \to \reals$ with 
% $\mathcal{N} = \bigcup_{i=1}^m \mathcal{N}_i$ as follows,
% for any $B \subseteq \mathcal{N}$, 
% $f(B)=\sum_{i=1}^m g_i(B \cap \mathcal{N}_i)$.
% Run $\thresam(f, k=, \tau=1, \epsi=, \delta=)$.
% % \begin{equation*} 
% %   f(B)= \left\{
% %     \begin{aligned}
% %       &1-||B|-1|&, \text{if } B \subseteq \mathcal{N}_i\\
% %       &\sum_{i=1}^m f(B \cap \mathcal{N}_i)&, \text{otherwise}
% %     \end{aligned}
% %   \right. .
% % \end{equation*}
% \begin{proof}
%   Fisrt, we prove that $f$ is a non-monotone submodular function.
%   Obviously, $g_i$ is a non-monotone submodular function for any $i$,
%   and $f$ is a non-monotone set function.
%   Then, for any $S \subseteq \mathcal{N}$ and $T \subseteq \mathcal{N}$,
%   \begin{align*}
%     f(S) + f(T) &= \sum_{i=1}^m \left( g_i(S \cap \mathcal{N}_i)
%     +g_i(T \cap \mathcal{N}_i)\right)\\
%     &\ge \sum_{i=1}^m \left( g_i((S \cap T)\cap \mathcal{N}_i)
%     +g_i((S\cup T) \cap \mathcal{N}_i)\right)\\
%     &= f(S \cap T) + f(S \cup T).
%   \end{align*}
%   Therefore, $f$ is a non-monotone submodular function.

%   Next, we provide the following lemma.
%   \begin{lemma}
%     Within the first round of \thresam,
%     it holds that,
%     \[\ex{I_t} = \ex{f(S)/|S|} =  \frac{n-t}{n-1},\]
%     where $I_t$ is defined in Definition~\ref{def:indicator},
%     $S \subseteq \mathcal{N}$ and $|S| = t$.
%   \end{lemma}
%   \begin{proof}
%     With Definition~\ref{def:indicator} and $\tau=1$,
%     $I_t = 1$ if and only if $\marge{x}{T}=1$, where $|T|=t-1$.
%     Without loss of generality, we assume that $x \in \mathcal{N}_1$.
%     To have that $\marge{x}{T}=1$, $x$ can be either one in $\mathcal{N}_1$,
%     and $T$ is selected from $\mathcal{N} \backslash \mathcal{N}_1$.
%     Then we can calculate $\ex{I_t}$ as follows,
%     \[\ex{I_t} = \frac{2 \cdot C_{n-2}^{t-1}}{2 \cdot C_{n-1}^{t-1}}=\frac{n-t}{n-1}.\]

%     By the definition of $f$, the objective value of $S$ is the number of 
%     $\mathcal{N}_i$ where $|S \cap \mathcal{N}_i|=1$.
%     Let $j$ be the number of $\mathcal{N}_i$ where $|S \cap \mathcal{N}_i|=2$.
%     Then $f(S) = t-2j$, where $\max\{0, t-m\} \le j \le \lfloor t/2\rfloor$. 
%     The expectation of objective value of $S$ can be calculated by
%     considering the possible combinations 
%     among the $m$ sets where there are $j$ sets such that
%     $|S \cap \mathcal{N}_i|=2$ and there are $t-2j$ sets such that
%     $|S \cap \mathcal{N}_i|=1$.
%     \[\ex{f(S)} =
%     \frac{1}{C_n^t}\cdot \sum_{j=\max\{0, t-m\}}^{\lfloor t/2\rfloor}
%     (t-2j)\cdot C_m^i \cdot C_{m-i}^{t-2i} 2^{t-2i}\overset{(a)}{=}
%     \frac{nC_{n-2}^{t-1}}{C_n^t}=\frac{t(n-t)}{n-1},\]
%     where Equation (a) follows from Lemma~\ref{lemma:comb} in Appendix~\ref{apx:prob}.
%   \end{proof}
% \end{proof}

\textbf{Counterexample 1.}
Define a set function $f: 2^{\mathcal{N}} \to \reals$ as follows,
\begin{equation*}
  f(B)= \left\{
    \begin{aligned}
      &n^2 + |B|,&\text{if } a \not \in B\\
      &n^2 +1- (|B|-1)n, &\text{if }a \in B
    \end{aligned}
  \right. .
\end{equation*}
Let $k = n = |\mathcal{N}|>400$, $\tau=1$, $\epsi=0.1$, $\delta=0.1$.
Run $\thresam(f, k, \tau, \epsi, \delta)$.
\begin{proof}
  For any $B \subseteq \mathcal{N}$ and $x\in \mathcal{N} \backslash B$, 
  the above set function follows that
  \begin{equation*}
    \marge{x}{B}= \left\{
      \begin{aligned}
        &1,&&\text{if } x \neq a \text{ and } a \not \in B\\
        &-n,&&\text{if } x \neq a \text{ and } a \in B\\
        &1-|B|(n+1), &&\text{if } x = a
      \end{aligned}
    \right. .
  \end{equation*}
Thus, $f$ is a non-negative, non-monotone submodular function.
For any $1 < t \le |\mathcal{N}|$, $|T| = t-1$, and $S = \emptyset$,
\[\ex{I_t} = \prob{f( S \cup T  + x) - f( S \cup T ) \ge \tau }
= \prob{x \neq a \text{ and } a \not \in T}
= 1-\frac{t}{n}.\]
So, with any value of $\epsi$, 
\reducedmean returns \texttt{true} when $t> \epsi n/2$.
The first round of \thresam
samples a set $T_1$ with $t_1'=|T_1| > \epsi n/2$.
Then update $S$ by $S = T_1$.

For the \thresam in \shortciteS{Fahrbach2018a} with stop condition
$|A| < 3k$, the algorithm stopped here after the first iteration,
no matter what is sampled.
In this case, the expectation of marginal gains of the set returned
by the algorithm would be as follows,
\begin{align*}
  \ex{\marge{S}{\emptyset}} &= \prob{a \in T_1} \marge{T_1}{\emptyset}
  + \prob{a \not \in T_1} \marge{T_1}{\emptyset}\\ 
  &= \frac{t_1'}{n}\left(1-(t_1'-1)n\right) + \frac{n-t_1'}{n}t_1' \\
  &= t_1'\left(2-t_1'+\frac{1-t_1'}{n}\right) < 0.
\end{align*}

Next, we consider the \thresam with stop condition $|A|=0$.
After the first iteration discussed above,
if $a \in T_1$, all the elements would be filtered out at the second round.
Algorithm stoped here and returned $S$, say $S_1$.
If $a \not \in T_1$, $T_1$ and $a$ would be filtered out at the second round,
which means $A = \mathcal{N} \backslash (S \cup \{a\})$.
And for any $T \subseteq A $ and 
$x \in A \backslash T$, 
\[f(S \cup T + x) - f(S \cup T) = 1.\]
Therefore, $\ex{I_t} = 1$ for all $t$.
After several iterations, $S = \mathcal{N} \backslash \{a\}$ would be returned, say $S_2$.

The expectation of objective value of the set returned would be as follows,
\begin{align*}
  \ex{\marge{S}{\emptyset}} &= \prob{a \in T_1} \marge{S_1}{\emptyset}
  + \prob{a \not \in T_1} \marge{S_2}{\emptyset}\\
  & = \frac{t_1'}{n}(1 - (t_1'-1)n) + \frac{n-t_1'}{n}(n-1)\\
  &= \frac{2t_1'}{n}-1-t_1'^2+n < 0,
\end{align*}
since $\epsi = 0.1$, $n > 400$, and $\epsi n/2 < t_1' < \epsi(1+\epsi/3)n/2$.

% If $a \not \in T$, let $g(B) = f(S \cup B)$,
% where $B \subseteq \mathcal{N} \backslash S$ and $a \not \in S$. Then,
% \begin{equation*}
%   g(B)= \left\{
%     \begin{aligned}
%       &n^2 + |S| + |B|,&\text{if } a \not \in B\\
%       &n^2 +1- (|S| +|B|-1)n, &\text{if }a \in B
%     \end{aligned}
%   \right. ,
% \end{equation*}
%   \begin{equation*}
%   \Delta_g(x|B)= \left\{
%     \begin{aligned}
%       &1,&&\text{if } a \neq x \text{ and } a \not \in B\\
%       &-n,&&\text{if } a \neq x \text{ and } a \in B\\
%       &1-(|S|+|B|)(n+1), &&\text{if } a = x
%     \end{aligned}
%   \right. .
% \end{equation*}
% Similarly to the first round, the second round samples another
% $T$ such that $|T| > 2$, and $\ex{\Delta_g(T|\emptyset)} < 0$.

% Thus, each round of the outer for loop samples a set $T$
% such that $\ex{\marge{T}{S}} < 0$.
\end{proof}

% For completeness, we give a second 
% \textbf{Counterexample 2.}
% Define a set function $f: 2^{\mathcal{N}} \to \reals$ as follows,
% \begin{equation*}
%   f(B)= \left\{
%     \begin{aligned}
%       &\frac{n}{2}(n-1) + |B|,&|B| \le \frac{n}{2}\\
%       &n^2 - n|B|, &|B| > \frac{n}{2}
%     \end{aligned}
%   \right. ,
% \end{equation*}
% where $n=|\mathcal{N}|$.
% Let $\tau=1$, $\epsi=0.1$, $\delta=0.1$, $k = |\mathcal{N}|=2n_0$.
% Run $\thresam(f, k, \tau, \epsi, \delta)$.
% \begin{proof}
%   For any $B \subseteq \mathcal{N}$ and $x\in \mathcal{N} \backslash B$, 
%   the above set function follows that
%   \begin{equation*}
%     \marge{x}{B}= \left\{
%       \begin{aligned}
%         &1,&|B| < \frac{n}{2}\\
%         &- n, &|B| \ge \frac{n}{2}
%       \end{aligned}
%     \right..
%   \end{equation*}
%   Therefore, $f$ is a non-monotone submodular function.
%   % Then we give the following lemma.
%   % \begin{lemma}\label{lemma:red-mean}
%   %   With the above submodular function $f: 2^{\mathcal{N}} \to \reals$, 
%   %   any $S$, $t$, $\epsi$ and $\delta$ as input, 
%   %   \textsc{Reduced-Mean} always returns a \texttt{false} when $|S| + t\le n_0$
%   %   and a \texttt{true} when $|S|+t> n_0$.
%   % \end{lemma}
%   % \begin{proof}

%     Since the objective value of the set only depends on 
%     the size of the set, 
%     for a run of \reducedmean($ \mathcal D_t, \hat \epsi, \hat \delta$ )
%     over current solution set $S$,
%     if $|S| + t\le n/2$, $\marge{x}{S \cup T}=1$ 
%     and 0 is never sampled from $\mathcal{D}$, 
%     which means \reducedmean always return a \texttt{false};
%     if $|S| + t> n/2$, $\marge{x}{S \cup T}=-n$ 
%     and 1 is never sampled from $\mathcal{D}$, 
%     which means \reducedmean always return a \texttt{true}.
%   % \end{proof}

%   Within the first iteration, 
%   no element is filtered out on Line~\ref{line:filter}.
%   Therefore, $A$ = $\mathcal{N}$.
%   Due to the above analysis, it holds that 
%   the first iteration samples $t'$ elements, and $t' > n/2$.
%   At the second iteration, all the elements are filtered out
%   since $\marge{x}{S} = -n$, where $|S| = t'$.
%   Alg.~\ref{alg:thresh} returns a set $S$ with $t'$ elements.
%   The expected average marginal can be bounded as follows,
%   \begin{align*}
%     \ex{\marge{S}{\emptyset}/|S|} = \left(n^2-nt'-\frac{n}{2}(n-1)\right)/t'
%     = n\left(\frac{n+1}{2}-t'\right)/t' < 0.
%   \end{align*}

  
%   % With the input value of $k$, $\epsi$ and $\delta$,
%   % let $t_{\text{max}}$ be the maximum $t$ within the inner for loop.
%   % It holds that,
%   % \[t_{\text{max}} \le \lfloor(1+\hat{\epsi})^{m} \rfloor
%   % < \left((1+\hat{\epsi})^{1/\hat{\epsi}}\right)^{\log(k) +\hat{\epsi}}
%   % < e^{\log(n) +\hat{\epsi}}<n.\]
%   % Then we consider the following two cases.

%   % First, if $\lfloor(1+\hat{\epsi})^{m} \rfloor > n-3$, 
%   % by Lemma~\ref{lemma:red-mean},
%   % the first iteration of the outer for loop
%   % returns a $t'$ such that $t'\ge n-2$. 
%   % Then we sample a set $T$ with $t'$ elements.
%   % In the second iteration, the whole ground set will be filtered out.
%   % The algorithm stops here and returns set $S=T$.
%   % It holds that,
%   % \[\ex{f(S)/|S|} = \frac{(n-3)(n-t')}{3t'} \le \frac{2(n-3)}{3(n-2)}
%   % < (1-\epsi)\tau=0.7.\]
  
%   % Next, we give an instance. 
%   % Let $n=15$. With $k=15$, $\tau=1$, $\epsi=0.3$ and $\delta=0.1$,
%   % there are 55 iterations of the outer for loop,
%   % and $t$ is selected from $\{1,2,\ldots, 11,13,14\}$ in the inner for loop.
%   % By Lemma~\ref{lemma:red-mean}, $rm[t]=\texttt{false}$ when $t\le 11$;
%   % $rm[t]=\texttt{true}$ when $t=13,14$.
%   % The first iteration of the outer for loop returns $t'=13$
%   % and a set $T$ with 13 elements is sampled from the ground set $\mathcal{N}$.
%   % In the second iteration, the whole ground set is filtered out 
%   % and the algorithm stops here.
%   % Therefore, with probability 1, 
%   % \thresam returns a set $S$ such that $|S|=13$.
%   % But $\ex{f(S)/|S|}=8/13 < (1-\epsi)\tau=0.7$.

%   % Second, if $\lfloor(1+\hat{\epsi})^{m} \rfloor < n-3$ 
%   % and $(n-3)\%\lfloor(1+\hat{\epsi})^{m} \rfloor \neq 0$,
%   % by Lemma~\ref{lemma:red-mean}, at the first several iterations,
%   % a set $T$ with $|T|=t_{\text{max}}$ is returned,
%   % until that $|S|+t_{\text{max}} \ge n-3$.
%   % Then at the iteration that $|S|+t_{\text{max}} > n-3$,
%   % a set $|T|$ with $t'\ge n-2-|S|$ elements is sampled.
%   % And the whole ground set is filtered out at the next iteration.
%   % Therefore, a set $S$ with $|S| \ge n-2$ is returned with probability 1.
%   % It holds that,
%   % \[\ex{f(S)/|S|} = \frac{(n-3)(n-|S|)}{3|S|} \le \frac{2(n-3)}{3(n-2)}
%   % < (1-\epsi)\tau=0.7.\]

%   % We also give an instance here.
%   % Let $n=60$. With $k=60$, $\tau=1$, $\epsi=0.3$ and $\delta=0.1$,
%   % there are 68 iterations of the outer for loop,
%   % and $t_{\text{max}} = 49$.
%   % Within the first iteration,
%   % by Lemma~\ref{lemma:red-mean}, $rm[t]=\texttt{false}$ for all $t$.
%   % Then a set $T$ with 49 elements is sampled from $\mathcal{N}$.
%   % Within the second iteration, 
%   % $A = \mathcal{N} \backslash T$,
%   % and $t$ is selected from 1 to 11.
%   % By Lemma~\ref{lemma:red-mean}, $\min\{t: rm[t]=\texttt{true}\}=9$.
%   % Then a set $T$ with 9 elements is sampled from $A$.
%   % Within the third iteration, for any $x \in A$,
%   % $\marge{x}{A} < \tau$.
%   % The algorithm stops at the third iteration,
%   % and returns a set $S$ with 58 elements with probability 1.
%   % But $\ex{f(S)/|S|}=38/58 < (1-\epsi)\tau=0.7$.
% \end{proof}

%%% Local Variables:
%%% mode: latex
%%% TeX-master: "main"
%%% End: