\documentclass{article}


% if you need to pass options to natbib, use, e.g.:
%     \PassOptionsToPackage{numbers, compress}{natbib}
% before loading neurips_2022


% ready for submission
% \usepackage{neurips_2022}


% to compile a preprint version, e.g., for submission to arXiv, add add the
% [preprint] option:
%     \usepackage[preprint]{neurips_2022}


% to compile a camera-ready version, add the [final] option, e.g.:
    \usepackage[final]{neurips_2022}


% to avoid loading the natbib package, add option nonatbib:
%    \usepackage[nonatbib]{neurips_2022}


\usepackage[utf8]{inputenc} % allow utf-8 input
\usepackage[T1]{fontenc}    % use 8-bit T1 fonts
\usepackage{hyperref}       % hyperlinks
\usepackage{url}            % simple URL typesetting
\usepackage{booktabs}       % professional-quality tables
\usepackage{amsfonts}       % blackboard math symbols
\usepackage{nicefrac}       % compact symbols for 1/2, etc.
\usepackage{microtype}      % microtypography
\usepackage{xcolor}         % colors


% -- 
\usepackage{enumitem}
% --- for professional tables
\usepackage{booktabs}  
% --- for figures
\usepackage{graphicx}
\usepackage{caption}
\usepackage{subcaption}
\usepackage{wrapfig}
% --- for formula
\usepackage{amssymb}
\usepackage{amsmath}
\DeclareMathOperator*{\argmax}{arg\,max}
\DeclareMathOperator*{\argmin}{arg\,min}
% --- for table and figure side by side
% \usepackage{floatrow}
\usepackage{makecell}
% Table float box with bottom caption, box width adjusted to content
% \newfloatcommand{capbtabbox}{table}[][\FBwidth]
% ----

\usepackage{multirow}
\usepackage{placeins}
\usepackage[normalem]{ulem}

% --- theorems
\usepackage{bbm}
\usepackage{amsthm}

\newtheorem{definition}{Definition}[section]
\newtheorem{example}{Example}[section]
\newtheorem{theorem}{Theorem}[section]
% \newtheorem{corollary}{Corollary}[theorem]
\newtheorem{lemma}[theorem]{Lemma}
\newtheorem{corollary}{Corollary}[section]
\newtheorem{proposition}{Proposition}[section]
% \newtheorem{remark}{Remark}[theorem]
% \newtheorem*{remark}{Remark}
\newtheorem{remark}{Remark}[section]
\newtheorem{assumption}{Assumption}[section]

% --- commands
\newcommand{\chengyu}[1]{{\color{blue}{\emph{CD: #1}}}}
% \newcommand{\chengyu}[1]{}
\newcommand{\todo}[1]{{\color{blue}{\textbf{TODO: #1}}}}
\newcommand{\jingbo}[1]{{\color{blue}{\textbf{Jingbo:} #1}}}
\newcommand{\lucas}[1]{{\color{red}{\textbf{lucas:} #1}}}
\newcommand{\lucasedit}[1]{{\color{gray}{#1}}}
\newcommand{\smallsection}[1]{\textbf{#1.~~~~}}
\newcommand{\note}[1]{{\color{red}{#1}}}

\newcommand{\noise}[1]{implicit label noise\xspace}
\newcommand{\Noise}[1]{Implicit label noise\xspace}

\newcommand{\specialcell}[2][c]{%
  \begin{tabular}[#1]{@{}c@{}}#2\end{tabular}}
  
  
% \title{Double Descent in Adversarial Training: An Implicit Label Noise Perspective}

% \title{Understanding Implicit Label Noise in Adversarial Training for Robust Deep Learning}
% \title{Label Noise Exists Implicitly in Adversarial Training}
% \title{Robust Overfitting May be Explained By the Label Noise in Adversarial Training}
% \title{Label Noise in Adversarial Training: A Novel Perspective to Explain and Alleviate Robust Overfitting}
\title{Label Noise in Adversarial Training: A Novel Perspective to Study Robust Overfitting}
% \title{Chart and harness the label noise in adversarial training}
% title based on 
% `Feature Purification: How Adversarial Training Performs Robust Deep Learning` Allen zhu.
% `UNDERSTANDING INTRINSIC ROBUSTNESS USING LABEL UNCERTAINTY`



% The \author macro works with any number of authors. There are two commands
% used to separate the names and addresses of multiple authors: \And and \AND.
%
% Using \And between authors leaves it to LaTeX to determine where to break the
% lines. Using \AND forces a line break at that point. So, if LaTeX puts 3 of 4
% authors names on the first line, and the last on the second line, try using
% \AND instead of \And before the third author name.


\author{%
Chengyu Dong \\
University of California, San Diego\\
\texttt{cdong@eng.ucsd.edu}\\
\And
Liyuan Liu\\
Microsoft Research\\
\texttt{lucliu@microsoft.com}\\
\And
Jingbo Shang\\
University of California, San Diego\\
\texttt{jshang@eng.ucsd.edu}\\
}


\begin{document}


\maketitle


\begin{abstract}

% \chengyu{Label noise exists -> explains overfitting -> double descent as a verification -> method to mitigate label noise}


% In adversarial training, inheriting labels for adversarial examples from their clean counterparts has been a common practice for years.
% In this paper, we argue that this common practice in fact leads to a mismatch between true label distribution and assigned label distribution, and therefore, introduces label noise to adversarial training implicitly. 
We show that label noise exists in adversarial training. 
Such label noise is due to the mismatch between the true label distribution of adversarial examples and the label inherited from clean examples 
% \jingbo{assigned label is not defined. I prefer the common practice way to describe this.} 
 -- the true label distribution is distorted by the adversarial perturbation, but is neglected by the common practice that inherits labels from clean examples. 
Recognizing label noise sheds insights on the prevalence of robust overfitting in adversarial training, and explains its intriguing dependence on perturbation radius and data quality. 
Also, our label noise perspective aligns well with our observations of the epoch-wise double descent in adversarial training. 
Guided by our analyses, we proposed a method to automatically calibrate the label to address the label noise and robust overfitting. 
Our method achieves consistent performance improvements across various models and datasets without introducing  new hyper-parameters or additional tuning.


% Here, we study the label noise issue for adversarial training. 
% We show that after adding adversarial perturbations to the original input, it is inevitable to shift the underlying label distribution.
% Correspondingly, it \jingbo{what is this ``it''? please try to make it more clear.} introduces label noise by inheriting labels from their clean counterparts for adversarial examples. 
% To examine the impact of label noise on adversarial training, we conduct empirical analyses on a label noise byproduct ( i.e., epoch-wise double descent) and observe a clear correlation between the level \jingbo{``the level'' reads a bit odd to me. degree? significance? idk.. and I think this part is not mentioned in intro now.} of epoch-wise double descent and the level of label noise. 
% Our observations verify our intuitions \jingbo{what are our intuitions? I would prefer to say it explicitly ``label noise exists in adversarial training''} and also shed insights on the the prevalence of robust overfitting in adversarial training. 
% Guided by our analyses, we proposed a method to automatically calibrate the label supervision to address the label noise and robust overfitting. 
% Our method achieves consistent performance improvements across various models and datasets without introducing  new hyper-parameters or additional tuning.


% Here, we study the a long-overlooked issue for adversarial training, i.e., implicit label noise. 
% We show that, adding adversarial perturbations to the original input image would cause the output  
% It is known that overfitting is more prominent in  adversarially robust deep learning than standard learning.
% In this paper, we extend the classic notion of label noise (label flipped for some instances) to implicit label noise (mismatch between true label distribution and assigned label distribution). 
% Under the perspective of implicit label noise, the prevalence of overfitting in adversarial training is properly aligned with the effect of label noise in standard learning in promoting variance. \jingbo{the last two ``in''s make me feel odd here.}
% We confirm that robust overfitting\jingbo{robust fitting is not mentioned before. We'd better to stick with the same term here.} can be viewed as a special case of epoch-wise double descent. 
% To mitigate implicit label noise in adversarial training, We show that model probability can approximate the true label distribution, in line with the existing practice in tackling robust overfitting. Finally, we show that it is possible to further boost the existing practice with temperature scaling and interpolation and validate its effectiveness by extensive experiments on benchmark datasets.



% In this paper, 
% we present the double descent phenomena in adversarial training and propose a novel perspective to understand it. 
% Specifically, 
% In this paper, we analyze robust overfitting from an epoch-wise double descent, i.e., we observe that the robust test error will start to decrease again after training the model for a considerable number of epochs.
% Inspired by our observations, 
% % we further tailored existing analyses and theory to better understand robust overfitting and the epoch-wise double descent. 
% we further advance the analyses of double descent to understand robust overfitting better. 
% \sout{In standard training, double descent has been shown to be a result of label flipping noise.}
% However, this reasoning is not applicable in our setting, since adversarial perturbations are believed not to change the label. 
% Going beyond label flipping noise, we propose to measure the mismatch between the assigned and (unknown) true label distributions, denoted as \emph{implicit label noise}.
% We show that the traditional labeling of adversarial examples inherited from their clean counterparts will lead to implicit label noise.
% % , but not label flipping noise.
% Towards better labeling, we show that predicted distribution from a classifier, after scaling and interpolation, can provably reduce the implicit label noise under mild assumptions.
% In light of our analyses, we tailored the training objective accordingly to effectively mitigate the double descent and verified its effectiveness on three benchmark datasets.



\end{abstract}


version https://git-lfs.github.com/spec/v1
oid sha256:df45afd8816c5b0d84c321bf454ca36621f6ffd7b2325931c7881fd21f0be683
size 24200

\section{Related Work}
\label{sect:related}


% \smallsection{Label noise}
% % \todo{Then discuss to explain such double descent we need to think about where the label noise comes from. It is not simply label flipping noise by wrongly labeling, but it could be implicit label noise. So merge the discussion of label noise to here.}
% Label noise is a long-standing problem in machine learning and has attracted systematic studies~\citep{Frnay2014ClassificationIT, Song2020LearningFN}. 
% In classification problems, label noise is typically defined in a conditional manner. Specifically, the true label of an example is altered by an unknown noise process with some probability before being observed by an algorithm~\citep{Angluin2005LearningFN}, which we refer as the label flipping noise that exists in the dataset as a ratio of incorrect labels. 
% However, it is also possible to directly consider the probability that the true label and the assigned label disagree, where a distribution mismatch can make the difference.
% % \jingbo{Shall we discuss this label noise separately? Maybe merge it with the standard training part is a better idea?}
% % \chengyu{Move the 'conventional label noise' part here.}
% \chengyu{Conventionally confusion matrix, meaning the noise process is defined. But now we only have the final distribution.}

% % To understand the double descent in adversarial training within the existing picture of double descent, 
% We find that label noise can be implicitly incurred by the mismatch between the assigned label distribution and true label distribution of the adversarial example. Such label noise can be interpreted as an instance-dependent, class dependent label noise in reference to the systematic studies on the taxonomy of label noise~\citep{Frnay2014ClassificationIT, Song2020LearningFN}. %  thus can be the origin of double descent in adversarial training.


\smallsection{Robust overfitting and double descent in adversarial training}
% Robust overfitting has been widely observed in adversarial training practice. 
Double descent refers to the phenomenon that overfitting by increasing model complexity will eventually improve test set performance~\citep{Neyshabur2017ExploringGI, Belkin2019ReconcilingMM}.
% \jingbo{this reads like mode-wise double descent. You may want to focus on the epoch-wise phenomenon. Otherwise, the readers will get confused. }
This appears to conflict with the robust overfitting phenomenon in adversarial training, where increasing model complexity by training longer will impair test set performance constantly after a certain point during training. It is thus believed in the literature that robust overfitting and epoch-wise double descent are separate phenomena~\citep{Rice2020OverfittingIA}. In this work we show this is not the complete picture by conducting adversarial training for exponentially more epochs than the typical practice.

% Towards a more complete understanding of robust overfitting, in this work, we conduct adversarial training for exponentially more epochs than the typical practice. We find that robust overfitting shall be viewed as the early part of an epoch-wise double descent. And increasing the model architecture size, another way to increase the model complexity, 
% %\jingbo{I'm not sure if training longer time means more model complexity. I assume the model complexity is almost the same as the architecture size? },
% can modulate the epoch-wise double descent curve such that either the overfitting curve is shown, or the entire double descent curve is revealed within the same number of training epochs as shown in Figure~\ref{fig:intro}. 
% % Therefore, Robust overfitting should be unified with double descent and 
% % Our analyses thus manage to unifying the two seemingly separately phenomena. % robust overfitting and double descent.
% Therefore, robust overfitting will not go beyond modern generalization theory as an exception and should be adequately explained by the origin of double descent such as label noise.


% % Previous works suggest that robust overfitting and double descent are separate phenomena by observing that training longer, while increasing the model complexity, results in worse test performance constantly after a certain point during training~\citep{Rice2020OverfittingIA}. 
% \chengyu{Mention more about why capacity is on Figure~\ref{fig:intro}}
% \jingbo{I think the ``by observing ...'' part is more like defining the robust overfitting but not saying that these two are separate phenomena. It's a little confusing and also verbose as we have just mentioned the same thing in the intro. I don't think we need to define is again here. }
% This appears to conflict the common belief that increasing model complexity benefits the test performance in modern generalization curves~\citep{Belkin2019ReconcilingMM}.
% \jingbo{We haven't formally defined what's double descent in the intro. Maybe it's a time to define it here? I think we want to make it clear that in standard training, overfitting has been shown as an early part of double descent, especially when the models are complex enough. This can draw the connection between the robust overfitting and double descent in adversarial training. }
% To tackle this dilemma, our work provides a more complete understanding of robust overfitting by showing that it is an early part of the epoch-wise double descent. 
% Therefore robust overfitting will not go beyond modern generalization theory as an exception and should be adequately explained by the origin of double descent such as label noise.


% \chengyu{A double descent in adversarial training is hard to observe because the training epochs is not sufficient. High model capacity may reduce the required training epochs, but also slow to train.}

% \note{Effect of lr scheduler. Our setting shows that the robust overfitting doesn't hinge on the learning rate decay, which is a commonly accepted understanding / hypothesis. }

 
% \note{Seems to remember there is one work on robust overfitting theoretically with optimization, maybe include it here.}

A recent work also considers a different notion of double descent that is defined with respect to the perturbation size~\citep{Yu2021UnderstandingGI}. 
Such double descent might be more related to the robustness-accuracy trade-off problem~\citep{Papernot2016TowardsTS, Su2018IsRT, Tsipras2019RobustnessMB, Zhang2019TheoreticallyPT}, rather than the classic understanding of double descent based on model complexity.


% % \jingbo{``origin'' is vague here. You may want to use ``Explanations'' or ``Root causes''. }
% \smallsection{Understand double descent in adversarial training}
% % \smallsection{Double descent in standard and adversarial training}
% In standard training, double descent is often attributed to increased variance, with label noise being a common source. Definitions of label noise vary in literature. Theoretically-grounded analyses of double descent focus on additive label noise~\citep{Advani2020HighdimensionalDO, Mei2019TheGE, Hastie2019SurprisesIH, Belkin2020TwoMO, dAscoli2020DoubleTI, Jacot2020ImplicitRO}, but only applicable to regression problems. Theoretical results on double descent are scarce on classification problems, with a few works introducing noise by randomly masking the feature vector~\citep{Deng2019AMO, Kini2020AnalyticSO}. Analyses of double descent on classification problems are more common in empirical studies, where a typical way to induce double descent is to inject label flipping noise, namely the labels of a random fraction of training examples are flipped to other labels~\citep{Nakkiran2020DeepDD, Yang2020RethinkingBT}. However, such a definition of label noise cannot properly fit the scenario in adversarial training, where labels are not likely to be flipped due to small adversarial perturbation.






% \jingbo{Make it clear that the first paragraph here is about standard training.}
% Double descent describes the generalization properties of deep neural networks and is ubiquitous in modern learning problems. 
% Towards a theoretical analysis of double descent, much effort has been focused on regression problems where additive label noise is often introduced to promote variance~\citep{Advani2020HighdimensionalDO, Mei2019TheGE, Hastie2019SurprisesIH, Belkin2020TwoMO, dAscoli2020DoubleTI}. 
% Few works focusing on classification problems instead introduce variance by randomly masking the feature vector~\citep{Deng2019AMO, Kini2020AnalyticSO}.
% \jingbo{How about label flipping noise? Maybe discuss it and echo again that we extend it to implicit label noise to explain the double descent in adversarial training.}

% \jingbo{Maybe swap the order of these two paragraphs. Adversarial training first and the standard training.}
% In adversarial training, model-wise double descent have been observed empirically~\citep{Nakkiran2020DeepDD, Rice2020OverfittingIA}, whereas epoch-wise double descent is believed to be not applicable~\citep{Rice2020OverfittingIA}. 
% A recent work also considers a different notion of double descent that is defined with respect to the perturbation radius~\citep{Yu2021UnderstandingGI}. 
% Such double descent might be more related to the robustness-accuracy trade-off problem~\citep{Papernot2016TowardsTS, Su2018IsRT, Tsipras2019RobustnessMB, Zhang2019TheoreticallyPT}.
% \jingbo{Since our conclusion seems conflicting with \citep{Rice2020OverfittingIA}, better to discuss a bit here why we finally observed this. Probably because of the exponential number of epoches? It's indeed harder to observe than those in standard training.}


\smallsection{Mitigate robust overfitting}
Robust overfitting hinders the practical deployment of adversarial training methods as the final performance is often sub-optimal. Various regularization methods including classic approaches such as $\ell_1$ and $\ell_2$ regularization and modern approaches such as cutout~\citep{Devries2017ImprovedRO} and mixup~\citep{Zhang2018mixupBE} have been attempted to tackle robust overfitting, whereas they are shown to perform no better than simply early stopping the training on a validation set~\citep{Rice2020OverfittingIA}. However, early stopping raises additional concern as the best checkpoint of the robust test accuracy and that of the standard accuracy often do not coincide~\citep{chen2021robust}, thus inevitably sacrificing the performance on either criterion. Various regularization methods specifically designed for adversarial training are thus proposed to outperform early stopping, including regularization the flatness of the weight loss landscape~\citep{Wu2020AdversarialWP, Stutz2021RelatingAR}, introducing low-curvature activation functions~\citep{Singla2021LowCA}, data-driven augmentations that adds high-quality additional data into the training~\citep{Rebuffi2021FixingDA} and adopting stochastic weight averaging~\citep{Izmailov2018AveragingWL} and knowledge distillation~\citep{Hinton2015DistillingTK}~\citep{chen2021robust}. These methods are likely to suppress the label noise in adversarial training, with the self-distillation framework (i.e. the teacher shares the same architecture as the student model) introduced by \citep{chen2021robust} as a particular example since introducing teacher's outputs as supervision is almost equivalent to the alternative labeling inspired by our understanding of the origin of label noise in adversarial training.
% \chengyu{By additional data}
% be consistent with our data-centric understanding as various regularization techniques may suppress the learning of low-quality examples.

% a recent proposed strategy~\citep{chen2021robust} has advanced the practice by incorporating knowledge distillation~\citep{Hinton2015DistillingTK} and stochastic weight averaging~\citep{Izmailov2018AveragingWL} into adversarial training. 

% \smallsection{Confidence calibration}
version https://git-lfs.github.com/spec/v1
oid sha256:7e1d914ab759c5b6dbfbf25ab730016ecf474362d6da8410b5525ba0d9b620c1
size 13827

% \section{Double Descent from an Implicit Label Noise Perspective}
% \section{Adversarial perturbation can cause label noise implicitly}
% \section{Explore Label Noise in Adversarial Training}
% \label{sect:reason}

% \chengyu{Make sure we show the point that adversarial training only \emph{magnifies} the label noise and thus makes the robust overfitting more evident.}

% \chengyu{Avoid mention double descent to the best}

% \chengyu{maybe also no need to define ``implicit label noise''. Then people will not require us to demonstrate such a new type of label noise in the reality.}

% In this section, we present a novel perspective to understand the double descent in adversarial training. The implicit label noise is originated from the improper labeling of the adversarial examples and can induce double descent in adversarial training.

% In this section, we first show that traditional adversarial perturbation does not cause label noise directly. We then argue that adversarial perturbation cause label noise implicitly, but significantly.

% To understand intriguing behaviors of adversarial training such as robust overfitting we focus on its training set. We first show that label noise does not explicitly exist in the adversarially augmented training set. We then argue that label noise will implicitly exist in the adversarially augmented training set due to the distribution mismatch and improper label construction in the common practice.


 




% version https://git-lfs.github.com/spec/v1
oid sha256:5abff5344bbf79a2c8780edff8201c0be72fea4edcd76fdc2c4fc9345f72d151
size 5002


version https://git-lfs.github.com/spec/v1
oid sha256:9956ecf8f72f8cc864de3724da585c24c2cb6e97975537af55723da00e24101f
size 19514


% \subsection{Equivalence between implicit label noise and label flipping noise}
% \subsection{Influence of implicit label noise in adversarial training}
% \subsection{Influence of implicit label noise}
% \subsection{Implicit label noise is a specific type of label noise}
% \smallsection{Connection and difference between implicit label noise and conventional label noise}
\smallsection{Intuitive interpretation of label noise in adversarial training}
% \chengyu{If no section 4.1, can also remove this section}
% In this section, we connect the implicit label noise to a more familiar definition of label noise and show it can have a significant impact.
%     \begin{proposition}[Implicit label noise is equivalent to instance-dependent and class-dependent label noise]
%     \label{proposition:label-noise}
%     Let $p_e(j, x) = P(\tilde{Y}\ne j | Y = j, x)$ be a typical definition of label noise which depends on both the class $j$ and input $x$. Then implicit label noise is equivalent to
%     \begin{equation}
%       % p(Y\ne Y^* | x) = 1 - \sum_j p(Y= j | Y^* = j, x) p(Y^* = j|x),
%       P(\tilde{Y}\ne Y | x) = 1 - \sum_j (1 - p_e(j, x)) P(Y = j|x),
%     \end{equation}
%   % It can easily seen that if $p(Y\ne Y^* | x) > 0$, $p(Y\ne j | Y^* = j, x) > 0$ for some $j$.
%   \end{proposition}
%   \begin{proof}
%   See Appendix~\ref{sect:label-noise-more-proof}.
%   \end{proof}
We introduce a simple example to help understand the emergence of label noise in adversarial training.
% Towards an intuitive understanding of implicit label noise, we introduce a simple example to discuss the differences and connections between implicit label noise and convention label noise.
% We now try to quantify the implicit label noise given its connection with typical label noise.
% \chengyu{We now discuss a simplified example of implicit label noise, and discuss its connection and difference with conventional label noise.}
\begin{example}
% [Quantify the implicit label noise]
[Label noise due to a symmetric distribution shift]
\label{example:label-noise-influence}
% We would like to note that the adversarial training setting would amplify the impact of the implicit label noise, since it adds perturbations to every training sample. 
% In fact, Theorem 3.1 in our main paper, which lower-bounds the implicit label noise by the distance between the assigned label distribution and the true label distribution, provides a way to intuitively quantify the implicit label noise. 
Let $\mathcal{D}=\{(x_i, y_i)\}_{i\in[N]}$ be a clean labeled training subset where all inputs $x_i=x$ are identical and have a one-hot true label distribution, i.e., $P(Y|x) = \mathbf{1}_y$. %, and there is no label noise in $\mathcal{D}$, i.e. $y = \tilde{y}$. 

We now construct an adversarially augmented training subset $\mathcal{D'} = \{(x'_i, \tilde{y}'_i)\}_{i\in[N]}$, where $\tilde{y}' = y$ and $x'$ is generated based on adversarial perturbation that distorts the true label distribution symmetrically. Specifically,
$$
P(Y'= j' | x') =
\begin{cases} 
1 - \eta, & \text{if}~~j = y, \\
\eta/(K-1), & \text{otherwise}. \\
\end{cases}
$$
Then by Lemma~\ref{theorem:implicit-label-noise} 
% we have $P(\tilde{Y}' \ne Y' | x') \ge \eta$.
% we have $E_{j'} P_e(j', x') \ge \eta$.
we have $p_e (\mathcal{D}')\gtrsim \eta$.
% \ge % \left\| P(Y^*|x) -  P(Y^*_\delta|x_\delta).\right\|_{\text{TV}}
% which is equivalent to $p_e(j,x) = \sigma$ by Proposition~\ref{proposition:label-noise}. % , meaning $10\%$ label noise is injected in $\mathcal{S}$.
% which means the label noise injected in $\mathcal{S}_\delta$ is at least $10\%$  by Proposition~\ref{proposition:label-noise}.
% the total variation distance between these distributions is 0.1, which means the implicit label noise is at least 0.1. This is already equivalent to 10\% label noise based on the connection between implicit label noise and the (instance-wise) probabilistic definition of label flipping noise (see Remark 3.2 in our paper).
% \jingbo{what is this 10? I didn't quite get this.}
\end{example}
   
% \chengyu{Talk about observation distribution, noisy process and true distribution?}

One can find that there is indeed $\eta$ faction of noisy labels in $D'$. This is because if we sample the labels of $x'$ based on its true label distribution, we expect $1 - \eta$ faction of $x'$ are labeled as $y$, while $\eta$ fraction of $x'$ are labeled to be other classes. However, in $D'$, all $x'$ are assigned with label $y$ , which means $\eta$ fraction of $x'$ are labeled incorrectly. In realistic datasets we can consider inputs with similar features for such reasoning.

% \chengyu{Add a interpretation from population view?? I remember there is a case about dogs and cats in previous revision.}
% Conventionally speaking, label noise is perceived as the fraction of the noisy labels in the training set, i.e. the assigned labels that are different from their corresponding true labels. However, in the adversarially augmented training set no assigned label is noisy since $\tilde{y}' = y'$ 
% ~\footnote{Recall $\tilde{y}' = \argmax_j P(\tilde{Y'}=j|x')$ and $y' = \argmax_j P(Y'=j|x')$}
% for every augmented input $x'$. Yet, rather counter-intuitively, at least $\eta$ label noise exists in $\mathcal{D}'$, which is due to the fact that every input is now more likely to be mislabeled after adversarial perturbation.


% \chengyu{Difference from a process view. There is no noisy process defined, only the final assigned distribution after noisy process is known. But as long as the final distribution is different, suggests the annnotation must go through some unknown noisy process.}
% \chengyu{Why same argmax doesn't mean there is no label noise? Because label noise is always associated with a noisy random process. Because the change of the underlying true distribution should be reflected in the sampled labels, otherwise there must be some noisy process an annotator goes through.}

The above example also shows that label noise in adversarial training may be stronger than one's impression. Even a slight distortion of the true label distribution, e.g. $\eta=0.1$, will be equivalent to at least $10\%$ noisy label in the training set. This is because the true label distribution of every training input is distorted, resulting in significant noise in the population. 
% \sout{Such example also implies that even static adversarial perturbation~\footnote{namely the adversarial perturbation is added to the training set only once and the standard training is performed subsequently} can produce clear double descent as shown in Appendix~\ref{sect:exp-static}.} Therefore we believe implicit label noise can be an important source of label noise that makes double descent more evident in adversarial training.



    % An informal proof can be sketched from a frequentist's view and help the understanding of implicit label noise.
    % % One can interpret the implicit label noise in a frequentist's view.
    % Say there are $M$ identical copies of $x_\delta$ in the training set $\mathcal{D}_\delta$, with their true labels and traditional adversarial labels distributing according to $p(y^*_\delta | x_\delta)$ and $p(\tilde{y}_\delta| x_\delta)$, respectively.
    % % by Remark~\ref{remark:common-practice} and Assumption~\ref{assumption:clean-dataset}, respectively.
    % % The true label of $x_\delta$ is sampled based on $p(y_\delta |x + \delta)$, the assigned label is sampled based on $p(y|x)$. 
    % The number of copies that have the same true label and assigned label is $ M \sum_j \min \{p(\tilde{Y}_\delta=j|x_\delta), p(Y^*_\delta=j |x_\delta)\}$.
    % The fraction of label noise exists in $\mathcal{D}_\delta$ is thus $1 - \sum_j \min \{p(\tilde{Y}_\delta=j|x_\delta), p(Y^*_\delta=j|x_\delta)\} = \|p(\tilde{y}_\delta|x_\delta) - p(y^*_\delta|x_\delta) \|_{\text{TV}}$  by the definition of the total variation distance.
    


% ----------------------------------------------    
% ----------------------------------------------
\smallsection{Dependence of label noise in adversarial training}
\label{sect:dependence-label-noise}
    % We now show the implicit label noise in adversarial training depends on the perturbation radius and the data quality given mild assumptions on the probabilistic classifier.
    % and show how these factors affect double descent in adversarial training.
    % \begin{corollary}[Dependence of implicit label noise]
    % % Assume the true label distribution $P(Y^*|x)$ is locally convex around $x$ and can be asymptotically described as
    % % \chengyu{Twice differentiable is also enough. Use Taylor approximate in first order to express gradient with perturbed distance. But here need to show it is negative correlation, meaning norm after hessian multiplication is negative correlated}
    % Assume $f(x)_y$ is $L$-locally Lipschitz around $x$ with Hessian bounded below. Let $m = \inf_{z \in \mathcal{B}_\varepsilon(x)} \sigma_{\min} (\nabla^2 f(z)_y) > 0$, we have
    % \begin{equation}
    % \label{theo:label-noise-dependence}
    %     % \|p_Y - p_{Y'}\|_{\text{TV}} 
    %     P(\tilde{Y’}\ne Y' | x')
    %     \ge
    %     \frac{\varepsilon}{2} (1 - q(x)) \frac{m}{L}  - \frac{\varepsilon^2}{4} M,
    % \end{equation}
    % where $q(x)$ is the data quality.
    % % \begin{equation}
    % % \label{eq:label-noise-assumption}
    % %     %  \|\nabla_x~ p(y^*=j|x)\| \propto 
    % %     % \begin{cases}
    % %     % 1 -  p(y^*=j|x),& \quad  p(y^*=j|x) \to 1\\
    % %     %      p(y^*=j|x),& \quad  p(y^*=j|x) \to 0,
    % %     % \end{cases}   
    % %      \|\nabla_x~ P(Y^*=j|x)\| \propto 1 -  P(Y^*=j|x),
    % % \end{equation}
    % % as $P(Y^*=j|x) \to 1$.
    % % % $\max_j p(y^*=j|x) \approx 1$, % Given in assumption
    % % % where $j^* = \argmax~p(y=j|x)$, given by above lemma 2.2
    % % We have
    % % $$
    % % \underline{\min}~P(\tilde{Y}_\delta \ne Y^*_\delta | x_\delta) \propto \varepsilon (1 - q(x)),
    % % $$
    % % where $\underline{\min}$ means the lower bound of the minimum label noise, and $q(x)$ is the data quality (\ref{definition:data-quality}).
    % \end{corollary}
    % \begin{proof}
    % % See Appendix~\ref{sect:label-noise-more-proof}, where we also show that Assumption (\ref{eq:label-noise-assumption}) holds true for a Gaussian mixture model.
    % Let $f(x) = f(x)_y$. Assume $f$ is locally Lipschitz around $x$. Let $x^* = \argmin_{z \in X, f(z) = 1} \|x - z\|$ (local maximum closest to $x$). Because $x^*$ is the local maximum and $f$ is continuously differentiable, $\nabla f(x^*) = 0$, thus
    % $$
    % \nabla f(x) 
    % & = \nabla f(x^*) + \nabla^2 f(z) (x - x^*) = \nabla^2 f(z) (x - x^*).
    % $$
    % Therefore we have
    % $$
    % \begin{aligned}
    % \|\nabla f(x)\| 
    % & =  \|\nabla^2 f(z) (x - x^*) \| \\
    % & \ge \sigma_{min} (\nabla^2 f(z)) \|x - x^*\| \\
    % & \ge \sigma_{min} (\nabla^2 f(z)) \frac{|f(x^*) - f(x)|}{L(f)} \\
    % & = \frac{\sigma_{min} (\nabla^2 f(z))}{L(f)} |1 - f(x)| \\
    % & = \frac{\sigma_{min} (\nabla^2 f(z))}{L(f)} |1 - q(x)| \\
    % \end{aligned}
    % $$
    % \end{proof}
    % The above theorem shows that, % when the data quality of the clean example is relatively high,
    % the probability of the true label of the clean example is relatively high, 
    Theorem~\ref{theo:main} shows that
    the label noise in adversarial training is proportional to (1) the perturbation radius (2) the data quality. 
    % Larger perturbation radius and low data quality induces higher implicit label noise, which echos the empirical observations made in Appendix~\ref{sect:double-descent-adversarial}.
    % Since implicit label noise modulates the double descent, and by Theorem~\ref{theo:label-noise-perturbation} it depends on the perturbation radius and data quality, the double descent in adversarial training should strongly correlate with the perturbation radius and data quality. 
    % Indeed, it has been observed respectively that small perturbation radius will not induce robust overfitting~\citep{Dong2021ExploringMI}, and high-quality data will not induce robust overfitting~\citep{Dong2021DataPF}.
    % , which will subsequently affect the double descent curves.
    Considering label noise can be an important source of variance in the generalization of deep neural networks~\citep{Nakkiran2020DeepDD, Yang2020RethinkingBT}, such dependence of label noise explains the intriguing observations in the literature that robust overfitting (or epoch-wise double descent) in adversarial training will vanish with small perturbation radii~\citep{Dong2021ExploringMI} or high-quality data~\citep{Dong2021DataPF}. 
    We conduct more controlled experiments to verify this correlation empirically, as shown in Figure~\ref{fig:dependence-perturbation-quality}, 
    % More controlled experiments are conduct in Appendix~\ref{sect:double-descent-adversarial} to verify this correlation empirically.
    
    \begin{figure*}[!ht]
      \centering
      \includegraphics[width=0.8\textwidth]{figures/dependence-perturbation-quality.pdf}
      \caption{(Left) Dependence of robust overfitting on the perturbation radius. A training subset of size 5k is randomly sampled to speed up the training.
      % As the perturbation radius (used for both training and testing) employed in adversarial training increases, both the epoch-wise and model-wise double descent become more prominent.
      $\varepsilon = 0/255$ indicates the standard training where no double descent occurs. 
      (Right) Dependence of robust overfitting on the data quality with a fixed perturbation radius ($\varepsilon = 8/255$). To construct a training subset with high data quality, we first calculate the predictive probability based on an ensemble of multiple models. We then rank all training examples based on the predictive probability and select the top-k ones.
      The curves are smoothed by a window of $5$ epochs to reduce overlapping.
      Here we conduct PGD training on CIFAR-10 with WRN-28-5. 
      More experiment details can be found in the Appendix. % ~\ref{sect:double-descent-adversarial}.
      % As the quality of training data in adversarial training degrades, both the epoch-wise and model-wise double descent become more prominent. In the epoch-wise double descent figure, We smooth each curve by a window of 5 epochs to reduce the overlapping area. For the model-wise double descent the test error at the last checkpoint (solid line) and the test error at the best checkpoint (dashed line) are both shown. 
    %   \chengyu{change data quality to $1 - f_\theta(y|x)$}
      }
      \vspace{-1em}
    \label{fig:dependence-perturbation-quality}
    \end{figure*}

version https://git-lfs.github.com/spec/v1
oid sha256:fb0eb5cb2492c8e56a702e69f7a49008cf0d73cbdb20a740d3ec707281443a22
size 9155





    



% ----------------------------------------------
% ----------------------------------------------
% \subsection{Implicit label noise increases variance in adversarial training}
% \subsection{Implicit label noise induces double descent}
% \label{sect: noise-variance}

% \note{replace this part with results on cifar-10h}
% \chengyu{This should be okay. Augmentation is a good way to show implicit label noise. We just probably need to show before that implicit label noise is nothing but label noise if in a large dataset.}



% % We now show the implicit label noise induces double descent in adversarial training. 
% \chengyu{Move to related work}
% In standard training, the effect of label noise on double descent has been rigorously studied based upon both analytical settings~\citep{Mei2019TheGE, Hastie2019SurprisesIH, Deng2019AMO, Belkin2020TwoMO} and bias-variance analyses~\citep{Jacot2020ImplicitRO, Yang2020RethinkingBT, dAscoli2020DoubleTI}.
% % , with an emphasis on model-wise double descent. 
% % Here, we follow a bias-variance understanding and empirically show that the implicit label noise can promote variance during training and thus produces the epoch-wise double descent.
% Since implicit label noise is just a special case of label noise (Remark~\ref{remark:label-noise}), 
% % \jingbo{do you mean that label flipping is a special case of implicit label noise?} \chengyu{See Remark 2.2}
% and adversarial training can be viewed as standard training on an augmented dataset (Equation~(\ref{eq:outer-minimization})), % it can be inferred that implicit label noise will cause double descent in adversarial training. 
% it can be inferred that implicit label noise will increase the variance and make an evident double descent in adversarial training.
% %  we will not repeat the analyses here, but instead demonstrate in a scenario other than adversarial training where implicit label noise causes double descent.
% To demonstrate this in a straightforward way, in Figure~\ref{fig:dependence-variance} \jingbo{we have a undefined reference here} we employ standard training on a dataset augmented by fixed adversarial perturbation and show it can indeed produce double descent.



% To clearly show the implicit label noise promotes the variance and induces double descent, we employ \emph{adversarial augmentation}, namely the adversarial perturbation is generated by a surrogate model and applied to the training set only once. 
% Standard training is then conducted on the augmented training set to simulate the effect of adversarial training. 
% Such experiment excludes the possibility that the double descent (or robust overfitting) results from the variation of the adversary strength during training.

% Figure~\ref{fig:dependence-variance} shows the adversarial augmentation induces the epoch-wise double descent similar to adversarial training. 
% We further conduct the training on multiple independent training subsets and perform a bias-variance decomposition of the $0$-$1$ loss (see Appendix~\ref{sect:bias-variance-0-1loss}). 
% One can find that the bias almost monotonically decreases throughout the training while the variance increase significantly when the overfitting happens and larger perturbation radius will induces higher variance. 

% \chengyu{Maybe just show figure (a), remove bias-variance analyses and move to appendix.}
% \begin{figure*}[htbp]
%   \centering
%   \includegraphics[width=0.95\textwidth]{figures/reason-variance.pdf}
%   \caption{``Risk'' (Average test error over independent training subsets) obtained when training on an adversarially augmented dataset, as well as the ``Bias'' and ``Variance'' following a bias-variance decomposition of the 0-1 loss. Detailed experiment settings can be found in Appendix~\ref{sect: exp-ad-augment}.
%   }
% \label{fig:dependence-variance}
% \end{figure*}




% Finally, we note that our above analyses echo the existing works. 
% Since implicit label noise modulates the double descent, and by Theorem~\ref{theo:label-noise-perturbation} it depends on the perturbation radius and data quality, the double descent in adversarial training should strongly correlate with the perturbation radius and data quality. Indeed, it has been observed respectively that small perturbation radius will not induce robust overfitting~\citep{Dong2021ExploringMI}, and high-quality data will not induce robust overfitting~\citep{Dong2021DataPF}.
% for small perturbation radius and high-quality dataset, the double descent may not be observed in adversarial training, which echos the recent empirical observation made in \citet{Dong2021ExploringMI} and \citet{Dong2021DataPF} respectively.
% \section{Mitigate Double Descent in Adversarial Training from the Implicit Label Noise Perspective}
\section{Mitigate Label Noise in Adversarial Training}
\label{sect:mitigate-double-descent}

% To mitigate the label noise in adversarial training, we focus on suppressing the implicit label noise from both theoretical and practical perspectives. 
Since the label noise is incurred by the mismatch between the true label distribution and assigned label distribution of adversarial examples in the training set, we wish to find an alternative label (distribution) for the adversarial example to reduce such distribution mismatch.
% \chengyu{But why this is better than the assigned distribution? Because the distribution mismatch is lower bounded by a positive, while here the mismatch can converge to $0$ as long as Lipschitz is sufficiently small.}
% Our goal is to construct an adversarially augmented dataset $\mathcal{D'}$ which contains little or no label noise.
% We've already shown that the predictive label distribution of a classifier trained on the conventional adversarially augmented dataset $\mathcal{D'} = \{(x', \tilde{y}'=\tilde{y})\}$, which we denote as \emph{model probability} in the following discussion, can in fact approximate the true label distribution. 
We've already shown that the predictive label distribution of a classifier trained by conventional adversarial training, which we denote as \emph{model probability} in the following discussion, can in fact approximate the true label distribution. 
Here we show that it is possible to further improve the predictive label distribution and reduce the label noise by calibration.
    
% Since the double descent is mainly caused by the mismatch between the assigned label and true label distributions and also the true label distribution is missing in most of real-world datasets, we propose to approximate the true label distribution.

% \subsection{Approximate the true label distribution}
% --- This previous title feels like we need true label distribution
\subsection{Rectify model probability to reduce distribution mismatch}
% \subsection{Calibrated model probability as adversarial labels}
% approximate the true label distribution}
\label{sect:approximate-true-distribution}





% \sout{We have shown that the double descent in adversarial training is due to the mismatch between the true label distribution and the assigned label distribution, which implicitly introduces label noise and promotes variance.} 
% \sout{A straightforward solution to double descent is thus to employ the true label distribution into training, based on which the label of the adversarial example can be sampled.
% However, since the perturbation radius allowed in adversarial training is typically small, the distribution mismatch might not be reflected by hard-label sampling if the training set is not sufficiently large. 
% To overcome this problem without significantly augmenting the training set, one can directly employ the true label distribution into the training objective.} 
% \sout{For cross-entropy loss it is easy to prove that the training objective with soft-label supervision is equivalent to that with hard labels sufficiently sampled based on the corresponding label distribution, namely}~\footnote{It is worth mentioning that $x$ in this section could either be a clean example or adversarial example. We therefore do not distinguish for simplicity.}
% \begin{equation}
%     \label{eq:loss-soft-label}
%     \mathbbm{E}_{(x, y) \sim p(x, y)} \ell(f(x; \theta), y) = \mathbbm{E}_{x \sim p(x)} \ell(f(x; \theta), p(y|x)).
% \end{equation}

    % \begin{theorem}[Error induced by an approximate label distribution]
    % \label{theorem: loss-error}
    % Let $\tilde{p}(y|x)$ be an approximation of $p(y|x)$, we have
    % \begin{equation}
    %     \ell(f(x;\theta), \tilde{p}) \le \ell(f(x;\theta), p) + 2 \left\|\tilde{p} - p\right\|_{TV} \left\|\log f(x; \theta)\right\|_1,
    % \end{equation}
    % where we neglect the expectation for simplicity. \jingbo{we may want to directly define the loss based on the label distribution as a weighted sum.}
    % \end{theorem}
    % It is worth mentioning that $x$ and $y$ here can be either a clean example or a adversarial example. 
    % One can simply replace them by $x + \delta$ and $y_\delta$ and the conclusion will still hold.
    % We therefore do not distinguish for simplicity.
    
    
% \note{We believe model probability can approach the true label distribution better than the assigned label} \chengyu{need this sentence as the next Section hasn't experiment on temperature and ratio yet. Or we have to emphasize in next section the optimal temperature is already 1.0!}




% \chengyu{We now wish to find such an assigned label of adversarial example such it is closer to the true label distribution $p_\theta(y_\delta | x)$}

%% -- Do Not Delete! -- \note{Problem of using calibration: validation set is not always available in practice. If a validation set is readily available, we can just use early stopping. Solution: We can also use other calibration methods without the need of a validation set, like learning well-calibrated confidence during training.}
%% --------------------

We show that it is possible to reduce the distribution mismatch by \emph{temperature scaling}~\citep{Hinton2015DistillingTK, Guo2017OnCO} enabled in the softmax function.
% We show that it is possible to reduce the distribution mismatch by utilizing the predictive probability of a classifier trained on traditional adversarial labels, which we refer as the \emph{model probability} for simplicity.
% % We denote the predictive probability of a classifier trained based on traditional adversarial labels as \textbf{model probability}. 
% We provide a theoretical guarantee to show that, with \emph{temperature scaling}~\citep{Hinton2015DistillingTK, Guo2017OnCO} enabled in the softmax function, model probability induces a distribution mismatch provably smaller than the traditional adversarial label.
% \note{Need another assumption here such the original label distribution is almost one-hot.} Given in notation section
% \begin{theorem}[Model probability induces smaller distribution mismatch than the traditional adversarial label]
\begin{theorem}[Temperature scaling can reduce the distribution mismatch]
\label{theorem: model-probability}
% \todo{Before we already show that robust model can learn the true label distribution. It is thus natural to use model probability as surrogate assigned label. Here we just need to show with temperature $T$ it can be better.}
    % Let $\mathbbm{1}(\hat{y})$ denote the one-hot vector of a label $\hat{y}$.
    Let $f_\theta(x; T)$ denote the predictive probability of a probabilistic classifier scaled by temperature $T$, namely $f_\theta(x; T)_j = \exp(z_j/T) / (\sum_j \exp(z_j / T)), $
    % $$
    %     f_\theta(x; T)_j = \frac{\exp(z_j/T)}{\sum_j \exp(z_j / T)},
    % $$
    where $z$ is the logits of the classifier from $x$. 
    Let $x'$ be an adversarial example correctly classified by a classifier $f_\theta$, i.e. $\argmax_j f_\theta(x')_j = y'$,
    % Assume the classifier can correctly classify an adversarial example $x_\delta$, i.e. $\argmax_j f(x_\delta; \theta, T)_j = p^*_\delta$,
    then there exists $T$, such that
    $$
    % \| f(x_\delta; \theta, T) - p(y^*_\delta|x) \|_{TV} \le \| \mathbbm{1}(\hat{y}) - p\|_{TV}.
    % \| f_\theta(x_\delta; T) - P(Y^*_\delta|x_\delta) \|_{TV} \le \| P(\tilde{Y}_\delta | x_\delta) - P(Y^*_\delta | x_\delta)\|_{TV}.
    \| f_\theta(x'; T) - P(Y'|x') \|_{TV} \le \| f_\theta(x') - P(Y' | x')\|_{TV}.
    $$

    % where $\mathbbm{1}(\hat{y})$ is the one-hot vector of $\hat{y}$.
\end{theorem}

% \todo{Replace the definition of traditional adversarial label  by the distribution of clean input directly.}


    Another way to further reduce the distribution mismatch is to interpolate between the model probability and the one-hot assigned label.
    % the traditional adversarial label. 
    We show that the interpolation works specifically for incorrectly classified examples and thus can be viewed as a complement to temperature scaling.
    \begin{theorem}[Interpolation can further reduce the distribution mismatch]
    \label{theorem: model-probability-coefficient}
    % \chengyu{Will it be a problem if $y$ is now no longer the argmax?}
        % Assume the classifier $f$ incorrectly classify an adversarial example $x_\delta$, i.e. $\argmax_j f(x_\delta; \theta)_j \ne p^*_\delta$.
        Let $x'$ be an adversarial example incorrectly classified by a classifier $f_\theta$, i.e. $\argmax_j f_\theta(x'; T)_j \ne y'$. % \argmax_j p(y=j|x)$. 
        % Assume $\hat{y} = \argmax_j~p(y=j|x)$ and $\max_j p(y=j|x) \ge 1/2$, then there exists $\lambda$, such that
        Assume $\max_j P(Y'=j|x') \ge 1/2$, then there exists an interpolation ratio $\lambda$, such that
        \begin{equation*}
        \small
        % \|  P_{\theta}^{T, \lambda}(Y'|x') - P(Y' | x') \|_{TV} \le \| f_\theta(x_\delta; T) - P(Y_\delta^* | x_\delta)\|_{TV},  
         \| f_\theta(x'; T, \lambda) - P(Y' | x') \|_{TV} \le \| f_\theta(x'; T) - P(Y' | x')\|_{\text{TV}},
        \normalsize
        \end{equation*}
        where $f_\theta(x'; T, \lambda) = \lambda \cdot f_\theta(x'; T) + (1 - \lambda)\cdot P(\tilde{Y}' | x')$.
        % where $P_{\theta}^{T, \lambda}(Y_\delta|x_\delta) = \lambda \cdot f_\theta(x_\delta; T) + (1 - \lambda)\cdot P(\tilde{Y}_\delta | x_\delta) $.
        % \jingbo{Also, did you forget the include $T$ in $f$?}\chengyu{This theorem should work for any $T$, so i neglect it.}
    \end{theorem}
    % Note the above theorem focus on incorrectly classified examples and thus can be regarded as a complement to Theorem~\ref{theorem: model-probability}.
    % We show that a proper interpolation approximates provably better for incorrectly classified examples, therefore this theorem can be regarded as a complement to the above one.
    
    
    As a summarization, 
    % an potentially better approximation of the true label distribution based on a model can be formulated as
    to reduce the distribution mismatch, we propose to use 
    $f_\theta(x'; T, \lambda)$
    % $P_{\theta}^{T, \lambda}(Y_\delta|x_\delta)$ 
    % following distribution 
    as the assigned label of the adversarial example in adversarial training, which we refer as the \emph{rectified model probability}.
    % \begin{equation}
    %     \label{eq:approximate-label-distribution}
    %     P_{\theta}^{T, \lambda}(Y_\delta|x_\delta) = \lambda \cdot f_\theta(x_\delta; T) + (1- \lambda) \cdot P(\tilde{Y}_\delta | x_\delta),
    % \end{equation}
    % We refer this label distribution as the \textbf{rectified model probability}.
    
    In Appendix~\ref{sect:optimal-temperature-mixup}, we show that the optimal hyper-parameters (i.e. $T$ and $\lambda$) of almost all training examples concentrate on the same set of values by studying on a synthetic dataset with known true label distribution. 
    % \jingbo{can we name the values here?}
    Therefore it is possible to find an universal set of hyper-parameters that reduce the distribution mismatch for all adversarial examples. 
    % the rectified model probability can often approximate the true distribution sufficiently well with



version https://git-lfs.github.com/spec/v1
oid sha256:fc01b3c0a6353e705e9d70f0e94f348c47073c0d6ebf49c38c83cccabcc985f0
size 5193

% \subsection{Real-world Experiments}
\subsection{Rectified model probability mitigates robust overfitting}
\label{sect:exp-practical-adversarial-training}

% \chengyu{This section validates the effectiveness of the confidence calibration. Therefore right after the analysis}







We now work on a realistic dataset (CIFAR-10) to demonstrate the rectified model probability 
% proposed in Equation~(\ref{eq:approximate-label-distribution})
can effectively mitigate the robust overfitting, or equivalently the epoch-wise double descent in adversarial training. The outer minimization of adversarial training (Equation~(\ref{eq:outer-minimization})) now becomes 
    % \begin{equation}
    %     \theta^* = \argmin_\theta \mathbbm{E}_\mathcal{D_\delta}~ \ell\left(f_\theta(x_\delta), P_{\theta^{\text{Trad}}}^{T, \lambda}(Y_\delta | x_\delta)\right),
    % \end{equation}
    \begin{equation}
        % \theta^* = \argmin_\theta \mathbbm{E}_\mathcal{D_\delta}~ \ell\left(f_\theta(x_\delta), P_{\theta^{\text{Trad}}}^{T, \lambda}(Y_\delta | x_\delta)\right),
        \theta^* = \argmin_\theta \mathbbm{E}_\mathcal{D'}~ \ell\left(f_\theta(x'), f_{\hat{\theta}}(x'; T, \lambda)_{y'}\right),
    \end{equation}
    where $\hat{\theta}$ denotes the parameters of a classifier adversarially trained beforehand.
    The details of the experimental setting are available in the Appendix. 
    % ~\ref{sect: exp-practical}.

% \todo{Rephrase our training framework: 1) get model probability by one training 2) find optimal $T$ and $\lambda$ 3) Adversarial training on the rectified model probability.}

% \jingbo{move Figure~\ref{fig:method-grid-search} to this page? Simplify the caption a bit by moving some of the sentences to the paragraph here.} 
As shown in Figure~\ref{fig:method-grid-search}, adversarial training on rectified model probability can mitigate the robust overfitting when the temperature $T$ and interpolation ratio $\lambda$ are optimal. Such optimal hyperparameters perfectly aligns with the ones automatically determined by Equation~(\ref{eq:calibration}).







\section{Experiments}
\label{sect:experiment}

\FloatBarrier










% \begin{table*}[!ht]
%   \small
%   \caption{Performance of our method combined with SWA and an additional standard teacher. %  on different datasets.
%   }
%   \vspace{0.5ex}
%   \label{table:result-technique}
%   \centering
%   \small
%   \begin{tabular}{clcccccccc}
%     \toprule
%     \multirow{2}{*}{Dataset} & \multirow{2}{*}{Setting} & \multirow{2}{*}{$T$} & \multirow{2}{*}{$\lambda$} & \multicolumn{3}{c}{Robust Acc. (\%)} & \multicolumn{3}{c}{Standard Acc. (\%)}\\
%      & & &  & Best & Last & Diff. & Best & Last & Diff.\\
%     \midrule
%     % \multirow{3}{*}{CIFAR-10} 
%     % & AT & - & - &  $47.35$ & $41.42$ & $ 5.93$ &  $82.67$ &  $84.91$ & -$2.24$ \\
%     % &  KD-AT + KD-Std + SWA & $2$ & $0.5$ & $49.98$ & $49.89$ & $0.09$ & $\textbf{85.06}$ & $\textbf{85.52}$ & -$0.46$\\
%     % & KD-AT-Auto + KD-Std + SWA & $1.47^*$ & $0.8^*$ & $\textbf{50.03}$ & $\textbf{50.05}$ & $\textbf{-0.02}$ & $84.69$ & $84.91$ & $\textbf{-0.22}$\\ 
%     %     \midrule
%     \multirow{3}{*}{SVHN} 
%     & AT & - & - & $47.83$ & $39.77$ & $8.06$ & $90.18$ & $91.11$ & -$0.93$\\
%     & KD-AT + KD-Std + SWA & $2$ & $0.5$ & $47.88$ & $46.46$ & $1.42$ & $\textbf{91.59}$ & $\textbf{91.76}$ & $\textbf{-0.17}$\\
%     & KD-AT-Auto + KD-Std + SWA  & $1.53^*$ & $0.83^*$ & $\textbf{50.58}$ & $\textbf{50.09}$ & $\textbf{0.49}$ & $90.54$ & $90.76$ & -$0.22$\\ 
%     \bottomrule
%     % \tablefootnote{$^*$ indicates the best hyper-parameter searched.}
%   \end{tabular}
% \end{table*}





% \note{Should describe our method again as a whole again? Maybe discuss the difference between our work and Chen et al in more detail in the related work. We further calibrate their method to achieve even smaller overfitting w/o additional human effort.}
% \note{So Why not SWA? Why not clean teacher? Not convenient? Focus on ad training?}
% \chengyu{This section extends to other settings, therefore an independent section}.


% \chengyu{Possible names of methods
% \note{KD-AT}
% \note{AKD-AT (Adaptive knowledge distillation)}
% \note{LabelRectifier}
% \note{Rectified }
% }

% In this section, we verify the effectiveness of our method on multiple datasets. 


\begin{figure*}[t]
  \centering
  \includegraphics[width=0.98\linewidth]{figures/mitigate-overfitting.pdf}
  \vspace{-1ex}
  \caption{Our method can effectively mitigate robust overfitting for different datasets. 
  }
\label{fig:mitigate-overfitting}
% \vspace{-2ex}
\end{figure*}

% \jingbo{it's a bit odd to me to put the figure in the beginning of a section. }

\smallsection{Experiment setup}
We conduct experiments on three datasets including CIFAR-10, CIFAR-100~\citep{Krizhevsky2009LearningML} and Tiny-ImageNet~\citep{Le2015TinyIV}. We conduct PGD training on pre-activation ResNet-18~\citep{He2016IdentityMI} with $10$ iterations and perturbation radius $8/255$ by default. We evaluate robustness against $\ell_\infty$ norm-bounded adversarial attack with perturbation radius $8/255$, and employ AutoAttack~\citep{Croce2020ReliableEO} for reliable evaluation.  
% We employ SGD as the optimizer with momentum and weight decay set as $0.9$ and $0.0005$ respectively. We conduct the training for $160$ epochs with the learning rate starting at $0.1$ and reduced by a factor of $10$ at epoch $80$ and $120$.
Appendix~\ref{sect:more-result} includes results on additional model architectures (e.g., VGG~\citep{Simonyan2015VeryDC}, WRN),  adversarial training methods (e.g., TRADES~\citep{Zhang2019TheoreticallyPT}, FGSM~\citep{Goodfellow2015ExplainingAH}), and evaluation metrics (e.g., PGD-1000 (PGD attack with $1000$ iterations), Square Attack~\citep{Andriushchenko2020SquareAA}, RayS~\citep{Chen2020RaySAR}). More setup details can be found in Appendix~\ref{sect:exp-setting-all}.

% as an example and include results on TRADES~\citep{Zhang2019TheoreticallyPT} and FGSM~\citep{Goodfellow2015ExplainingAH} in Appendix~\ref{sect:more-result}.  
% The results against more evaluation metrics such as PGD-$1000$ (PGD attack with $1000$ iterations), Square Attack~\citep{Andriushchenko2020SquareAA} and RayS~\citep{Chen2020RaySAR} will be provided in Appendix~\ref{sect:more-result}. 
% We choose pre-activation ResNet-18~\citep{He2016IdentityMI} in the main paper and experiment on more architectures including VGG~\citep{Simonyan2015VeryDC} and WRN in Appendix~\ref{sect:more-result}.



% -------------- Dataset ----------------
\begin{table*}[!t]
  \small
  \caption{Performance of our method on different datasets. $^*$ denotes the hyper-parameters automatically determined by our method. % \chengyu{Results on SVHN. Figure too?}
   }
  \vspace{0.5ex}
  \label{table:result-dataset}
  \centering
  \small
  \begin{tabular}{clcccccccc}
    \toprule
    \multirow{2}{*}{Dataset} & \multirow{2}{*}{Setting} & \multirow{2}{*}{$T$} & \multirow{2}{*}{$\lambda$} & \multicolumn{3}{c}{Robust Acc. (\%)} & \multicolumn{3}{c}{Standard Acc. (\%)}\\
     & & &  & Best & Last & Diff. & Best & Last & Diff.\\
    \midrule
\multirow{3}{*}{CIFAR-10} 
& AT & - & - &  $47.35$ & $41.42$ & $ 5.93$ &  $82.67$ &  $84.91$ & -$2.24$ \\
 & KD-AT & $2$ & $0.5$ &  $48.76$ & $46.33$ & $ 2.43$ &  $82.89$ &  $\textbf{85.49}$ & -$2.60$ \\ 
 & KD-AT-Auto & $1.47^*$ & $0.8^*$ &  $\textbf{49.05}$ & $\textbf{48.80}$ & $ \textbf{0.25}$ &  $\textbf{84.26}$ &  $84.47$ & $\textbf{-0.21}$ \\ 
    \midrule
\multirow{3}{*}{CIFAR-100}
& AT & - & - &  $24.79$ & $19.75$ & $ 5.04$ &  $57.33$ &  $57.42$ & -$0.09$ \\ 
& KD-AT & $2$ & $0.5$ &  $25.77$ & $23.58$ & $ 2.19$ &  $57.24$ &  $\textbf{60.04}$ & -$2.80$ \\ 
& KD-AT-Auto & $1.53^*$ & $0.83^*$ &  $\textbf{26.36}$ & $\textbf{26.24}$ & $\textbf{0.12}$ &  $\textbf{58.80}$ &  $59.05$ & $\textbf{-0.25}$ \\ 
\midrule
\multirow{3}{*}{Tiny-ImageNet} 
& AT & - & - &  $17.20$ & $15.40$ & $ 1.80$ &  $47.72$ &  $47.62$ & $ 0.10$ \\ 
& KD-AT & $2$ & $0.5$ &  $17.86$ & $17.18$ & $ 0.68$ &  $\textbf{47.73}$ &  $\textbf{48.28}$ & -$0.55$ \\ 
& KD-AT-Auto  & $1.23^*$ & $0.85^*$ &  $\textbf{18.29}$ & $\textbf{18.39}$ & $\textbf{-0.10}$ &  $47.46$ &  $47.56$ & $\textbf{-0.10}$ \\ 
    \bottomrule
    % \tablefootnote{$^*$ indicates the best hyper-parameter searched.}
  \end{tabular}
\end{table*}
% \vspace{-1.5em}


\smallsection{Results \& Discussions}
Our method is essentially the baseline adversarial training with a robust-trained self-teacher, equipped with an algorithm automatically deciding the optimal hyper-parameters, which we now denote as KD-AT-Auto.
We compare KD-AT-Auto with two baselines: regular adversarial training (AT), and adversarial training combined with self-distillation (KD-AT) with fixed temperature $T=2$ and interpolation ratio $\lambda=0.5$ as suggested by \citet{chen2021robust}. 

As shown in Figure~\ref{fig:mitigate-overfitting}, our method can effectively mitigate robust overfitting for all datasets, with both standard accuracy (SA) and robust accuracy (RA) constantly increasing throughout training. In Table~\ref{table:result-dataset}, we measure the difference between the RA at the best checkpoint (Best) and at the last checkpoint (Last) to clearly show the overfitting gap. Our method can reduce the overfitting gap to less than $0.5\%$ for all datasets. One may note that self-distillation with fixed hyper-parameters is in fact inferior in terms of reducing robust overfitting, while its effectiveness can be significantly improved with the optimal hyper-parameters automatically determined by our method, which further verifies our understanding of robust overfitting. 
Compared with self-distillation with fixed hyper-parameters, our method can also boost both RA and SA at the best checkpoint for all datasets.


Our method can further be combined with orthogonal techniques such as Stochastic Weight Averaging (SWA)~\citep{Izmailov2018AveragingWL} and additional standard teachers as mentioned in previous work~\citep{chen2021robust} to achieve better performance. More results and discussion can be found in Appendix~\ref{sect:additional-technique}.


% \note{One thing is not doing good is the last standard acc.}


    % We note that employing Equation~(\ref{eq:approximate-label-distribution}) as the supervision in adversarial training with cross-entropy loss is equivalent to the self-training method proposed by~\citet{chen2021robust} without a standard-trained teacher\footnote{We find the standard-trained teacher does not help mitigate the robust overfitting, but rather improve the standard accuracy following a regular knowledge distillation practice.}. 
    % However, the temperature $T$ and the interpolation ratio $\lambda$ have to be manually set in the self-training method, whereas they can be automatically tuned in our method. Figure~\ref{fig:mitigate-overfitting} shows that our automatically-tuned can reduce the overfitting gap better than the fixed hyperparameters for a variety of datasets.
    
    
% \subsection{Combined with additional orthogonal techniques}
% \label{sect:additional-technique}

% Our method can further be combined with other orthogonal techniques such as Stochastic Weight Averaging (SWA)~\citep{Izmailov2018AveragingWL} and additional standard teachers as mentioned in previous work~\citep{chen2021robust} to achieve better performance. 

% More results and discussion can be found in Appendix~\ref{sect:additional-technique}.

% Here, we show that combined with the additional techniques proposed in~\citep{chen2021robust}, our method can achieve better performance.

% We note that our proposed method is essentially the baseline knowledge distillation for adversarial training with a robustly trained self-teacher, equipped with an algorithm that automatically finds its optimal hyperparameters (i.e. the temperature $T$ and the interpolation ratio $\lambda$). Stochastic Weight Averaging (SWA) and additional standard teachers employed in~\citep{chen2021robust} are orthogonal contributions. KD-AT-Auto can certainly be combined with SWA and KD-Std to achieve better performance. 

% We note that the method proposed by \citet{chen2021robust}, namely combining knowledge distillation with self-distillation, an additional standard teacher and SWA (KD-AT + KD-Std + SWA), can already reduce the overfitting gap to almost $0$. It is thus hard to see any further reduction by combining our method. To this end, we introduce an extra dataset SVHN~\citep{Netzer2011ReadingDI}. As shown in Table~\ref{table:result-technique}, on SVHN, the above method KD-AT + KD-Std + SWA still produces a high overfitting gap (also see Appendix A1.3 in~\citep{chen2021robust}), whereas by combining with our method (KD-AT-Auto + KD-Std + SWA), the overfitting gap can be further reduced to almost $0$. This demonstrates the advantage of our principle-guided method on mitigating robust overfitting. More results regarding other datasets and detailed experiment setup can be found in Appendix~\ref{sect:additional-technique}.

% As shown in Table~\ref{table:result-technique}, on CIFAR-10, KD-AT + KD-Std + SWA~\citep{chen2021robust} can already reduce the overfitting gap (difference between the best and last robust accuracy) to almost 0, while KD-AT-Auto + KD-Std + SWA maintains an overfitting gap close to 0. Interestingly, on the SVHN dataset~\citep{Netzer2011ReadingDI}, where KD-AT + KD-Std + SWA still produces a high overfitting gap (also see Appendix A1.3 in~\citep{chen2021robust}), KD-AT-Auto + KD-Std + SWA can further push this gap to close to 0. 

% Here, the interpolation ratio of the standard teacher is fixed as $0.2$ and the SWA starts at the first learning rate decay for all experiments. We employ PGD-AT~\citep{Madry2018TowardsDL} as the base adversarial training method and conduct experiments with a pre-activation ResNet-18. The robust accuracy is evaluated with AutoAttack. Other experiment details are in line with Appendix~\ref{sect: exp-practical}.


% Furthermore, we note that~\citep{chen2021robust} shows SWA and KD-Std are essential components to mitigate robust overfitting on top of KD-AT, while we show that KD-AT itself can mitigate robust overfitting by proper parameter tuning. We are thus able to separate these components and allow a more flexible selection of hyperparameters in diverse training scenarios without fear of overfitting. In particular, although~\citep{chen2021robust} suggests SWA starting at the first learning rate decay (exactly when the overfitting starts) mitigates robust overfitting, the effectiveness of SWA on mitigating overfitting may strongly depend on its hyper-parameter selection including $s_0$, i.e., the starting epoch and $\tau$, i.e., the decay rate\footnote{SWA can be implemented using an exponential moving average $\theta'$ of the model parameters $\theta$ with a decay rate $\tau$, namely $\theta' \leftarrow \tau \cdot \theta' + (1-\tau) \cdot \theta$ at each training step~\citep{Rebuffi2021FixingDA}.}, which is also mentioned in recent work~\citep{Rebuffi2021FixingDA}. We also did some additional experiments on CIFAR-10 following the SWA setting in~\citep{Rebuffi2021FixingDA} to demonstrate the wide applicability of our method. As shown by Table~\ref{table:result-swa}, when changing the hyperparameters of SWA, KD-AT + KD-Std + SWA cannot consistently mitigate robust overfitting, while KD-AT-Auto + KD-Std + SWA can maintain an overfitting gap close to 0 and achieve better robustness as well. 



version https://git-lfs.github.com/spec/v1
oid sha256:cecbdb5451bd04200beba375ca04996b5070926e8461bb6431b9dacd289b0760
size 6613


% \newpage
\bibliography{icml2022}
\bibliographystyle{icml2022}

% \newpage
version https://git-lfs.github.com/spec/v1
oid sha256:3e2119875f889bfc4b5d4e6563fd5130ef1bde9eaf9e92eab76016a8f6472291
size 4329


\newpage
\appendix\onecolumn
\section{Proofs}
\label{sect:proof-all}


version https://git-lfs.github.com/spec/v1
oid sha256:e9c25e80de0bebe391d1d7eba9b176ee6c483733f3d6a6e006d0d2b2b7bbe9f9
size 1257


% ------------------------------------------
% \subsection{Proofs and remarks for the existence of label noise}
\subsection{Proofs in Section~\ref{sect:reason-true}}
\label{sect:label-noise-more-proof}


\smallsection{Proof of Lemma~\ref{theorem:distribution-mismatch-true-model}}

\begin{proof}
For simplicity, we consider the adversarial perturbation generated by FGSM. Other adversarial perturbation can be viewed as a Taylor series of such perturbation.

    \begin{equation}
        \delta = -\varepsilon  \frac{\nabla~f(x)_y}{\|\nabla~ f(x)_y\|},  
    \end{equation}

First, we bound the distribution mismatch by gradient norm.
$$
    \begin{aligned}
    \|P(Y|x) - P(Y'|x')\|_{\text{TV}}
    & = \frac{1}{2} \sum_j \left|P(Y=j|x) - P(Y'=j|x')\right|\quad \boxed{\text{TV distance}}\\
    & \ge \frac{1}{2} \left|P(Y=y|x) - P(Y'=y|x')\right|\\
    & = \frac{1}{2} \left|f(x)_{y} - f(x')_{y}\right| \\
    & = \frac{1}{2} \left[ -\nabla f(x)_{y} \cdot \delta - \frac{1}{2}\delta^T \nabla^2 f(z)_{y} \delta \right]\\
    & \ge \frac{1}{2} \left[-\nabla f(x)_{y} \cdot \delta - \frac{\sigma_M}{2}\|\delta\|_2^2\right]  \quad \boxed{\text{Bounded Hessian}} \\ % \boxed{\text{Local convexity}} \\
    & \ge \frac{1}{2} \left[\varepsilon \frac{\|\nabla f(x)_{y}\|^2_2}{\|\nabla f(x)_{y}\|} - \frac{\sigma_M}{2} \varepsilon^2 \frac{\|\nabla f(x)_{y}\|^2_2}{\|\nabla f(x)_{y}\|^2}  \right].\\
    \end{aligned}
$$
Now if $\|\cdot\| = \|\cdot\|_2$, we have
\begin{equation}
  \label{eq:mismatch-bound-l2}
  \|P(Y|x) - P(Y'|x')\|_{\text{TV}} \ge \frac{1}{2} \left[\varepsilon \|\nabla f(x)_{y}\|_2 - \frac{\sigma_M}{2}\varepsilon^2\right].    
\end{equation}
If $\|\cdot\| = \|\cdot\|_\infty$, we can utilize the fact that $\|\cdot\|_\infty \le \|\cdot\|_2 \le \sqrt{d} \|\cdot\|_\infty$,
thus 
\begin{equation}
    \label{eq:mismatch-bound-linf}
    \|P(Y|x) - P(Y'|x')\|_{\text{TV}} \ge \frac{1}{2} \left[\varepsilon \|\nabla f(x)_{y}\|_\infty - \frac{\sigma_M}{2} \varepsilon^2 \sqrt{d}\right].
\end{equation}
    
Second, we bound the gradient norm by the $L$-local Lipschitzness assumption.
    % \chengyu{Implicit assumption, there is a $x^*$ within the $\varepsilon$ ball of $x$ where $f$ is Lipschitz.}     
    Let $x^*$ be a closest input that achieves the local maximum on the predicted probability at $y$, namely $x^* = \argmin_{z \in X, f(z)_y = 1} \|x - z\|$. Because $x^*$ is the local maximum and $f$ is continuously differentiable, $\nabla f(x^*)_y = 0$, thus
    $$
    \nabla f(x)_y 
    = \nabla f(x^*)_y  + \nabla^2 f(z)_y  (x - x^*) = \nabla^2 f(z)_y  (x - x^*).
    $$
    Therefore we have
    $$
    \begin{aligned}
    \|\nabla f(x)_y \| 
    & =  \|\nabla^2 f(z)_y  (x - x^*) \| \\
    & \ge \sigma_{m} \|x - x^*\| \\
    & \ge \sigma_{m}  \frac{|f(x^*)_y  - f(x)_y |}{L} \\
    & = \frac{\sigma_{m}}{L} (1 - f(x)_y). \\
    % & \ge \frac{\sigma_{m} }{L(f)} |1 - q(x)| \\
    \end{aligned}
    $$
Plug this into Equation~(\ref{eq:mismatch-bound-l2}) or Equation~(\ref{eq:mismatch-bound-linf}) we then obtain the desired result.
\end{proof}




\smallsection{Proof of Lemma~\ref{theorem:implicit-label-noise}}
    \begin{proof}
    
    First, we show that the expectation of the label error is lower bounded by the mismatch between the true label distribution and the assigned label distribution.
    \begin{equation}
    % \small
    \begin{aligned}
    \|P(\tilde{Y}|x) - P(Y|x)\|_{TV} 
    & = \frac{1}{2} \sum_j |P(\tilde{Y}=j|x) - P(Y=j|x)|  \\
    & = \frac{1}{2} \sum_j |P(\tilde{Y}=j, Y=j|x) + P(\tilde{Y}=j, Y\ne j|x) \\
    & \quad - P(Y=j, \tilde{Y}=j|x)- P(Y=j, \tilde{Y}\ne j|x)| \\
    & =  \frac{1}{2} \sum_j | P(\tilde{Y}=j, Y\ne j|x) - P(Y=j, \tilde{Y}\ne j|x)| \\
    & \le \frac{1}{2} \sum_j  P(\tilde{Y}=j, Y\ne j|x) + P(Y=j, \tilde{Y}\ne j|x) \\
    & = P(Y’\ne Y | x) \\
    & = P(E=1|x)\\
    \end{aligned}
    \end{equation}
    
    Second, given a sampled training set $\mathcal{D}=\{(x_i, \tilde{y}_i)\}_{i\in [N]}$, the empirical measure of label error $E$ should converge to its expectation almost surely, namely
    $$
    \lim_{N\to \infty} p_e(\mathcal{D}) = \lim_{N\to \infty} \frac{1}{N} \sum_{i\in [N]} e_i = \mathbb{E} [E] = P(E = 1).
    $$
    Using standard concentration inequality such as Hoeffding's inequality we have, 
    % $$
    % P(|p_e(\mathcal{D}) - P(E = 1) \le \epsilon|) \ge 1 - 2 \exp\left(-2N\epsilon^2\right),
    % $$
    % which means 
    with probability $ 1 - \delta$,
    $$
    |p_e(\mathcal{D}) - P(E = 1)| \le \sqrt{\frac{1}{2N}\log\frac{2}{\delta}}.
    $$
    This implies
    $$
    p_e(\mathcal{D}) \ge P(E=1) -  \sqrt{\frac{1}{2N}\log\frac{2}{\delta}}.    
    $$
    
    Since $P(E = 1) = \mathbb{E}_x P(E=1|x)$, we have, with probability $1 - \delta$,
    $$
    p_e(\mathcal{D}) \ge \mathbb{E}_x \|P(\tilde{Y}| x) - P(Y | x)\|_{\text{TV}} -\sqrt{\frac{1}{2N}\log\frac{2}{\delta}} .
    $$
    which means $p_e(\mathcal{D}) > 0$ as long as $N$ is large.
    
    \end{proof}
    
    
    
    
    
    
\smallsection{Proof of Theorem~\ref{theo:main}}
    \begin{proof}
    First, by the fact that $P(\tilde{Y}'| x') = P(\tilde{Y} | x)$ and $P(\tilde{Y} | x) = P(Y | x)$ we have $P(\tilde{Y}'| x') =  P(Y | x)$.
    
    Therefore, apply Lemma~\ref{theorem:implicit-label-noise} to an adversarially augmented training set we have with probability $1 - \delta$,
    $$
    \begin{aligned}
    p_e(\mathcal{D'}) 
    & \ge \mathbb{E}_x \|P(\tilde{Y}'| x') - P(Y' | x')\|_{\text{TV}} -\sqrt{\frac{1}{2N}\log\frac{2}{\delta}} \\
    & \ge \mathbb{E}_x \|P(Y | x) - P(Y' | x')\|_{\text{TV}} -\sqrt{\frac{1}{2N}\log\frac{2}{\delta}}. \\
    \end{aligned}
    $$
    Further, apply Lemma~\ref{theorem:distribution-mismatch-true-model} and the definition of data quality, we have with probability $1 - \delta$,
    $$
    \begin{aligned}
    p_e(\mathcal{D'}) 
    & \ge \frac{\varepsilon}{2} (1 - \mathbb{E}_x f(x)_y) \frac{\sigma_m}{L}  - \frac{\varepsilon^2}{4} \sigma_M - \sqrt{\frac{1}{2N}\log\frac{2}{\delta}} \\
    & \ge \frac{\varepsilon}{2} (1 - q(\mathcal{D})) \frac{\sigma_m}{L}  - \frac{\varepsilon^2}{4} \sigma_M - \sqrt{\frac{1}{2N}\log\frac{2}{\delta}}. \\    
    \end{aligned}
    $$
    % $$
    % \begin{aligned}
    % \mathbb{E}_{y'} P(\tilde{Y’}\ne Y' | Y'=y', x')
    % & \ge
    % \|P(\tilde{Y’}|x') - P(Y’|x')\|_{TV} \\
    % & = \|P(\tilde{Y’}|x') - P(\tilde{Y}|x) + P(\tilde{Y}|x) - P(Y|x) + P(Y|x) - P(Y’|x')\|_{TV} \\
    % & = \|P(Y|x) - P(Y’|x')\|_{TV} \\
    % \end{aligned}
    % $$
    
    \end{proof}
    

    

version https://git-lfs.github.com/spec/v1
oid sha256:4c20a5b53d73abccb1fce4c76c5dece6f3aea2e8a55b7bc4e02c54d8f9e776c4
size 5859


version https://git-lfs.github.com/spec/v1
oid sha256:c1b6ee6986c75d24a1d6e66db5aad82072562a37af9cff13917d1ba3a4b01b54
size 5467

version https://git-lfs.github.com/spec/v1
oid sha256:0df0a5bf203733811ea888a1d46f8bdc8087b86bf6566202bea5d1b325066ef6
size 922

version https://git-lfs.github.com/spec/v1
oid sha256:2ffb060e5391d3a0589c6d7672e4779304570ae3291d81a0f9e1ad5d499cf952
size 10156

version https://git-lfs.github.com/spec/v1
oid sha256:e72bab2b96cb5a401a7b14b64bb64b6c5206805e55e2320b228b193015f4d349
size 15194

version https://git-lfs.github.com/spec/v1
oid sha256:2a8ae2e4d230a5994262ff3ceeb82d6d76cb7e71d64c5ebfce1abe145ddb5416
size 7539

version https://git-lfs.github.com/spec/v1
oid sha256:b2edfb452854a8e1f4e110f3a41818115a41f642221a6bac121bc6ac60f1ae2b
size 8760

version https://git-lfs.github.com/spec/v1
oid sha256:36414b9a7b727f751e8059d35eb64ff800cf2b4c9c20db36a10616b88f51f1c4
size 6576



\end{document}