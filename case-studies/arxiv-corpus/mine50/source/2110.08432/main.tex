\documentclass[twoside,11pt]{article}

% Any additional packages needed should be included after jmlr2e.
% Note that jmlr2e.sty includes epsfig, amssymb, natbib and graphicx,
% and defines many common macros, such as 'proof' and 'example'.
%
% It also sets the bibliographystyle to plainnat; for more information on
% natbib citation styles, see the natbib documentation, a copy of which
% is archived at http://www.jmlr.org/format/natbib.pdf

\usepackage{jmlr2e}

\usepackage{amsmath,amssymb,xspace}

\usepackage[utf8]{inputenc} % allow utf-8 input
\usepackage[T1]{fontenc}    % use 8-bit T1 fonts
\usepackage{hyperref}       % hyperlinks
\usepackage{url}            % simple URL typesetting
\usepackage{booktabs}       % professional-quality tables
\usepackage{amsfonts}       % blackboard math symbols
\usepackage{nicefrac}       % compact symbols for 1/2, etc.
\usepackage{microtype}      % microtypography
\usepackage{xcolor}         % colors

% Recommended, but optional, packages for figures and better typesetting:
\usepackage{microtype}
\usepackage{graphicx}
%\usepackage{subfigure}
\usepackage{booktabs} % for professional tables

% Recommended, but optional, packages for figures and better typesetting:
\usepackage{microtype}
\usepackage{graphicx}

% For citations
\usepackage{natbib}
% \usepackage{amsfonts}
% \usepackage{amssymb, amsmath,amsthm}

% For algorithms
\usepackage{algorithm}
\usepackage{algorithmic}
\usepackage{paralist}
\usepackage{multirow}
% Added by Author
% use Times
\usepackage{times}
% For figures
\usepackage{wrapfig}
%\usepackage[authoryear]{natbib}

% For algorithms
\usepackage{url,enumerate}
\usepackage{color,xcolor}
\usepackage{makeidx}  % allows for indexgeneration
% \usepackage{amsmath,amssymb}
\usepackage{mathtools}
% \usepackage[small, compact]{titlesec}
\usepackage{xspace}
\usepackage{epstopdf}
\usepackage{cite}

% For algorithms
\usepackage{mathrsfs}
\usepackage{times}
\usepackage{enumerate}
\usepackage{color}
\usepackage{graphicx,epsfig}
% \usepackage{amsmath,amssymb,xspace}
\usepackage{url}
%\usepackage{subfigure}
\usepackage{hyperref}
\usepackage{bm}
\usepackage{bbm}
\usepackage{upgreek}
\usepackage{cleveref}
\usepackage{multirow}
\usepackage{ulem}
\usepackage{cancel}
\usepackage{subcaption}
\usepackage{dsfont}
\usepackage{adjustbox}

version https://git-lfs.github.com/spec/v1
oid sha256:f49906d90d0fdec59ef023186372d8cd41bc550cde97aa683448f5ceecf400d7
size 6103


\newcommand{\dataset}{{\cal D}}
\newcommand{\fracpartial}[2]{\frac{\partial #1}{\partial  #2}}

% \def\thefootnote{*}\footnotetext{equal contribution}

%-------------------------------------------------------------------------
\newcommand\blfootnote[1]{%
  \begingroup
  \renewcommand\thefootnote{}\footnote{#1}%
  \addtocounter{footnote}{-1}%
  \endgroup
}

% % Enter the paper's authors in order
% \addauthor{Author1$^\ast$}{mail1@mail.com}{1}
% \addauthor{Author2$^\ast$}{mail2@mail.com}{1} 
% \addauthor{Author3}{mail3@mail.com}{1}

% % Enter the institutions
% % \addinstitution{Name\\Address}
% \addinstitution{
% Institute\\
%  Address
% }

% \def\eg{\emph{e.g}\bmvaOneDot}
% \def\Eg{\emph{E.g}\bmvaOneDot}
% \def\etal{\emph{et al}\bmvaOneDot}

\newcommand{\ours}{{A-MAML}\xspace}
\newcommand{\bmaml}{{{B-MAML}}\xspace}
\newcommand{\pmaml}{{{P-MAML}}\xspace}
\newcommand{\emaml}{{{E-MAML}}\xspace}
\newcommand{\fomaml}{{{FOMAML}}\xspace}
\newcommand{\imaml}{{{iMAML}}\xspace}
\newcommand{\maml}{{{MAML}}\xspace}
\newcommand{\rap}{{{Reptile}}\xspace}
%\newcommand{\rap}{{\color{red}{{{Reptile}}\xspace}}}
\newcommand{\zhec}[1]{\textcolor{blue}{#1}}
%\newcommand{\zhec}[1]{#1}
\newcommand{\cmt}[1]{}
\newcommand{\eg}{{\textit{e.g.},}\xspace}
\newcommand{\ie}{{\textit{i.e.},}\xspace}
\newcommand{\etc}{{\textit{etc}.}\xspace}

\newcommand{\zsdc}[1]{{#1}}

\newcommand{\akil}[1]{{\leavevmode\color{red}{#1}}}
\renewcommand{\akil}[1]{}
% \newcommand{\cmt}[1]{}
% \newcommand{\eg}{{\textit{e.g.},}\xspace}
% \newcommand{\ie}{{\textit{i.e.},}\xspace}
% \newcommand{\etc}{{\textit{etc}.}\xspace}

% %-------------------------------------------------------------------------
% % Document starts here
% \begin{document}

% \maketitle
% % This creates the footnote text
% %\blfootnote{$^\ast$ Equal Contribution.}

% Heading arguments are {volume}{year}{pages}{submitted}{published}{author-full-names}

% \jmlrheading{1}{2000}{1-48}{4/00}{10/00}{Marina Meil\u{a} and Michael I. Jordan}

% Short headings should be running head and authors last names

% \ShortHeadings{Meta Learning of Interface Conditions for Multi-Domain Physics-Informed Neural Networks}
\firstpageno{1}

\begin{document}

\title{Meta-Learning with Adjoint Methods}

\author{\name Shibo Li \email shibo@cs.utah.edu \\
       \addr School of Computing\\
       University of Utah
       \AND
       \name Zheng Wang \email wzhut@cs.utah.edu\\
       \addr School of Computing\\
       University of Utah
       \AND
       \name Akil Narayan \email akil@sci.utah.edu\\
       \addr Department of Mathematics, Scientific Computing and Imaging Institute\\
       University of Utah
       \AND
       \name Robert M. Kirby \email kirby@cs.utah.edu \\
       \addr School of Computing, Scientific Computing and Imaging Institute\\
       University of Utah
       \AND
       \name Shandian Zhe \email zhe@cs.utah.edu \\
       \addr School of Computing\\
       University of Utah }

% \editor{Leslie Pack Kaelbling}


\maketitle

version https://git-lfs.github.com/spec/v1
oid sha256:9463f7b4c5a3711714e4014f739fa4718208a3e963fd584b4d97594ac45c38b9
size 4347


\section{Introduction}

Machine learning fundamentally relies on the availability of data, which can
be sensitive or confidential.
It is now well-known that preventing learned models from leaking information
about individual training points requires particular attention
\citep{shokri2017Membership}.
A standard approach for training models while provably controlling the amount of
leakage is to solve an empirical risk minimization (ERM) problem
under a differential privacy (DP) constraint \citep{chaudhuri2011Differentially}.
In this work, we aim to design a differentially private algorithm which
approximates the solution to a composite ERM problem of the form:
\begin{align}
  \label{eq:dp-erm}
  w^* \in
  \argmin_{w \in \mathbb{R}^p}
  \left\{
  \frac{1}{n} \sum_{i=1}^n \ell(w; d_i) + \psi(w)
  \right\}
  \enspace,
\end{align}
where $D = (d_1, \dots, d_n)$
is a dataset of $n$ samples drawn from a universe $\cX$,
$\ell: \RR^p \times \cX \rightarrow \RR$ is a loss function which is convex
and smooth in $w$, and
$\psi: \RR^p \rightarrow \RR$ is a convex regularizer which is separable (\ie
$\psi(w) = \sum_{j=1}^p \psi_j(w_j)$) and typically nonsmooth (\eg
$\ell_1$-norm).

Differential privacy constraints induce a trade-off between the privacy and
the utility (i.e., optimization error) of the solution of~\eqref{eq:dp-erm}.
This trade-off was made explicit by \citet{bassily2014Private}, who derived
lower bounds on the achievable error given a fixed privacy budget.
To solve the DP-ERM problem in practice, the most popular approaches are based
on Differentially Private variants of Stochastic Gradient Descent (DP-SGD)
\citep{bassily2014Private,abadi2016Deep,wang2017Differentially}, in which
random perturbations are added to the (stochastic) gradients.
\citet{bassily2014Private} analyzed DP-SGD in the non-smooth DP-ERM setting,
and \citet{wang2017Differentially} then proposed an efficient DP-SVRG
algorithm for composite DP-ERM.
Both algorithms match known lower bounds.
SGD-style algorithms perform well in a wide variety of settings, but
also have some flaws: they either require small (or decreasing) step
sizes or variance reduction schemes to guarantee convergence, and they
can be slow when gradients' coordinates are imbalanced.
These flaws propagate to the private counterparts of these
algorithms.
Despite a few attempts at designing other differentially private solvers for
ERM under different setups
\citep{talwar2015Nearly,damaskinos2021Differentially}, the differentially
private optimization toolbox remains limited, which undoubtedly restricts the
resolution of practical problems.



In this paper, we propose and analyze a Differentially Private proximal
Coordinate
Descent algorithm (DP-CD), which performs updates based on perturbed
coordinate-wise gradients (\ie partial derivatives).  Coordinate
Descent (CD) methods have encountered a large success in non-private
machine learning due to their simplicity and effectiveness
\citep{liu2009Blockwise,friedman2010Regularization,chang2008Coordinate,sardy2000Block},
and have seen a surge of practical and theoretical interest in the
last decade \citep{Nesterov12,wright2015Coordinate,shi2017Primer,
  richtarik2014Iteration,fercoq2014Accelerated,tappenden2016Inexact,
  hanzely2020Variance,nutini2015Coordinate,karimireddy2019Efficient}.
In contrast to SGD, they converge with constant step sizes that adapt to
the coordinate-wise smoothness of
the objective. Additionally, CD updates naturally tend to
have a lower sensitivity. Operating with partial gradients thus enables
our private algorithm to reduce the perturbation required to
guarantee privacy without resorting to
amplification by
subsampling \citep{Balle_subsampling,mironov2019Enyi}.


We propose a novel analysis of proximal CD with perturbed gradients to
derive optimal upper bounds on the privacy-utility trade-off achieved
by DP-CD.
We prove a
recursion on distances of CD iterates to an optimal point that keeps track of
coordinate-wise regularity
constants in a tight manner and allows to use
large, constant step sizes that
yield high utility. Our results highlight the fact that DP-CD
can exploit imbalanced gradient coordinates to outperform DP-SGD.
They also improve upon known convergence rates for inexact CD in the
non-private setting
\citep{tappenden2016Inexact}.
We assess the optimality of DP-CD by deriving lower bounds
that capture coordinate-wise Lipschitz regularity measures, and show that
DP-CD matches those bounds up to logarithmic factors.
Our lower bounds also suggest interesting perspectives for future work on
DP-CD algorithms.

Our theoretical results
have important consequences for practical
implementations, which heavily rely on gradient clipping to achieve good
utility.
In contrast to DP-SGD, DP-CD requires to set \emph{coordinate-wise} clipping
thresholds, which can lead to impractical coordinate-wise hyperparameter tuning.
We instead propose a simple rule for adapting these thresholds from a
single hyperparameter. We also show how the coordinate-wise smoothness
constants used by DP-CD can be
estimated privately. We validate our theory with numerical
experiments on real and synthetic datasets. These experiments further
show that even in balanced problems, DP-CD can still improve over
DP-SGD, confirming the relevance of DP-CD for DP-ERM.

Our main contributions can be summarized as follows:
\begin{enumerate}
  \item We propose the first proximal CD algorithm for composite DP-ERM,
        formally prove its utility, and highlight regimes where it outperforms DP-SGD.
  \item We show matching lower bounds under coordinate-wise regularity
        assumptions.
      \item We give practical guidelines to use DP-CD, and show its
        relevance through numerical experiments.
\end{enumerate}


The rest of this paper is organized as follows.
We first describe some mathematical background in
\Cref{sec:preliminaries}.
In \Cref{sec:diff-priv-coord}, we present our DP-CD algorithm,
show that it satisfies DP, establish utility guarantees, and
compare these guarantees with those of DP-SGD.
In \Cref{sec:utility-lower-bounds}, we derive lower bounds under
coordinate-wise regularity assumptions, and
show that DP-CD can match them. \Cref{sec:dp-cd-practice} discusses practical
questions related to gradient clipping and the private estimation of
smoothness constants.
\Cref{sec:numerical-experiments} presents our numerical experiments,
comparing DP-CD and DP-SGD on LASSO and $\ell_2$-regularized
logistic regression problems. %
Finally, we review existing work in
\Cref{sec:related-works}, and conclude with promising lines of future work in
\Cref{sec:conclusion-and-discussion}.


version https://git-lfs.github.com/spec/v1
oid sha256:773f8cd61d0a6d6593bcf6a674cb40fe521472ae453fc90bb9522717b1539912
size 18018

\section{Related Work}
\label{related_work}

\subsection{Soft underwater robots}
Soft robots are difficult to optimally design and control when compared to their rigid counterparts due to the infinite dimensionality of their compliant structures.
Due to this modeling complexity, an experienced designer must hand craft each design guided by intuition, experiments, and approximate models.
Marchese et al. offer approaches to designing and fabricating soft fluidic elastomer robots, the type of robot we are also using in this work~\cite{marchese2015recipe}.
%
Katzschmann et al. present the design, fabrication, control, and testing of a soft robotic fish with interior cavities that is hydraulically actuated. Their manually designed robot can swim at multiple depths and record aquatic life in the ocean~\cite{katzschmann2016hydraulic,katzschmann2018exploration}. 
Zhu et al. manually optimize the swimming performance of their robotic fish, Tunabot~\cite{zhu2019tuna}. The authors measured kinematics, speed, and power at increasing flapping frequencies to quantify swimming performance and find agreement in performance between real fish and their Tunabot over a wide range of frequencies.
Zheng et al. propose to design soft robots by pre-checking controllability during the numerical design phase~\cite{zheng2019controllability}. FEM is used to model the dynamics of cable-driven parallel soft robot and a differential geometric method is applied to analyze the controllability of the points of interest.
Katzschmann et al.~\cite{katzschmann2019dynamically} manually tweak the material parameters of their reduced-order FEM~\cite{thieffry2018control} with an experimental soft robotic arm to perform dynamic closed-loop control.
Van et al. present a DC motor driven soft robotic fish which is optimized for speed and efficiency based on experimental, numerical and theoretical investigation into oscillating propulsion~\cite{van2020biomimetic}.
Wolf et al. use a pneumatically-actuated fish-like stationary model to investigate how parameters like stiffness, strength, and frequency affect thrust force generation~\cite{wolf2020fish}. Wolf et al. measure thrust, side forces, and torques generated during propulsion and use a statistical linear model to examine the effects of parameter combinations on thrust generation; they show that both stiffness and frequency substantially affect swimming kinematics.
We are not aware of any work that uses a fast differentiable FEM simulation environment to learn material parameters for soft robotic fish using a bollard-pull style experimental setup.

\subsection{Differentiable soft-body simulators}
Our work is also relevant to the recent developments of robotic simulators, particularly for soft robots.
Geilinger et al. \cite{geilinger2020add} present a differentiable multi-body dynamics solver that is able to handle frictional contact for rigid and deformable objects.
Coevoet et al. \cite{coevoet2017software} notably present a non-differentiable framework for modeling, simulation, and control of soft-bodied robots using continuum mechanics for modeling the robotic components and using Lagrange multipliers for boundary conditions like actuators and contacts.
Most related to our work are the recent works on differentiable soft-body and fluid simulators~\cite{du2020stokes,du2021diffpd,hahn2019real2sim,hu2019difftaichi,hu2019chainqueen,huang2021plasticine,ma2021diffaqua}. 
These papers develop numerical methods for computing gradients in a traditional simulators. Furthermore, they demonstrate the power of gradient information in robotics applications, e.g., system identification or trajectory optimization. Most of the works present simulation results only, with ChainQueen~\cite{hu2019chainqueen} and Real2Sim~\cite{hahn2019real2sim} being two notable exceptions that discuss real-world soft-robot applications.
Notably, \cite{hahn2019real2sim} optimizes visco-elastic material parameters of a finite element simulation to approximate the dynamic deformations of real-world soft objects, such as an open-loop controlled tendon-driven crawling robot.
Bern et al. \cite{bern2020soft} have also demonstrated the use of differentiable simulation to learn from a quasi-static data set for the purpose of optimizing open-loop control inputs.
Dubied et al. \cite{dubied2022sim} is the most recent example that demonstrates sim2real matching for a soft robotic fish tail, shows system identification on a passive structure for just the Young's modulus, and investigates the mismatch in damping between reality and simulation. In this previous work, the fish tail actuation is simulated using a simplified muscle model and only one design is shown whereas in this paper, the pressure boundary condition is simulated exactly as fabricated for each pneumatic chamber geometry for three different designs. Simulating the pneumatic chambers improves accuracy and allows for physically-plausible Young's moduli and Poisson ratios to be identified. In this current work, we further demonstrate that the gradient-based optimization can be carried out to higher dimensional design spaces that include more than one material parameter.

\subsection{Hydrodynamic Surrogates}
For underwater soft robots, the challenge of simulation is exacerbated by the hydrodynamic interaction with the soft body.
Several previous works tackle the fluid-structure interaction problem through different methods, including heuristic hydrodynamics~\cite{du2021underwater,ma2021diffaqua,min2019softcon}, physically-informed neural network approaches~\cite{wandel_learning_2021}, and data-driven learning approaches~\cite{chen2018neural}.

Compared to these previous methods that simulate underwater soft robots such as~\cite{du2021underwater}, our work models pneumatic actuation using the exact chamber geometry rather than artificial muscles facilitating greater accuracy at large deformations (see \Cref{fig:pneumatic_fish_tail}), uses a neural network thrust predictor rather than approximate analytical or heuristic hydrodynamics, and presents a more sophisticated hardware pipeline that can be used to validate simulation. 
version https://git-lfs.github.com/spec/v1
oid sha256:0b11aff639681f19e4d839378bd9666abc87904258786fd8833212e9667d4530
size 26258

\section{Conclusion}
\label{conclusion}
We present an experimentally-verified simulation framework that can be used to accurately predict the deformations of a pneumatically actuated fish tail with a flexible spine.
Our pipeline can accurately learn material parameters from a quasi-static data sets without having to do expensive and time-consuming material testing. It also eliminates the need to do manual tuning of material constants to get accurate simulation results. The parameters we found are not only within typical range of measured material parameters for our materials, but can be used to successfully predict the behavior of dynamic experiments for different pressure actuation amplitudes and frequencies to within $3\%$ positional error normalized to a actuator length of \SI{10}{cm}. Although we use an isotropic corotated material, which is linear elastic, we find that this model is more sufficient to model large deformations on average giving acceptable displacement results for our engineering application. In these experiments, the damping of the material and the hydrodynamic effects are found to be negligible. This is because the actuation pressures used dominate the deformation compared to losses and hydrodynamic pressure. 

We show a data-driven approach can be used to do simple prediction on a useful performance metric such as thrust force given a suitable hardware setup. However, more work is needed to produce a more robust thrust predictor if the morphology of the actuator changes substantially. We claim that for small design changes such as the choice of silicone or the number of internal chambers this framework can be used to quickly assess the relative merits of each design with a relatively sparse data set of approximately 30 types of different actuation signals.

Our aim is to further progress towards a systematic method by which soft roboticists can simulate and optimize their designs and controllers, whether they be soft fish, manipulators, or other flavors of soft robots. A fast and physically-verified co-optimization method of design and control is the goal.

% % Acknowledgements should go at the end, before appendices and references

% \acks{We would like to acknowledge support for this project
% from the National Science Foundation (NSF grant IIS-9988642)
% and the Multidisciplinary Research Program of the Department
% of Defense (MURI N00014-00-1-0637). }

% % Manual newpage inserted to improve layout of sample file - not
% % needed in general before appendices/bibliography.

\newpage

% \input{supp}

% % Note: in this sample, the section number is hard-coded in. Following
% % proper LaTeX conventions, it should properly be coded as a reference:

% %In this appendix we prove the following theorem from
% %Section~\ref{sec:textree-generalization}:

% In this appendix we prove the following theorem from
% Section~6.2:

% \noindent
% {\bf Theorem} {\it Let $u,v,w$ be discrete variables such that $v, w$ do
% not co-occur with $u$ (i.e., $u\neq0\;\Rightarrow \;v=w=0$ in a given
% dataset $\dataset$). Let $N_{v0},N_{w0}$ be the number of data points for
% which $v=0, w=0$ respectively, and let $I_{uv},I_{uw}$ be the
% respective empirical mutual information values based on the sample
% $\dataset$. Then
% \[
% 	N_{v0} \;>\; N_{w0}\;\;\Rightarrow\;\;I_{uv} \;\leq\;I_{uw}
% \]
% with equality only if $u$ is identically 0.} \hfill\BlackBox

% \noindent
% {\bf Proof}. We use the notation:
% \[
% P_v(i) \;=\;\frac{N_v^i}{N},\;\;\;i \neq 0;\;\;\;
% P_{v0}\;\equiv\;P_v(0)\; = \;1 - \sum_{i\neq 0}P_v(i).
% \]
% These values represent the (empirical) probabilities of $v$
% taking value $i\neq 0$ and 0 respectively.  Entropies will be denoted
% by $H$. We aim to show that $\fracpartial{I_{uv}}{P_{v0}} < 0$....\\

% {\noindent \em Remainder omitted in this sample. See http://www.jmlr.org/papers/ for full paper.}


% \vskip 0.2in
% \bibliography{sample}

% \bibliographystyle{apalike}
\bibliography{AMAML}

\end{document}