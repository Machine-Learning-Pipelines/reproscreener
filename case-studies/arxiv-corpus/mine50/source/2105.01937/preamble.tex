%
% this file defines packages to be used and new commands
%
\usepackage{comment}
\usepackage{caption}
\usepackage{stackengine} % required for 'ifdef. this package is already used by tog but not by iccv

% all \ifxxx commands should be false. the true values should be defined in the main files.

\ifundef{\ifanonymous}
{\newif\ifanonymous \anonymousfalse}
{}

\ifundef{\ifdraft}
{\newif\ifdraft \draftfalse}
{}

\ifundef{\ifeccv}
{\newif\ifeccv \eccvfalse}
{}

\ifundef{\ifcvpr}
{\newif\ifcvpr \cvprfalse}
{}

\ifundef{\ifcvprreview}
{\newif\ifcvprreview \cvprreviewfalse}
{}

% \ifcvprreview
% {\global\anonymoustrue} % must be global. maybe because of the makeletter env?
% \else
% {\global\anonymousfalse}
% \fi

% \ifappendix tells us whether the textual sup mat is in the appendix or in a separate doc 
\ifundef{\ifappendix}
{\newif\ifappendix \appendixfalse}
{}

\ifundef{\ifarxiv}
{\newif\ifarxiv \arxivfalse}
{}

\ifundef{\ificcv}
{\newif\ificcv \iccvfalse}
{}

\ifundef{\iftog}
{\newif\iftog \togfalse}
{}

\ifundef{\ificcvfinal}
{\newif\ificcvfinal \iccvfinalfalse}
{}

\ifarxiv
\appendixtrue
\fi

\iftog
\appendixtrue
\fi

\ifanonymous
{\global\iccvfinalfalse}
\else
{\global\iccvfinaltrue}
\fi

\ifcvpr{
\newcommand{\citet}{\cite}
\newcommand{\keywords}{\null}
}\fi

\ifeccv
\else
\usepackage{times}
\fi
% \usepackage[numbers,sort]{natbib}
\usepackage{ifthen}
\usepackage{cancel}
\usepackage{epsfig}
\usepackage{graphicx}
\usepackage{amsmath}
\iftog {} \else \usepackage{amssymb} \fi % fails for TOG style

\usepackage{xcolor}
\usepackage{enumitem}
\usepackage{wrapfig}
\usepackage{float}
\usepackage{xspace} % needed for \etal etc.
% \usepackage{hyperref}
\usepackage{orcidlink}

\newcommand{\bK}{{\bf K}}
\newcommand{\bQ}{{\bf Q}}
\newcommand{\bW}{{\bf W}}
\newcommand{\bV}{{\bf V}}
\newcommand{\bJ}{{\bf J}}
\newcommand{\bP}{{\bf P}}
\newcommand{\bL}{{\bf L}}
\newcommand{\bT}{{\bf T}}
\newcommand{\bD}{{\bf D}}
\newcommand{\bF}{{\bf F}}
\newcommand{\bG}{{\bf G}}
\newcommand{\bI}{{\bf I}}
\newcommand{\bc}{{\bf c}}
\newcommand{\bff}{{\bf f}}
\newcommand{\bp}{{\bf p}}
\newcommand{\bq}{{\bf q}}
\newcommand{\br}{{\bf r}}
\newcommand{\bss}{{\bf s}}
\newcommand{\bv}{{\bf v}}
\newcommand{\bk}{{\bf k}}
\newcommand{\Loss}{\mathcal{L}}
\newcommand{\mm}{\mathcal{M}}
\newcommand{\mms}{\mathcal{S}}
\newcommand{\Dp}{\bbd_{\text{p}}}
\newcommand{\Ds}{\bbd_{\text{s}}}
\newcommand{\cc}{\mathcal{C}}
\newcommand{\nn}{\mathcal{N}}
\newcommand{\pp}{\mathcal{P}}
\newcommand{\qq}{\mathcal{Q}}
\newcommand{\bbe}{\mathbb{E}}
\newcommand{\bbr}{\mathbb{R}}

\iftog 
\else
\newcommand{\shortcite}{\cite}
\fi

%%%%%%% or's imports
\usepackage{url}
\usepackage{graphics}
% \usepackage{layouts} % causes two warnings, not sure this package is needed
\usepackage[normalem]{ulem} % ulem defines \sout (strikethrough). normalem disables the change of emph from italic to underline.
\usepackage{multirow}
\usepackage{tabu,stackengine}
\usepackage{wrapfig}
% \usepackage{floatrow}
\usepackage{booktabs}
\usepackage{soul}
% Table float box with bottom caption, box width adjusted to content
% \newfloatcommand{capbtabbox}{table}[][\FBwidth]
% \usepackage{caption}
% \usepackage{subcaption}
\usepackage{makecell}
\ifcvpr
\else
\usepackage{cleveref}
\crefname{section}{Sec.}{Secs.}
\Crefname{section}{Section}{Sections}
\Crefname{table}{Table}{Tables}
\crefname{table}{Tab.}{Tabs.}
\fi

\newcommand{\todo}[1]{{\color{red}#1}}
\newcommand{\tbd}{{\color{red}xxx}}
\newcommand{\bluebold}[1]{{\textbf{\color{blue} #1}}}
\newcommand{\graybluebold}[1]{{\textbf{\color{blue!50} #1}}} % 50% blue and 50% white, to be used when cam params are perturbed

\definecolor{applegreen}{rgb}{0.55, 0.71, 0.0}
\definecolor{burgundy}{rgb}{0.5, 0.0, 0.13}
\definecolor{calpolypomonagreen}{rgb}{0.12, 0.3, 0.17}

\ifdraft
\newcommand{\setcolor}[1]{\color{#1}}
\newcommand{\sr}[1]{{\color{violet} #1}}
\newcommand{\sroo}[1]{{\color{calpolypomonagreen} #1}}
\newcommand{\srrep}[2]{\sout{#1} \sr{#2}}
\newcommand{\bg}[1]{{\color{orange} #1}}
\else
\newcommand{\setcolor}[1]{}
\newcommand{\sr}[1]{{#1}}
\newcommand{\sroo}[1]{#1}
\newcommand{\srrep}[2]{#2}
\newcommand{\bg}[1]{#1}
\fi
\newcommand{\srr}[1]{{\color{red}\textbf{SR:} #1}}
\newcommand{\srm}[1]{{\color{violet}\textbf{SR:} #1}}
\newcommand{\sro}[1]{{\color{calpolypomonagreen}\textbf{SR:} #1}}

\newcommand{\dcc}[1]{{\color{red}\textbf{DC:} #1}}
\newcommand{\dc}[1]{{\color{red} #1}}

\newcommand{\rgc}[1]{{\color{cyan}\textbf{RG} #1}}
\newcommand{\rg}[1]{{\color{cyan} #1}}

\newcommand{\brr}[1]{{\color{red}\textbf{BG:} #1}}
\newcommand{\brrep}[2]{\sout{#1} \bro{#2}}
%\newcommand{\brchange}[2]{\sout{~\mbox{#1}} \bro{#2}}
\newcommand{\bro}[1]{{\color{orange}\textbf{BG:} #1}}
\newcommand{\brchange}[2]{\sout{#1} \bro{#2}}

\newcommand{\gum}[1]{{\color{violet}\textbf{GA:} #1}}
\newcommand{\gur}[1]{{\color{red}\textbf{GA:} #1}}
\newcommand{\gurep}[2]{\sout{#1} \bro{#2}}

\newcommand{\algoname}{FLEX}
\newcommand{\projectpage}{\url{https://briang13.github.io/FLEX}}

\setlength{\abovedisplayshortskip}{-30pt}
\setlength{\belowdisplayshortskip}{-30pt}
\setlength{\abovedisplayskip}{-30pt}
\setlength{\belowdisplayskip}{-30pt}


\makeatletter
\iftog
\newcommand{\sectiontinyvert}{\section}
\newcommand{\paragraphtinyvert}{\paragraph}
\newcommand{\paragraphnovert}{\paragraph}
\newcommand{\subparagraphnovert}{\subparagraph}
\else

% \renewcommand\section{\@startsection{section}{1}{\z@}%
%                       {-18\p@ \@plus -4\p@ \@minus -4\p@}%
%                       {12\p@ \@plus 4\p@ \@minus 4\p@}%
%                       {\normalfont\large\bfseries\boldmath
%                         \rightskip=\z@ \@plus 8em\pretolerance=10000 }}
% \renewcommand\subsection{\@startsection{subsection}{2}{\z@}%
%                       {-18\p@ \@plus -4\p@ \@minus -4\p@}%
%                       {8\p@ \@plus 4\p@ \@minus 4\p@}%
%                       {\normalfont\normalsize\bfseries\boldmath
%                         \rightskip=\z@ \@plus 8em\pretolerance=10000 }}

% \renewenvironment{abstract}{%
%       \list{}{\advance\topsep by0.35cm\relax\small
%       \leftmargin=1cm
%       \labelwidth=\z@
%       \listparindent=\z@
%       \itemindent\listparindent
%       \rightmargin\leftmargin}\item[\hskip\labelsep
%                                     \bfseries\abstractname]}
%     {\endlist}

% \renewenvironment{equation*}{%
%   \mathdisplay@push
%   \st@rredtrue \global\@eqnswfalse
%   \color{black}
%   \mathdisplay{equation*}%
% }{%
%   \endmathdisplay{equation*}%
%   \mathdisplay@pop
%   \ignorespacesafterend
% }


\newenvironment{myequation}{%
% \list{}{
%\topsep by-70pt 
% \smallskip
% \vspace{-40\p@}
  \incr@eqnum
  \mathdisplay@push
  \st@rredfalse \global\@eqnswtrue
  \mathdisplay{equation}%
% }
}{%
  \endmathdisplay{equation}%
  \mathdisplay@pop
  \ignorespacesafterend
%   \endlist
}

% \newenvironment{equation}{%
%   \mathdisplay@push
%   \st@rredfalse \global\@eqnswtrue
%   \mathdisplay{equation}%
%   \incr@eqnum\mathopen{}%
% }{%
%   \endmathdisplay{equation}%
%   \mathdisplay@pop
%   \ignorespacesafterend
% }


\newenvironment{myequation*}{%
\list{}{\topsep by-7pt
\mathdisplay@push
  \st@rredtrue \global\@eqnswfalse
  \mathdisplay{equation}%
}}{%
  \endmathdisplay{equation}%
  \mathdisplay@pop
  \ignorespacesafterend
  \endlist
}

\newenvironment{myabstract}{%
      \list{}{\advance\topsep by-7pt\relax\small
      \leftmargin=0.95cm
      \labelwidth=\z@
      \listparindent=\z@
      \itemindent\listparindent
      \rightmargin\leftmargin}\item[\hskip\labelsep
                                    \bfseries\abstractname]}
    {\endlist}

\newcommand{\sectiontinyvert}{
  \@startsection{section}               % name 
  {1}                                   % level
  {\z@}                                 % indent
  {10pt \@plus 3pt \@minus 0pt}         % before skip
  {5pt \@plus 2pt \@minus 0pt}          % after skip
  {\normalfont\large\bfseries\boldmath\rightskip=\z@ \@plus 8em\pretolerance=10000}          % style
}

\newcommand{\subsectiontinyvert}{
  \@startsection{subsection}            % name 
  {2}                                   % level
  {\z@}                                 % indent
  {10pt \@plus 3pt \@minus 0pt}         % before skip
  {5pt \@plus 2pt \@minus 0pt}          % after skip
  {\normalfont\normalsize\bfseries\boldmath\rightskip=\z@ \@plus 8em\pretolerance=10000}     % style
}
\newcommand\subsubsectiontinyvert{
    \@startsection{subsubsection}
    {3}
    {\z@}
    {10pt \@plus 3pt \@minus 0pt}%
    {-0.5em \@plus -0.22em \@minus -0.1em}%
    {\normalfont\normalsize\bfseries\boldmath}}
    
% \renewcommand\paragraph{\@startsection{paragraph}{4}{\z@}%
%                       {-12\p@ \@plus -4\p@ \@minus -4\p@}%
%                       {-0.5em \@plus -0.22em \@minus -0.1em}%
%                       {\normalfont\normalsize\itshape}}
\newcommand{\paragraphtinyvert}{%
  \@startsection{paragraph}{4}%
  {\z@}{1ex \@plus 0.0ex \@minus 0.2ex}{-1em}% was "plus 0.2 ex"
  {\normalfont\normalsize\bfseries\boldmath} % like subsubsection in llncs
}
\newcommand{\paragraphnovert}{%
  \@startsection{paragraph}{4}%
  {\z@}{0ex \@plus 0ex \@minus 0ex}{-0.5 em}%
  {\normalfont\normalsize\bfseries}%
}
\newcommand{\subparagraphnovert}{%
  \@startsection{subparagraph}{5}%
  {3ex}{0ex \@plus 0ex \@minus 0ex}{-1em}%
  {\normalfont\normalsize\bfseries}%
}

% If the paper title is too long for the running head, you can set
% an abbreviated paper title here
% \newcommand{\orcid}[1]{\href{https://orcid.org/#1} {\protect\includegraphics[width=8pt]{./images/ORCID-iD_icon-128x128.png}}}
% \newcommand{\orcid}[1]{\orcidID{#1}}
\newcommand{\orcid}[1]{\orcidlink{#1}}

\fi

\ificcv
\else
% the following are copied from iccv.sty
% Add a period to the end of an abbreviation unless there's one
% already, then \xspace.
\DeclareRobustCommand\onedot{\futurelet\@let@token\@onedot}
\def\@onedot{\ifx\@let@token.\else.\null\fi\xspace}

\def\eg{\emph{e.g}\onedot} \def\Eg{\emph{E.g}\onedot}
\def\ie{\emph{i.e}\onedot} \def\Ie{\emph{I.e}\onedot}
\def\cf{\emph{c.f}\onedot} \def\Cf{\emph{C.f}\onedot}
\def\etc{\emph{etc}\onedot} \def\vs{\emph{vs}\onedot}
\def\wrt{w.r.t\onedot} \def\dof{d.o.f\onedot}
\def\etal{\emph{et al}\onedot}
\fi

\makeatother

