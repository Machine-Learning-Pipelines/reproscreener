\section{Discussion}
\label{discussion}

\subsection{DiffPD Simulation}
We conduct our simulation experiments based on a linear elastic corotated material setting. This energy model is popular for physical simulation and animation, but it is not the most accurate choice for closing the sim-to-real gap when deformations larger than about 200\% occur. This problem can be fixed by introducing a more sophisticated elastic energy model, for example a Neo-Hookean material.

\subsection{Experimental Setup}
Although the linear bearings in the experimental setup are well lubricated, they still contribute some friction. In the scope of this work, this friction force is assumed to be negligible. Furthermore, due to the rather small size of the tank, a standing wave persists in the tank even quite long after interacting with it. This can lead to a disturbance force on the load cell. However, since the frequency of this disturbance is known, it can be filtered out during post-processing.

% put in discussion about measurement dynamics

\subsection{Hydrodynamics}


% \subsection{Future Work}
